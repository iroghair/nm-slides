\part{Systems of linear equations I - Introduction}
\section{Introduction}
\subsection*{General}
\begin{frame}[label=contentslin1]
  \frametitle{Today's outline}
  \mode<beamer>{
    \only<1>{\tableofcontents}
  }
  \only<2>{\tableofcontents[currentsection,currentsubsection]}
\end{frame}

\begin{frame}
  \frametitle{Overview}
  \begin{block}{Goals}
    \begin{itemize}
      \item Different ways of looking at a system of linear equations
      \item Determination of the inverse, determinant and the rank of a matrix
      \item The existence of a solution to a set of linear equations
  \end{itemize}
  \end{block}
\end{frame}

% 
\subsection*{Linear systems}
{\nologo
\frame{\frametitle{Different views of linear systems}
  \begin{columns}
    \column{0.4\textwidth}
  \begin{itemize}
    \vskip-1ex
    \item Separate equations:\vspace{-1em}\begin{align*}
      x + y +  z &= 4 \\
      2x + y + 3z &= 7 \\
      3x + y + 6z &= 5
    \end{align*} \pause
    \item Matrix mapping $Mx=b$:\vspace{-1ex} \[ 
      \begin{bmatrix}
	1 & 1 & 1\\ 
	2 & 1 & 3\\ 
	3 & 1 & 6
      \end{bmatrix}
      \begin{bmatrix}
	x \\
	y \\
	z 
      \end{bmatrix} = 
      \begin{bmatrix}
	4 \\
	7 \\
	5 
      \end{bmatrix} 
      \]
      \pause
    \item Linear combination:\vspace{-1ex} \[ 
      x \begin{bmatrix}
	1\\
	2\\
	3
	\end{bmatrix} +
      y \begin{bmatrix}
	1\\
	1\\
	1
	\end{bmatrix} +
      z \begin{bmatrix}
	1\\
	3\\
	6
	\end{bmatrix} =
	\begin{bmatrix}
	4\\
	7\\
	5
      \end{bmatrix} 
    \]
    \end{itemize}
    \column{0.6\textwidth}
    \pause
    \centering\includegraphics[width=\columnwidth]{planes_xing.png}
  \end{columns}
}
}

\section{Solving a linear system}
\subsection*{Solving}
\againframe<2>{contentslin1}
\frame{
  \frametitle{Inverse of a matrix}
  \begin{itemize}
    \item The inverse $M^{-1}$ is defined such that:
   \[ MM^{-1} = I \quad{} \text{and} \quad{} M^{-1}M=I\]
   \item Use the inverse to solve a set of linear equations:
   \begin{align*}
    M\vec{x} &= \vec{b} \\
    M^{-1}M\vec{x} &= M^{-1}\vec{b} \\
    I\vec{x} &= M^{-1}\vec{b} \\
    \vec{x} &= M^{-1}\vec{b}
   \end{align*}
  \end{itemize}
}

{\nologo
\begin{frame}[fragile]
  \frametitle{Solving a linear system in Python using the inverse}
  \begin{itemize}
    \item Create the matrix:
    \begin{lstlisting}
>>> A = np.array([[1, 1, 1], [2, 1, 3], [3, 1, 6]])    
    \end{lstlisting}\pause
    \item Create solution vector:
    \begin{lstlisting}
>>> b = np.array([4, 7, 5]) 
    \end{lstlisting}\pause
    \item Get the matrix inverse:
    \begin{lstlisting}
>>> Ainv = np.linalg.inv(A) 
    \end{lstlisting}\pause
    \item Compute the solution:
    \begin{lstlisting}
>>> x = np.dot(Ainv, b)   
    \end{lstlisting}\pause
    \item Python's internal direct solver:
    \begin{lstlisting}
>>> x = np.linalg.solve(A, b)
    \end{lstlisting}
    \item \tikzmarkin[txt=style white]{bb}These are black boxes! We are going over some methods later!\tikzmarkend{bb}
  \end{itemize}
\end{frame}
}

\begin{frame}<handout:0|beamer:1->[fragile]
  \frametitle{Exercise: performance of inverse computation}
  Create a script that generates matrices with random elements of various sizes $N\times N$ (e.g. values of $N\in\left\{10,20,50,100,200,\ldots,5000,10000\right\}$). Compute the inverse of each matrix, and use \lstinline$import time$ and \lstinline$time.time()$ to see the computing time for each inversion. Plot the time as a function of the matrix size $N$. \pause
    \begin{lstlisting}[language=Python, basicstyle=\tiny\ttfamily]
import numpy as np
import matplotlib.pyplot as plt
import time

# Generate random matrices of various sizes 's'. 
# Invert the matrices and store the time required 
# for the inversion. Plot the times vs 's'
s = np.array([10, 20, 50, 100, 200, 500, 1000, 2000, 5000, 10000])
t_inv = []
for n in s:
    print(f'Working on size {n}')
    A = np.random.rand(n, n)
    start_time = time.time()
    Ainv = np.linalg.inv(A)
    t_inv.append(time.time() - start_time)

plt.loglog(s, t_inv)
plt.xlabel('N')
plt.ylabel('Time [s]')
plt.show()
    \end{lstlisting}
\end{frame}


\begin{frame}<beamer:0|handout:1>[fragile]
  \frametitle{Exercise: performance of inverse computation}
  Create a script that generates matrices with random elements of various sizes $N\times N$ (e.g. values of $N\in\left\{10,20,50,100,200,\ldots,5000,10000\right\}$). Compute the inverse of each matrix, and use \lstinline$tic$ and \lstinline$toc$ to see the computing time for each inversion. Plot the time as a function of the matrix size $N$.
  \vskip2em
  \begin{hints}
  Hints:
  \begin{itemize}
      \item Create an array that contains the sizes of the systems $n$
      \item Loop over the array elements to:
      \begin{itemize}
          \item Create a random matrix of size $n\times n$
          \item Perform the matrix inversion
          \item Record the time required
      \end{itemize}
      \item Plot the time required for inversion vs size of the system on a double-log scale
  \end{itemize}
  \end{hints}
\end{frame}

{\nologo
\begin{frame}[fragile]
  \frametitle{Exercise: sample results}
  Each computer produces slightly different results because of background tasks, different matrices, etc. This is especially noticable for small systems.
  \begin{center}
  \begin{tikzpicture}
    \begin{loglogaxis}[
      xlabel={$N$},
      ylabel={Time [s]},
%       grid = major,
      width=0.6\textwidth, height=5.5cm]
     \addplot[graph,draw=scharlaken,thick,mark = x] table [y index={1}] {data/tictocINV.dat};
     \addplot [black,very thick] table {
	900 0.065
	11000 64.87
      } coordinate [pos=0.15] (A) % save two points on the regression line for drawing the slope triangle
        coordinate [pos=0.85] (B);
	\draw[thick] (A) -| (B)  % draw the opposite and adjacent sides of the triangle
        node [pos=0.25, anchor=north] {1} % label the horizontal line
        node [pos=0.75,anchor=west] {3};
     \end{loglogaxis}
  \end{tikzpicture}
  \end{center}
  \vskip-1em
  \tikzmarkin[txt=style white]{complexity}The time increases by 3 orders of magnitude, for every magnitude in $N$. The \emph{computational complexity} of matrix inversion scales with $\mathcal{O}(N^3)$!\tikzmarkend{complexity}
\end{frame}
}

\section{Existence of solution}
\subsection*{Rank of a matrix}
\begin{frame}[fragile]
  \frametitle{Solutions of linear systems}
 Rank of a matrix: the number of linearly independent columns (columns that can not be expressed as a linear combination of the other columns) of a matrix.
 \vskip1em
 \begin{columns}
  \column{0.5\textwidth}
  \[
   M = \begin{bmatrix}
        5 & 3 & 2 \\
        0 & 9 & 1 \\
        0 & 0 & 1
       \end{bmatrix}
  \]
  \begin{itemize}
   \item 3 independent columns
   \item In Python:
   \begin{lstlisting}
>>> numpy.linalg.matrix_rank(M)
   \end{lstlisting}
  \end{itemize}
  \column{0.5\textwidth}
    \[
   M = \left[\begin{array}{cccc}
        \tikzmarkin[mat=style tueturq]{c1}1 & \tikzmarkin[mat=style yellow]{c2}2 & \tikzmarkin[mat=style tuesteel]{c3}1 & \tikzmarkin[mat=style tueorange]{c4}0 \\
        0 & 0 & 1 & 1 \\
        0\tikzmarkend{c1} & 0\tikzmarkend{c2} & 0\tikzmarkend{c3} & 0\tikzmarkend{c4}
       \end{array}\right]
  \]
  \begin{itemize}
   \item \tikzmarkin[txt=style yellow]{c22}col 2\tikzmarkend{c22} $= 2 \times$ \tikzmarkin[txt=style tueturq]{c12}col 1\tikzmarkend{c12}
   \item \tikzmarkin[txt=style tueorange]{c42}col 4\tikzmarkend{c42} $=$ \tikzmarkin[txt=style tuesteel]{c32}col 3\tikzmarkend{c32} $-$ \tikzmarkin[txt=style tueturq]{c13}col 1\tikzmarkend{c13}
   \item 2 independent columns: rank = 2
  \end{itemize}
 \end{columns}
\end{frame}

\begin{frame}[fragile]
  \frametitle{Solutions of linear systems}
  The solution of a system of linear equations may or may not exist, and it may or may not be unique. Existence of solutions can be determined by comparing the rank of the Matrix $M$ with the rank of the augmented matrix $M_a$:
  \begin{lstlisting}
>>> numpy.linalg.matrix_rank(A)
>>> numpy.linalg.matrix_rank(np.column_stack((A,b))) # Concatenated matrices
  \end{lstlisting}
  Our system: $Mx = b$\\
  \[ 
    M = \begin{bmatrix}
    M_{11} & M_{12} & M_{13}\\ 
    M_{21} & M_{22} & M_{23}\\ 
    M_{31} & M_{32} & M_{33}
    \end{bmatrix} \text{,} b=\begin{bmatrix}b_1\\b_2\\b_3  \end{bmatrix} \Rightarrow 
    M_a =     \begin{bmatrix}
    M_{11} & M_{12} & M_{13} & b_1\\ 
    M_{21} & M_{22} & M_{23} & b_2\\ 
    M_{31} & M_{32} & M_{33} & b_3
    \end{bmatrix}
  \]
\end{frame}

\begin{frame}
 \frametitle{Existence of solutions for linear systems}
  For a matrix $M$ of size $n \times n$, and augmented matrix $M_a$:
 \begin{columns}
  \column{0.5\textwidth}
 \begin{itemize}
   \item $\text{Rank}(M) = n$:\\ Unique solution
 \end{itemize}
  \column{0.5\textwidth}
   \includegraphics[width=0.3\columnwidth]{Rank_1-solution}
 \end{columns}
 \pause
  \begin{columns}
  \column{0.5\textwidth}
 \begin{itemize}
   \item $\text{Rank}(M) = \text{Rank}(M_a) < n$:\\ Infinite number of solutions
 \end{itemize}
  \column{0.5\textwidth}
   \includegraphics[width=0.3\columnwidth]{Rank_Inf-solutions}
 \end{columns}
 \pause
  \begin{columns}
  \column{0.5\textwidth}
 \begin{itemize}
   \item $\text{Rank}(M) < n$, $\text{Rank}(M) < \text{Rank}(M_a)$:\\ No solutions
 \end{itemize}
  \column{0.5\textwidth}
   \includegraphics[width=0.3\columnwidth]{Rank_No-solutions1} \ \ 
   \includegraphics[width=0.3\columnwidth]{Rank_No-solutions2}
 \end{columns}
\end{frame}
 
\begin{frame}[fragile]
  \frametitle{Two examples}
  \[
    M=
    \begin{bmatrix}
      1 & 1 & 2\\
      0 & 3 & 1\\
      0 & 0 & 2
    \end{bmatrix}\quad
    b=
    \begin{bmatrix}
      17\\11\\4
    \end{bmatrix}
    \Rightarrow
    M_a=
    \begin{bmatrix}
      1 & 1 & 2 & 17\\
      0 & 3 & 1 & 11\\
      0 & 0 & 2 & 4
    \end{bmatrix}
  \]
  $\rank(M)=3=n \Rightarrow $ Unique solution \pause \\ \vfill
    \[
    M=
    \begin{bmatrix}
      1 & 1 & 2\\
      0 & 3 & 1\\
      0 & 0 & 0
    \end{bmatrix}\quad
    b=
    \begin{bmatrix}
      17\\11\\0
    \end{bmatrix}
    \Rightarrow
    M_a=
    \begin{bmatrix}
      1 & 1 & 2 & 17\\
      0 & 3 & 1 & 11\\
      0 & 0 & 0 & 0
    \end{bmatrix}
  \]
  $\rank(M)=\rank(M_a)=2<n \Rightarrow $ Infinite number of solutions
\end{frame} 

\section{Summary}
\subsection*{Summary}
\againframe<2>{contentslin1}

\begin{frame}[fragile]
  \frametitle{Summary}
  \begin{itemize}
    \item Linear equations can be written as matrices
    \item Using the inverse, the solution can be determined
    \item Introduced the concept of computational complexity: matrix inversion scales with $N^3$
    \item Existence of a solution depends on the rank of a matrix
  \end{itemize}
\end{frame}