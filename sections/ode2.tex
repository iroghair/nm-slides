\part{Ordinary differential equations II - Implicit methods and systems of ODEs}
\section{Introduction}
\subsection*{General}
\begin{frame}[label=contents_ode2]
  \frametitle{Today's outline}
  \mode<beamer>{
    \only<1>{\tableofcontents}
  }
  \only<2>{\tableofcontents[currentsection,currentsubsection]}
\end{frame}

\subsection{Backward Euler}
\begin{frame}
  \frametitle{Problems with Euler's method: instability}
  Consider the ODE:
  \[
    \frac{dy}{dx} = f(x,y(x)) \qquad \text{with} \qquad y(x=0) = y_0
  \]
  \vskip1em
  \pause
  First order approximation of derivative: $\frac{dy}{dx} = \frac{y_{i+1}-y_i}{\Delta x}$. 
  \vskip1em
  Where to evaluate the function $f$?
  \vskip1em
  \pause
  \begin{enumerate}
    \item Evaluation at $x_i$:  Explicit Euler method (forward Euler)
    \item Evaluation at $x_{i+1}$: Implicit Euler method (backward Euler)
  \end{enumerate}
\end{frame}

\begin{frame}
  \frametitle{Problems with Euler's method: instability -- forward Euler}
  Explicit Euler method (forward Euler):
    \begin{itemize}
      \item Use values at $x_i$: \\ $\frac{y_{i+1}-y_i}{\Delta x} = f(x_i,y_i) \Rightarrow y_{i+1} = y_i + h f(x_i,y_i)$. 
      \item This is an explicit equation for $y_{i+1}$ in terms of $y_i$.
      \item It can give instabilities with large function values.
    \end{itemize}
    \pause \vskip1em
    Consider the first order batch reactor: 
    \[
      \frac{dc}{dt} = -kc \Rightarrow c_{i+1} = c_i - k{\color{tuered}c_i}\Delta t \Rightarrow \frac{c_{i+1}}{c_i} = 1-k\Delta t
    \]
    \pause
    It follows that unphysical results are obtained for $k\Delta t \geq 1$!! \vskip1em
    \begin{block}{Stability requirement}
      \centering $k \Delta t < 1$\\ (but probably accuracy requirements are more stringent here!)
    \end{block}
\end{frame}
\begin{frame}
  \frametitle{Problems with Euler's method: instability -- backward Euler}
  Implicit Euler method (backward Euler):
    \begin{itemize}
      \item Use values at $x_{i+1}$: $\frac{y_{i+1}-y_i}{\Delta x} = f(x_{i+1},y_{i+1}) \Rightarrow y_{i+1} = y_i + h f(x_{i+1},y_{i+1})$. 
      \item This is an implicit equation for $y_{i+1}$, because it also depends on terms of $y_{i+1}$.
    \end{itemize}
    \pause \vskip1em
    Consider the first order batch reactor: 
    \[
      \frac{dc}{dt} = -kc \Rightarrow c_{i+1} = c_i - k{\color{tuered}c_{i+1}}\Delta t \Rightarrow \frac{c_{i+1}}{c_i} = \frac{1}{1+k\Delta t}
    \]
    \pause
    This equation does never give unphysical results!\\
    The implicit Euler method is \emph{unconditionally stable} \\
    (but maybe not very accurate or efficient).
\end{frame}


\begin{frame}
  \frametitle{Semi-implicit Euler method}
  \footnotesize\selectfont
  Usually $f$ is a non-linear function of $y$, so that linearization is required (recall Newton's method).
  
  \begin{align*}
    \frac{dy}{dx} &= f(y) \Rightarrow y_{i+1} = y_i + h f (y_{i+1}) \quad \text{using} \quad f(y_{i+1}) = f(y_i) + \left.\frac{df}{dy}\right|_i(y_{i+1}-y_i) + \ldots \\
  &\Rightarrow y_{i+1} =  y_i + h \left[ f(y_i) + \left.\frac{df}{dy}\right|_i (y_{i+1}-y_i) \right] \\ 
  &\Rightarrow \left(1-h\left.\frac{df}{dy}\right|_i \right)y_{i+1} = \left(1-h\left.\frac{df}{dy}\right|_i\right)y_i + h f(y_i) 
  \end{align*}
  
  \tikz{\node[emphblock,text width=\textwidth] {$\displaystyle \Rightarrow y_{i+1} = y_i + h \left( 1 - h \left.\frac{df}{dy}\right|_i \right)^{-1} f(y_i) $};}
  \pause
  For the case that $f(x,y(x))$ we could add the variable $x$ as an additional variable $y_{n+1}=x$. Or add one fully implicit Euler step (which avoids the computation of $\frac{\partial f}{\partial x}$): \vspace*{-1em}
  \[
    y_{i+1} = y_i + h f(x_{i+1},y_{i+1}) \Rightarrow y_{i+1} = y_i + h \left(1-h\left.\frac{df}{dy}\right|_i \right)^{-1} f(x_{i+1},y_i)
  \]
\end{frame}

\begin{frame}
  \frametitle{Semi-implicit Euler method - example}
  Second order reaction in a batch reactor:\\
  $\frac{dc}{dt} = -kc^2$ with $c_0 = 1$~\si{\mole\per\cubic\meter}, $k = 1$ \si{\cubic\meter\per\mole\per\second}, $t_\text{end} = 2$ \si{\second} \\
  Analytical solution: $c(t) = \frac{c_0}{1+kc_0t}$
  \vskip1em \pause  
  Define $f = -kc^2$, then $\frac{df}{dc} = -2kc \Rightarrow c_{i+1} = c_i - \frac{hkc_i^2}{1+2hkc_i}$.
  \pause
    \begin{longtable}{cccc}
    \hline
    $N$ & $\zeta$ & $\frac{\zeta^{}_\text{numerical}-\zeta_\text{analytical}}{\zeta_\text{analytical}}$ & $ r = \frac{\log\left(\frac{\epsilon_i}{\epsilon_{i-1}}\right)}{\log \left( \frac{N_{i-1}}{N_i}\right)} $ \\ \hline
    20  & 0.654066262 & \num{1.89E-002} & ---\\
    40  & 0.660462687 & \num{9.31E-003} & 1.02220\\
    80  & 0.663589561 & \num{4.62E-003} & 1.01162\\
    160 & 0.665134433 & \num{2.30E-003} & 1.00594\\
    320 & 0.665902142 & \num{1.15E-003} & 1.00300\\
    \hline
  \end{longtable}
\end{frame}

\subsection{Implicit midpoint method}
\begin{frame}
  \frametitle{Second order implicit method: Implicit midpoint method}
  \footnotesize\selectfont
\begin{longtable}{c c}
    \hline
    \begin{minipage}{0.4\textwidth}\centering Implicit midpoint rule \\(second order)\end{minipage} & \begin{minipage}{0.4\textwidth}\centering Explicit midpoint rule  (modified~Euler~method)\end{minipage} \\ \hline
    $y_{i+1} = y_i + hf\left(x_i + \frac{1}{2}h, {\color{tuered}\frac{1}{2}(y_i + y_{i+1})}\right)$  & $y_{i+1} = y_i + hf(x_i + \frac{1}{2}h, {\color{tuered}y_i + \frac{1}{2} h k_1})$ \\
    \hline
  \end{longtable}
  in case $f(y)$ then:
  \[
    f\left(\frac{1}{2}(y_i+y_{i+1})\right) = f_i + \left.\frac{df}{dy}\right|_i \left( \frac{1}{2}(y_i + y_{i+1})-y_i\right) = f_i + \frac{1}{2}\left.\frac{df}{dy}\right|_i(y_{i+1}-y_i)
  \]
  \pause
  Implicit midpoint rule reduces to: 
  \begin{align*}
    y_{i+1} &= y_i + h f_i + \frac{h}{2}\left.\frac{df}{dy}\right|_i(y_{i+1}-y_i)\\
    &\Rightarrow \left(1 - \frac{h}{2} \left.\frac{df}{dy}\right|_i\right)y_{i+1} = \left(1 - \frac{h}{2} \left.\frac{df}{dy}\right|_i\right)y_i + h f_i
  \end{align*}
  \tikz{\node[emphblock,text width=0.7\textwidth] {
    $ \displaystyle \Rightarrow y_{i+1} = y_i + h \left( 1 - \frac{h}{2} \left.\frac{df}{dy}\right|_i\right)^{-1} f_i$
    };}
\end{frame}

\begin{frame}
  \frametitle{Implicit midpoint method --- example}
  Second order reaction in a batch reactor: \\
  $\frac{dc}{dt} = -kc^2$ with $c_0 = 1$~\si{\mole\per\cubic\meter}, $k = 1$ \si{\cubic\meter\per\mole\per\second}, $t_\text{end} = 2$ \si{\second} (Analytical solution: $c(t) = \frac{c_0}{1+kc_0t}$). 
  \vskip1em \pause  
  Define $f = -kc^2$, then $\frac{df}{dc} = -2kc$.\vskip1em \pause
  Substitution: 
  \begin{align*}
    c_{i+1} &= c_i + h \left(1-\frac{h}{2} \cdot (-2kc_i)\right)^{-1} \cdot(-kc^2_i) \\
	    &= c_i - \frac{hkc_i^2}{1+hkc_i} = \frac{c_i + h k c_i^2 - h k c_i^2 }{1+hkc_i} \Rightarrow c_{i+1} = \frac{c_i}{1+hkc_i}
  \end{align*}
  
  \pause
  You will find that this method is exact for all step sizes $h$ because of the quadratic source term!
\end{frame}

\begin{frame}
  \frametitle{Implicit midpoint method --- example}
  {\color{tuealert}Second order} reaction in a batch reactor:\\
  $\frac{dc}{dt} = -kc^2$ with $c_0 = 1$~\si{\mole\per\cubic\meter}, $k = 1$ \si{\cubic\meter\per\mole\per\second}, $t_\text{end} = 2$ \si{\second}\\
  Analytical solution: $c(t) = \frac{c_0}{1+kc_0t}$
  \[
    c_{i+1} = \frac{c_i}{1+hkc_i}
  \]
  \pause  
  \begin{longtable}{cccc}
    \hline
    $N$ & $\zeta$ & $\frac{\zeta^{}_\text{numerical}-\zeta_\text{analytical}}{\zeta_\text{analytical}}$ & $ r = \frac{\log\left(\frac{\epsilon_i}{\epsilon_{i-1}}\right)}{\log \left( \frac{N_{i-1}}{N_i}\right)} $ \\ \hline
    20  & 0.6666666667 & \num{1.665E-016} & ---\\
    40  & 0.6666666667 & \num{0} & ---\\
    80  & 0.6666666667 & \num{0} & ---\\
    160 & 0.6666666667 & \num{0} & ---\\
    320 & 0.6666666667 & \num{0} & ---\\
    \hline
  \end{longtable}
\end{frame}

\begin{frame}
  \frametitle{Implicit midpoint method --- example}
  {\color{tuealert}Third order} reaction in a batch reactor: $\frac{dc}{dt} = -kc^3$\\
  Analytical solution: $c(t) = \frac{c_0}{\sqrt{1+2kc_0^2t}}$
  \[
    c_{i+1} = c_i - \frac{hkc_i^3}{1+\frac{3}{2}hkc_i^2}
  \]
  \pause  
  \begin{longtable}{cccc}
    \hline
    $N$ & $\zeta$ & $\frac{\zeta^{}_\text{numerical}-\zeta_\text{analytical}}{\zeta_\text{analytical}}$ & $ r = \frac{\log\left(\frac{\epsilon_i}{\epsilon_{i-1}}\right)}{\log \left( \frac{N_{i-1}}{N_i}\right)} $ \\ \hline
    20  & 0.5526916174 & \num{1.71E-004} & ---\\
    40  & 0.5527633731 & \num{4.17E-005} & 2.041\\
    80  & 0.5527807304 & \num{1.03E-005} & 2.021\\
    160 & 0.5527849965 & \num{2.55E-006} & 2.011\\
    320 & 0.5527860538 & \num{6.34E-007} & 2.005\\
    \hline
  \end{longtable}
\end{frame}

\section{Systems of ODEs}
\againframe<2>{contents_ode2}
\subsection{Solution methods for systems of ODEs}
\begin{frame}
  \frametitle{Systems of ODEs}
  A system of ODEs is specified using vector notation:
  \[
    \frac{d\vec{y}}{dx} = \vec{f}(x,\vec{y}(x))
  \]
  for
  \begin{multline*}
    \frac{dy_1}{dx} = f_1(x,y_1(x),y_2(x)) \quad \text{or} \quad f_1(x,y_1,y_2)\\
    \frac{dy_2}{dx} = f_2(x,y_1(x),y_2(x)) \quad \text{or} \quad f_2(x,y_1,y_2)\\
  \end{multline*}
  \pause
  \tikz{\node[emphblock,text width=\textwidth] {The solution techniques discussed before can also be used to solve systems of equations.};}
\end{frame}

\begin{frame}
  \frametitle{Systems of ODEs: Explicit methods}
  \begin{block}{Forward Euler method}
    $ \displaystyle  \vec{y}_{i+1} = \vec{y}_i + h \vec{f}(x_i,\vec{y}_i) $
  \end{block}
  \begin{block}{Improved Euler method (classical RK2)}
    $\displaystyle \vec{y}_{i+1} = \vec{y}_i + \frac{h}{2}(\vec{k}_1+\vec{k}_2)$
    \quad using \quad \begin{minipage}{0.4\textwidth}
      $\displaystyle \vec{k}_1 = \vec{f}(x_i,\vec{y}_i)$\\
      $\displaystyle \vec{k}_2 = \vec{f}(x_i+h,\vec{y}_i+h\vec{k}_1)$
    \end{minipage}
  \end{block}  
  \begin{block}{Modified Euler method (midpoint rule)}
    $\displaystyle \vec{y}_{i+1} = \vec{y}_i + h\vec{k}_2$
    \quad using \quad \begin{minipage}{0.4\textwidth}
      $\displaystyle \vec{k}_1 = \vec{f}(x_i,\vec{y}_i)$\\
      $\displaystyle \vec{k}_2 = \vec{f}(x_i+\frac{h}{2},\vec{y}_i+\frac{h}{2}\vec{k}_1)$
    \end{minipage}
  \end{block}   
\end{frame}

\begin{frame}
  \frametitle{Systems of ODEs: Explicit methods}
  \begin{block}{Classical fourth order Runge-Kutta method (RK4)}
    $\displaystyle \vec{y}_{i+1} = \vec{y}_i + h\left(\frac{\vec{k}_1}{6}+\frac{1}{3}\left( \vec{k}_2 + \vec{k}_3\right) + \frac{\vec{k}_4}{6} \right)$
    \\ \vskip2em
\quad using \quad \begin{minipage}{0.4\textwidth}
      $\displaystyle \vec{k}_1 = \vec{f}(x_i,\vec{y}_i)$\\ \vskip1ex
      $\displaystyle \vec{k}_2 = \vec{f}(x_i+\frac{h}{2},\vec{y}_i+\frac{h}{2}\vec{k}_1)$ \\ \vskip1ex
      $\displaystyle \vec{k}_3 = \vec{f}(x_i+\frac{h}{2},\vec{y}_i+\frac{h}{2}\vec{k}_2)$ \\ \vskip1ex
      $\displaystyle \vec{k}_4 = \vec{f}(x_i+h,\vec{y}_i+h\vec{k}_3)$ \\ \vskip1ex
    \end{minipage}
%     \end{center}
  \end{block}  
\end{frame}

\subsection{Solving systems of ODEs in Python}
\begin{frame}
  \frametitle{Solving systems of ODEs in Python}
  Solving systems of ODEs in Python follows a similar process to Matlab but leverages the SciPy library:
  \vspace*{1em}
  \begin{enumerate}
    \item Create a function that specifies the ODEs. This function returns the $\frac{d\vec{y}}{dx}$ vector.
    \item Initialize solver variables and settings (e.g. step size, initial conditions, tolerance) within your Python script. Initial conditions and tolerances should be given per-equation, i.e. as a list or array.
    \item Call the ODE solver function from the SciPy library, passing the ODE function described in point 1 as an argument.
    \begin{itemize}
      \item The ODE solver will return an object that contains the solution arrays for each dependent variable.
    \end{itemize}
  \end{enumerate}
\end{frame}

\begin{frame}[fragile]
  \frametitle{Solving systems of ODEs in Python: example}
  We solve the system: $\displaystyle \frac{dx_1}{dt} = -x_1 - x_2, \quad  \frac{dx_2}{dt} = x_1 - 2x_2 $
  \begin{block}{Create an ODE function}
    \begin{lstlisting}[language=Python]
from scipy.integrate import solve_ivp

def my_ode_function(t, x):
    dxdt = [-x[0] - x[1], x[0] - 2*x[1]]
    return dxdt
    \end{lstlisting}
  \end{block}
  \pause
  \begin{block}{Create a solution script}
    \begin{lstlisting}[language=Python]
x_init = [0, 1]        # Initial conditions
t_span = [0, 10]       # Time span
sol = solve_ivp(my_ode_function, t_span, x_init, rtol=1e-4, atol=[1e-4, 1e-4])

# The solution can be accessed as sol.t (time points) and sol.y (solutions at each time point)
    \end{lstlisting}
  \end{block}
\vfill
\end{frame}


{\nologo
\begin{frame}[fragile]
  \frametitle{Solving systems of ODEs in Python: example}
  Plot the solution using `matplotlib`:
  \begin{lstlisting}[language=Python]
plt.plot(sol.t, sol.y[0], 'r-x', label='x1')
plt.plot(sol.t, sol.y[1], 'b-o', label='x2')
plt.xlabel('t'); plt.ylabel('x'); plt.legend(); plt.grid(); plt.show()
  \end{lstlisting}
  \pause
  \begin{center}
    \begin{tikzpicture}
      \begin{axis}[%every axis/.append style={font=\footnotesize},
        width=\textwidth, height=6.5cm,     % size of the image
        grid = major,grid style={dashed, gray!30},
        axis background/.style={fill=white},
        axis x line=middle,axis y line=middle,ylabel=$x$,xlabel=$t$]

        \addplot[graph,mark=x] table []{data/ODE_matlabsolve1.dat};  \addlegendentry{$x_1$}
        \addplot[graph,mark=o,draw=tueblue,mark options={fill=white}] table []{data/ODE_matlabsolve2.dat};  \addlegendentry{$x_2$}
      \end{axis}
    \end{tikzpicture}
  \end{center}
\end{frame}
}

\begin{frame}[fragile]
  \frametitle{Solving systems of ODEs in Python: repeated notes}
  A few notes on working with `solve\_ivp` and other SciPy solvers. If we want to give additional arguments (e.g. `a`, `b`, and `c`) to our ODE function, we can list them in the function signature:
  \begin{lstlisting}[language=Python]
def my_ode(t, x, a, b, c):
  \end{lstlisting}
  The additional arguments can now be set in the solution script by passing them using the `args` parameter:
  \begin{lstlisting}[language=Python]
sol = solve_ivp(my_ode, t_span, x_init, args=(a, b, c))
  \end{lstlisting}
  \pause
  \begin{itemize}
    \item Of course, in the solution script, the variables do not need to be called `a`, `b`, and `c`. They could be named anything:
      \begin{lstlisting}[language=Python]
sol = solve_ivp(my_ode, t_span, x_init, args=(k1, phi, V))
      \end{lstlisting}
    \item These variables can be of any type (lists, arrays, dictionaries). Especially a dictionary can be useful to carry many values in one variable.
    \item Python inherently supports functions with more arguments than the default `(t, x)` required by `solve\_ivp`, so there is no need for an alternative approach as in MATLAB.
  \end{itemize}
\end{frame}


\begin{frame}[fragile]
  \frametitle{Solving systems of ODEs in Python: example}
  You may have noticed that the step size in \( t \) varied. This is because we have given the begin and end times of our time span, and \texttt{solve\_ivp} uses adaptive step size for efficiency:
  \begin{lstlisting}[language=Python]
t_span = [0, 10]
  \end{lstlisting}
  \pause
  You can also retrieve the solution at specific steps, by supplying all steps explicitly as an array, e.g.:
  \begin{lstlisting}[language=Python]
t_eval = np.linspace(0, 10, 101)
  \end{lstlisting}
  This example provides 101 explicit time steps between 0 and 10 seconds.
  \vskip1em
  Note that this is an interpolated result. The solver uses, in the background, still the adaptive step size functionality!
  \vfill
\end{frame}

\subsection{Stiff systems of ODEs}
\begin{frame}
  \frametitle{Systems of ODEs: Implicit methods}
  \begin{block}{Backward Euler method}
    $ \displaystyle  \vec{y}_{i+1} = \vec{y}_i + h \left(\vec{I} - h\left. \frac{d\vec{f}}{d\vec{y}}\right|_i \right)^{-1}\vec{f}(\vec{y}_i)$
  \end{block}
  \begin{block}{Implicit midpoint method}
    $ \displaystyle  \vec{y}_{i+1} = \vec{y}_i + h \left(\vec{I} - \frac{h}{2}\left. \frac{d\vec{f}}{d\vec{y}}\right|_i \right)^{-1} \vec{f}(\vec{y}_i)$
  \end{block}  
\end{frame}


\begin{frame}
  \frametitle{Stiff systems of ODEs}
  A system of ODEs can be stiff and require a different solution method. \pause
  For example:
  \[
    \frac{dc_1}{dt} = 998c_1 + 1998c_2 \qquad 
    \frac{dc_2}{dt} = -999c_1 -1999c_2
  \]
  with boundary conditions $c_1(t=0)=1$ and $c_2(t=0)=0$. \\
  The analytical solution is: 
  \[
    c_1 = 2e^{-t}-e^{-1000t} \qquad
    c_2 =-e^{-t}+e^{-1000t}
  \]
  For the explicit method we require $\Delta t<10^{-3}$ despite the fact that the term is completely negligible, but essential to keep stability. \pause
  \tikz{\node[emphblock,text width=\textwidth] {The ``disease'' of stiff equations: we need to follow the solution on the shortest length scale to maintain stability of the integration, although accuracy requirements would allow a much larger time step.};}
\end{frame}

\begin{frame}
  \frametitle{Demonstration with example}
  Forward Euler (explicit) 
%   \vskip2em
  \begin{align*}
 \frac{c_{1,i+1} - c_{1,i}}{dt} &= 998c_{1,i}+1998c_{2,i}\\
  \frac{c_{2,i+1} - c_{2,i}}{dt} &= -999c_{1,i}-1999c_{2,i}
  \end{align*}
    \qquad $\Rightarrow $ 
    \begin{minipage}{0.7\textwidth}
      $c_{1,i+1} = \left(1+998\Delta t\right)c_{1,i} + 1998\Delta t c_{2,i}$\\
      $c_{2,i+1} = -999\Delta t c_{1,i} + \left( 1- 1999\Delta t\right) c_{2,i}$
  \end{minipage}\vskip1em
  \phantom{
  $A \vec{c}_{i+1} = \vec{c}_i$ with 
  $A = \begin{pmatrix}
    1-998\Delta t & -1998\Delta t \\
    999\Delta t & 1+1999\Delta t
  \end{pmatrix}$ 
  and $\vec{b} = \begin{pmatrix}
             c_{1,i}\\
             c_{2,i}\\
           \end{pmatrix}$}
\end{frame}

\begin{frame}
  \frametitle{Demonstration with example}
 \rowcolors[]{20}{white}{white}
  Backward Euler (implicit)
%   \vskip1em
  \begin{align*}
 \frac{c_{1,i+1} - c_{1,i}}{\Delta t} &= 998c_{1,i+1}+1998c_{2,i+1}\\
  \frac{c_{2,i+1} - c_{2,i}}{\Delta t} &= -999c_{1,i+1}-1999c_{2,i+1}
  \end{align*}
    \qquad $\Rightarrow $ 
    \begin{minipage}{0.7\textwidth}
      $\left(1-998\Delta t\right)c_{1,i+1} - 1998\Delta t c_{2,i} = c_{1,i}$\\
      $999\Delta t c_{1,i+1}+\left(1+999\Delta t\right)c_{2,i+1} = c_{2,i}$
  \end{minipage} \vskip1em \pause
  $A \vec{c}_{i+1} = \vec{c}_i$ with 
  $A = \begin{pmatrix}
    1-998\Delta t & -1998\Delta t \\
    999\Delta t & 1+1999\Delta t
  \end{pmatrix}$ 
  and $\vec{b} = \begin{pmatrix}
             c_{1,i}\\
             c_{2,i}\\
           \end{pmatrix}$
\end{frame}

\begin{frame}
  \frametitle{Demonstration with example}
 \rowcolors[]{20}{white}{white}
  Backward Euler (implicit)
%   \vskip1em
  $A \vec{c}_{i+1} = \vec{c}_i$ with 
  $A = \begin{pmatrix}
    1-998\Delta t & -1998\Delta t \\
    999\Delta t & 1+1999\Delta t
  \end{pmatrix}$ 
  \quad and \quad $\vec{b} = \begin{pmatrix}
             c_{1,i}\\
             c_{2,i}\\
           \end{pmatrix}$
\vskip1em \pause
Cramers rule:\\
$c_{1,i+1} = \dfrac{\begin{vmatrix}
                                   c_{1,i} & -1998\Delta t \\
                                   c_{2,i} & 1+1999\Delta t
                                 \end{vmatrix}
}{\det{A}} = \frac{\left(1+1999\Delta t\right)c_{1,i}+1998 \Delta t c_{2,i}}{\left( 1-998\Delta t\right)\left( 1+1999\Delta t\right) + 1998\cdot 999 \Delta t^2}$\\

$c_{2,i+1} = \dfrac{\begin{vmatrix}
                                   1-998\Delta t & c_{1,i}\\
                                   999\Delta t & c_{2,i}
                                 \end{vmatrix}
}{\det{A}} = \frac{-999\Delta t c_{1,i} + \left(1-998\Delta t\right)c_{2,i}}{\left(1-998\Delta t\right)\left( 1+1999\Delta t\right) + 1998\cdot 999 \Delta t^2}$\\ \vskip1em 
% \tikz{\node[emphblock,text width=\textwidth] {
Forward Euler: $\Delta t \leq 0.001$ for stability\\
Backward Euler: always stable, even for $\Delta t > 100$ (but then not very accurate!)
% };}
\end{frame}

\begin{frame}
  \frametitle{Demonstration with example}
  \tikz{\node[emphblock,text width=\textwidth] {
  Cure for stiff problems: use implicit methods!
  To find out whether your system is stiff: check whether one of the eigenvalues have an imaginary part};}
\end{frame}
\begin{frame}[fragile]
  \frametitle{Implicit methods in Python}
  Python offers a solver, \lstinline$solve\_ivp$, for stiff and non-stiff problems.
  $$
    \frac{dc_1}{dt} = 998c_1 + 1998c_2 \quad 
    \frac{dc_2}{dt} = -999c_1 -1999c_2,\, c_1(0) = 1,\, c_2(0) = 0
  $$
  \begin{itemize}
    \item Create the ode function
    \begin{lstlisting}[language=Python,basicstyle=\tiny]
def stiff_ode(t, c):
    dcdt = [0, 0]  # Pre-allocation
    dcdt[0] = 998 * c[0] + 1998 * c[1]
    dcdt[1] = -999 * c[0] - 1999 * c[1]
    return dcdt
    \end{lstlisting}
  \item Compare the resolution of the solutions
    \begin{lstlisting}[language=Python,basicstyle=\tiny]
from scipy.integrate import solve_ivp
import matplotlib.pyplot as plt

# Using RK45 (similar to ode45 in MATLAB)
sol = solve_ivp(stiff_ode, [0, 1], [1, 0], method='RK45')
plt.subplot(2, 1, 1)
plt.plot(sol.t, sol.y.T, "-x")

# Using BDF (similar to ode15s in MATLAB)
sol = solve_ivp(stiff_ode, [0, 1], [1, 0], method='BDF')
plt.subplot(2, 1, 2)
plt.plot(sol.t, sol.y.T, "-x")

plt.show()
    \end{lstlisting}
  \end{itemize}
\end{frame}


{\nologo
\begin{frame}[fragile]
\frametitle{Implicit methods in Python}
\begin{center}
  \begin{tikzpicture}
    \begin{axis}[
      width=\textwidth, height=4cm,
      grid = major,grid style={dashed, gray!30},
      axis background/.style={fill=white},
      axis x line=middle,axis y line=middle,ylabel=$x$,xlabel=$t$,title=ode45]

      \addplot[graph,mark=x] table []{data/ODE_matlabsolve_implicit11.dat};  \addlegendentry{$x_1$}
      \addplot[graph,mark=o,draw=tueblue,mark options={fill=white}] table []{data/ODE_matlabsolve_implicit12.dat};  \addlegendentry{$x_2$}
    \end{axis}
  \end{tikzpicture}
  \begin{tikzpicture}
    \begin{axis}[
      width=\textwidth, height=4cm,
      grid = major,grid style={dashed, gray!30},
      axis background/.style={fill=white},
      axis x line=middle,axis y line=middle,ylabel=$x$,xlabel=$t$,title=solve\_ivp]

      \addplot[graph,mark=x] table []{data/ODE_matlabsolve_implicit21.dat};  \addlegendentry{$x_1$}
      \addplot[graph,mark=o,draw=tueblue,mark options={fill=white}] table []{data/ODE_matlabsolve_implicit22.dat};  \addlegendentry{$x_2$}
    \end{axis}
  \end{tikzpicture}
\end{center}
The explicit solver requires 1245 data points (default settings), the implicit solver just 48!
\end{frame}
}

\begin{frame}<handout:0>[fragile]
\frametitle{Implicit methods in Python: Generic backward Euler}
\begin{lstlisting}[language=Python,basicstyle=\tiny]
import numpy as np
from scipy.linalg import inv
import matplotlib.pyplot as plt

# Input
N = 40
t_end = 1
y0 = np.array([1, 0])
dh = 1e-12
def func(t, y):
    # Define your system of ODEs here
    pass  # Remove this line when you add your ODEs

# Preallocate and calculate
time = np.linspace(0, t_end, N+1)
h = time[1] - time[0]
y = np.zeros((len(time), 2))
y[0, :] = y0

for i in range(N):
    # Get df_dy1 and df_dy2 (both column vectors)
    jac_dfdy1 = (func(0, y[i, :] + [dh, 0]) - func(0, y[i, :])) / dh
    jac_dfdy2 = (func(0, y[i, :] + [0, dh]) - func(0, y[i, :])) / dh
    jacobian = np.column_stack((jac_dfdy1, jac_dfdy2))
    
    # Update formula
    y[i+1, :]  = y[i, :] + h * inv(np.eye(len(y0)) - h * jacobian).dot(func(0, y[i, :]))

plt.plot(time, y)
plt.show()
\end{lstlisting}
\end{frame}


\begin{frame}<handout:1|beamer:0>[fragile]
  \frametitle{Implicit methods in Python: Generic backward Euler}
  How to make a generic Backward Euler implementation? Recall the update formula:
      $ \displaystyle  \vec{y}_{i+1} = \vec{y}_i + h \left(\vec{I} - h\left. \frac{d\vec{f}}{d\vec{y}}\right|_i \right)^{-1}\vec{f}(\vec{y}_i)$
  \vskip1em
  \begin{itemize}
      \item Set up input: Number of steps, end time, initial conditions
      \item Preallocate and calculate: create a full time vector using \lstinline|numpy.linspace|, calculate the step size \( h \), preallocate \( y \) with zeros using \lstinline|numpy.zeros| and store the initial condition as the first \( y \).
      \item Loop over the number of iterations:
      \begin{itemize}
          \item Compute the Jacobian: calculate the function both for \( y_i \) as well as for \( y_i + s \), where \( s \) is a very small number. Recall:
          \[
          \frac{df}{dy} = \frac{f(y+s) - f(y)}{s}
          \]
          \item Compute the update formula for \( y_{i+1} \). Use \lstinline|numpy.eye|, \lstinline|scipy.linalg.inv|.
      \end{itemize}
  \end{itemize}
  \end{frame}
  

\section{Boundary value problems}
\againframe<2>{contents_ode2}
\subsection{Shooting method}
%\againframe{ivpbvp}
\begin{frame}
  \frametitle{Shooting method}
  How to solve a BVP using the shooting method:\\ \vskip1em
  \centering
    \begin{tikzpicture}[scale=5]
      \node[] (y1s) at (0,0.1) {$y_{1,s}$};
\node[] (y1f) at (1,0.1) {\color{scharlaken}$y_{1,f}$};
\node[] (y2s) at (0,0) {\color{scharlaken}$y_{2,s}$};
\node[] (y2f) at (1,0) {$y_{2,f}$};
\node[fdot] (xs) at (0,-0.1) {};
\node[fdot] (xf) at (1,-0.1) {};
\node[anchor=north] at (xs.south) {$x_s$};
\node[anchor=north] at (xf.south) {$x_f$};
\draw[line] (xs) -- (xf);
\draw[line,->,densely dashed] (0.1,0.05) -- node[midway,above] {``Shooting''} (0.9,0.05);

\coordinate[] (b) at ($(y1f)!0.5!(y2f) + (0.1,0) $) {};
\coordinate[right of=b] (c1) {};
\coordinate[below=1.4cm] (c2) at (c1)  {};
%       \coordinate[below of=1cm] at (c2) (c5) {};

\coordinate[] (s) at ($(y1s)!0.5!(y2s) - (0.1,0) $) {};
\coordinate[left of=s] (c3) {};
\coordinate[below=1.4cm] (c4) at (c3)  {};
%       \coordinate[below of=c4] (c6) {};

%       \node[] (b) at ($(y1f)!0.5!(y2f)$) {};
\draw[line,->,densely dashed,draw=tuegreen,rounded corners=10pt] (b) -- (c1) -- (c2) -- (c4) -- (c3) -- (s);
%       \draw[line,densely dashed,draw=tuegreen] (y2f) to [controls=(1,-0.2) and +(-0.1,-0.2)] (y2s);
    \end{tikzpicture}
  \begin{itemize}
    \item Define the system of ODEs
    \item Provide an initial guess for the unknown boundary condition
    \item Solve the system and compare the resulting boundary condition to the expected value
    \item Adjust the guessed boundary value, and solve again. Repeat until convergence.
    \begin{itemize}
      \item Of course, you can subtract the expected value from the computed value at the boundary, and use a non-linear root finding method
    \end{itemize}

  \end{itemize}
\end{frame}

\begin{frame}[t]
  \frametitle{BVP: example in Excel}
  \footnotesize\selectfont
  \setlength{\mathindent}{12pt}
  Consider a chemical reaction in a liquid film layer of thickness $\delta$:\\
  $\displaystyle \mathcal{D}\frac{d^2c}{dx^2} = k_Rc $ with \begin{minipage}{0.7\textwidth}
    \begin{align*}
      c(x=0) &= C_{A,i,L} = 1 &\quad \text{(interface concentration)} \\
      c(x=\delta) &= 0 &\quad \text{(bulk concentration)}\\
    \end{align*}
  \end{minipage} \\
  Question: compute the concentration profile in the film layer.
  \pause
  \only<2|handout:1>{
    \begin{block}{Step 1: Define the system of ODEs}
      This second-order ODE can be rewritten as a system of first-order ODEs, if we define the flux $q$ as:
      \[
  q = -\mathcal{D}\frac{dc}{dx}
      \]
      Now, we find:
      \begin{align*}
  \frac{dc}{dx} &= -\frac{1}{\mathcal{D}}q\\
  \frac{dq}{dx} &= -k_Rc
      \end{align*}
    \end{block}
  }
  \only<3|handout:2>{
    \vskip1em
  \begin{columns}
    \column{0.5\textwidth}
    \begin{block}{Step 2: Set the boundary conditions}
      The boundary conditions for the concentrations at $x=0$ and $x=\delta$ are known.\\ \vskip1em The flux at the interface, however, is not known, and should be solved for.
    \end{block}
    \column{0.5\textwidth}
    \centering\tikz\node [emphblock,cloud, draw,cloud puffs=11,minimum width=\columnwidth] 
    {
      \begin{minipage}{2cm}
        $\displaystyle \frac{dc}{dx} = -\frac{1}{\mathcal{D}}q$ \\ \vskip1ex
        $\displaystyle \frac{dq}{dx} = -k_Rc$
      \end{minipage}
    };
  \end{columns}
  }
\end{frame}

{\nologo
\begin{frame}
  \frametitle{BVP: example in Excel}  
  \scriptsize\selectfont
  \rowcolors[]{80}{white}{white}\renewcommand\arraystretch{1.1}
  
  Solving the two first-order ODEs in Excel. First, the cells with constants:
  \begin{columns}
    \column{0.5\textwidth}
    \begin{longtable}{|>{\columncolor{gray!40}}R{0.3cm}*{1}{|L{0.8cm}}*{1}{|L{1.2cm}}*{1}{|L{1.3cm}}|}
      \hline
      \rowcolor{gray!40}& \centering A  & \centering B& \centering C \tabularnewline
      \hline
      1 & CAiL& 1 & mol/m3  \\
      \hline
      2 & D \hfill & 1e--8 & m2/s \\
      \hline
      3 & kR \hfill& 10& 1/s \\
      \hline
      4 & delta\hfill & 1e--4 & m \\
      \hline
      5 & N \hfill& 100& \\
      \hline
      6 & dx \hfill& =B4/B5 &  \\
      \hline
    \end{longtable}
    \column{0.5\textwidth}
    \centering\tikz\node [emphblock,cloud, draw,cloud puffs=11,minimum width=\columnwidth] 
    {
      \begin{minipage}{2cm}
        $\displaystyle \frac{dc}{dx} = -\frac{1}{\mathcal{D}}q$ \\ \vskip1ex
        $\displaystyle \frac{dq}{dx} = -k_Rc$
      \end{minipage}
    };
  \end{columns}
  \pause \vskip1em
  Now, we program the forward Euler (explicit) schemes for $c$ and $q$ below: \\ \setlength{\LTleft}{-0.5cm}
  \begin{longtable}{|>{\columncolor{gray!40}}R{0.3cm}*{1}{|L{1.8cm}}*{1}{|L{4.05cm}}*{1}{|L{3.7cm}}|}
    \hline
    \rowcolor{gray!40}& \centering A  & \centering B& \centering C \tabularnewline
    \hline
    10 & x \hfill & c\hfill & q  \hfill\\
    \hline
    11 & 0     & =B1 & \color{tuered} 10 \\
    \hline
    12 & =A11+\$B\$6  & =B11+\$B\$6*(--1/\$B\$2*C11) & =C11+\$B\$6*(-\$B\$3*B11)\\
    \hline
    13 & =A12+\$B\$6  & =B12+\$B\$6*(--1/\$B\$2*C12) & =C12+\$B\$6*(-\$B\$3*B12)\\
    \hline
    $\dots$ & $\dots$ & $\dots$& $\dots$\\
    \hline
    111 & =A110+\$B\$6 & =B110+\$B\$6*(--1/\$B\$2*C110) & =C110+\$B\$6*(-\$B\$3*B110) \\
    \hline
  \end{longtable}
\end{frame}
}

\begin{frame}
  \frametitle{BVP: example in Excel}  
  \begin{itemize}
    \item We now have profiles for $c$ and $q$ as a function of position $x$.
    \item The concentration $c(x=\delta)$ depends (eventually) on the boundary condition at the interface $q(x=0)$
    \item We can use the solver to change $q(x=0)$ such that the concentration at the bulk meets our requirement: $c(x=\delta)=0$
  \end{itemize}

\end{frame}

\begin{frame}[fragile]
  \frametitle{BVP: example in Python}
  We first program the system of ODEs in a separate function:
  \begin{align*}
    \frac{dc}{dx} &= -\frac{1}{\mathcal{D}}q\\
    \frac{dq}{dx} &= -k_Rc
  \end{align*}
  \begin{lstlisting}[language=Python]
def diffReactSystem(x, y, ps):
    c, q = y
    dcdx = -q / ps['D']
    dqdx = -ps['kR'] * c
    return [dcdx, dqdx]
  \end{lstlisting}
  \pause\vskip1em
  Note that we pass a dictionary that contains the required parameters: \lstinline|ps|.
\end{frame}

\begin{frame}[fragile, label={diffReactODEtrial}]
  \frametitle{BVP: example in Python}
  Let's first try to solve the ODE system using \lstinline|scipy.integrate.solve_ivp|:
  \begin{lstlisting}[language=Python,basicstyle=\scriptsize]
from scipy.integrate import solve_ivp
import numpy as np

# Set up parameters
ps = {'D': 1e-8, 'kR': 10, 'delta': 1e-4, 'C_a_L': 1, 'q0': 1e-3}

# Solve ODE system
sol = solve_ivp(diffReactSystem, [0, ps['delta']], [ps['C_a_L'], ps['q0']], args=(ps,))

# Postprocessing
c = sol.y[0]
q = sol.y[1]
t = sol.t

import matplotlib.pyplot as plt
plt.plot(t, c)
plt.xlabel('Position in film layer [m]')
plt.ylabel('Concentration value')
plt.show()
  \end{lstlisting}
\end{frame}

\begin{frame}[fragile]
  \frametitle{BVP: example in Python}
  Now we want to fit the value for \(q\) at \(x=0\) (defined below as \lstinline|bcq|), such that the concentration at \(x=\delta\) equals zero. We create a function with the output defined as the deviation from the target value:
  \begin{lstlisting}[language=Python]
def diffReactCrit(bcq, ps):
    bcq = bcq[0] # Cannot be array if it is to be fed to solve_ivp
    sol = solve_ivp(diffReactSystem, [0, ps['delta']], [ps['C_a_L'], bcq], args=(ps,))
    c = sol.y[0]
    f = c[-1]  # We subtract the desired value from the concentration at x=delta (0 in this case)
    return f
  \end{lstlisting}
  \pause
  Note the following:
  \begin{itemize}
    \item We use the interval \(0\leq x \leq \delta\)
    \item Boundary conditions are given as: \(c(x=0)=1\) and \(q(x=0)=\) \lstinline|bcq|, which is given as an argument to the function (i.e. changeable from 'outside'!)
    \item The function returns \lstinline|f|, the difference between the computed and desired concentration at \(x=\delta\).
  \end{itemize}
\end{frame}

\begin{frame}[fragile]
  \frametitle{BVP: example in Python}
  Finally, we should solve the system to obtain the correct boundary condition \(q=\) \lstinline|bcq| such that \(c(x=\delta)=0\). We can use the built-in function \lstinline|scipy.optimize.root| to do this:
  \begin{lstlisting}[language=Python]
from scipy.optimize import root

# Set up parameters
ps = {'D': 1e-8, 'kR': 10, 'delta': 1e-4, 'C_a_L': 1, 'q0': 2e-4}

# Fit boundary condition for q on x=0 such that c(end)=0
result = root(diffReactCrit, ps['q0'], args=(ps,))
fitted_q = result.x[0]

# Solve ODE once more to plot the final data
sol = solve_ivp(diffReactSystem, [0, ps['delta']], [ps['C_a_L'], fitted_q], args=(ps,))
  \end{lstlisting}
  Postprocessing of the data can be done similar to the example in slide~\ref{diffReactODEtrial}.
\end{frame}


\begin{frame}[fragile]
  \frametitle{BVP example: analytical solution}
  Compare with the analytical solution: \\ \vskip1em
    \tikz \node[emphblock,text width=\textwidth] {
  $\displaystyle q = k_L E_A C_{A,i,L} \quad$ with \\ 
  \begin{minipage}{0.6\textwidth}
      \begin{align*}
  E_A &= \frac{\text{Ha}}{\tanh \text{Ha}} &\quad \text{(Enhancement factor)} \\
  \text{Ha} &= \frac{\sqrt{k_R\mathcal{D}}}{k_L} &\quad \text{(Hatta number)} \\ 
  k_L &= \frac{\mathcal{D}}{\delta} &\quad \text{(mass transfer coefficient)}
      \end{align*}
    \end{minipage}
};
\end{frame}

\section{Conclusion}
\againframe<2>{contents_ode2}
\begin{frame}
  \frametitle{Other methods}
  Other explicit methods:
  \begin{itemize}
    \item Bulirsch-Stoer method (Richardson extrapolation + modified midpoint method)
  \end{itemize}
  \vskip1em
  Other implicit methods:
  \begin{itemize}
    \item Rosenbrock methods (higher order implicit Runge-Kutta methods)
    \item Predictor-corrector methods
  \end{itemize}
\end{frame}

\begin{frame}
  \frametitle{Summary}
  \begin{itemize}
    \item Several solution methods and their derivation were discussed:
    \begin{itemize}
      \item Explicit solution methods: Euler, Improved Euler, Midpoint method, RK45
      \item Implicit methods: Implicit Euler and Implicit midpoint method
      \item A few examples of their spreadsheet implementation were shown
    \end{itemize}
    \item We have paid attention to accuracy and instability, rate of convergence and step size
    \item Systems of ODEs can be solved by the same algorithms. Stiff problems should be treated with care.
    \item An example of solving ODEs with Python was demonstrated.
  \end{itemize}
\end{frame}