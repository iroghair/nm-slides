\part{Examples}
\section{Style testing}
\subsection*{General}
\begin{frame}[label=contents_sample]
  \frametitle{This is a slide deck just for testing}
  \mode<beamer>{
    \only<1>{\tableofcontents}
  }
  \only<2>{\tableofcontents[currentsection]}
\end{frame}


\begin{frame}[fragile]
  \frametitle{Practice}
  Given a vector 
  \[ 
     x = \left[2 \ 4 \ 6 \ 8 \ 10 \ 12 \ 14 \ 16 \ 18 \ 20 \ 30 \ 40 \ 50 \ 60 \ 70 \ 80 \right]
  \]
  \begin{itemize}
   \item Define the vector using \lstinline|range|'s, without typing all individual elements
   \pause
   \begin{onlyenv}<beamer> \begin{lstlisting}[]
 >>> x = list(range(2,20,2)) + list(range(20,90,10))
     \end{lstlisting}
     \pause
    \end{onlyenv}
   \item Investigate the meaning of the following commands:
   \begin{lstlisting}[language=Python, numbers=none]
 >>> x[2]            
 >>> x[0:5]          
 >>> x[:-1]          
 >>> y = x[4:]       
 >>> y[3]            
 >>> y.pop(3)      
 >>> sum(x)    
 >>> max(x)       
 >>> min(x) 
 >>> x[::-1]       
     \end{lstlisting}    
  \end{itemize}
 \end{frame}
