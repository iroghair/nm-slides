\part{Examples}
\section{Style testing}
\subsection*{General}
\begin{frame}[label=contents_sample]
  \frametitle{This is a slide deck just for testing}
  \mode<beamer>{
    \only<1>{\tableofcontents}
  }
  \only<2>{\tableofcontents[currentsection]}
\end{frame}

\begin{frame}
 \frametitle{The following is a list with appearing items?}
 \begin{itemize}[<+->]
  \item Scientific techniques depend in an increasing fashion upon computer programs and simulation methods
  \item Knowledge of programming allows you to automate routine tasks 
  \item Ability to understand algorithms by inspection of the code 
  \item Learn to think by dissecting a problem into smaller, easier to solve, parts 
 \end{itemize}
\end{frame}

\begin{frame}
 \frametitle{Here are some block contents}
 \begin{block}{What is a program?}
  \emph{A program is a sequence of instructions that is written to perform a certain task on a computer.} % SOURCE http://www.greenteapress.com/thinkpython/html/thinkpython002.html
  \end{block}
  \begin{itemize}
    \item The computation might be something mathematical, a symbolic operation, image analysis, etc.%such as solving a system of equations or finding the roots of a polynomial
    \item It can also be a symbolic computation, such as searching and replacing text in a document 
    \item A program may even be used to compile another program
    \item A program consists of one or more \emph{algorithms}
  \end{itemize}
  \begin{block}{Program layout}
    \begin{enumerate}
        \item Input (Get the radius of a circle)
        \item Operations (Compute and store the area of the circle)
        \item Output (Print the area to the screen)
    \end{enumerate}
  \end{block}
\end{frame}

\begin{frame}[fragile]
\frametitle{Versatility of Python: Image analysis}
\begin{columns}
  \column{0.5\textwidth}
\begin{lstlisting}
# Importing necessary libraries
import numpy as np
from scipy import ndimage
from PIL.Image import fromarray
from skimage import io, color, feature, measure

# Loading and processing image 
I = io.imread('bub0.png')
BW = color.rgb2gray(I)
E = feature.canny(BW) 
F = ndimage.binary_fill_holes(E)

# Show final image
fromarray(F).show()
\end{lstlisting}  
\column{0.5\textwidth}
  \vfill
    This is an empty column
\end{columns}
\end{frame}

\begin{frame}[fragile]
\frametitle{Versatility of Python: Image analysis}
\begin{lstlisting}
# Beginning of render.py
import sys, os
import subprocess as sp
from pathlib import Path 

# AUX endings
aux = (".toc", ".snm", ".out", ".nav", ".aux", ".log", ".vrb", ".dvi", ".fls", ".gz")
outfile = Path("./Slides.pdf")


def main(argv):
    # Collecting the filenames 
    filenames = [Path(f).stem for f in argv]
    f_sep_commas = ",".join([f for f in filenames])
    f_with_path = ",".join(["sections/%s"%f for f in filenames])

    # Writing includes to include.tex
    if filenames:
        includes = "\\includeonlylecture{%s}\n\\includeonly{%s}"
        includes = includes%(f_sep_commas, f_with_path)
    else:
        includes = ""

    with open("./preamble/include.tex","w") as f:
        f.write(includes)
\end{lstlisting}  
\end{frame}

\begin{frame}[fragile]
  \frametitle{Getting started}
   Start Python, and enter the following commands on the command line. Evaluate the output.
   \pause
   \begin{columns}
    \column{0.75\textwidth} 
      \begin{lstlisting}[numbers=none]
>> 2 + 3        # Some simple calculations
>> 2 * 3
>> 2 * 3**2     # Powers are done using ** (*@ \pause @*)
>> a = 2        # Storing values into the workspace
>> b = 3
>> c = (2 * 3)**2  # Parentheses set priority
>> 8 / a - b (*@ \pause @*)
>> import math  # Mathematical functions can be used 
>> math.sin(a)  
>> math.sin(0.5 * math.pi)  # math.pi is an internal Python variable
>> import cmath  # for working with complex numbers
>> cmath.sqrt(-1)  # ... as are imaginary numbers    
      \end{lstlisting}\pause
    \column{0.25\textwidth} 
      \begin{lstlisting}[style=PyOutput,numbers=none]
5
6
18



1.0

0.9092974268256817
1.0

1j
      \end{lstlisting}
  \end{columns}
 \end{frame}
 
\begin{frame}[fragile]
  \frametitle{Lists in Python (1)}
  \begin{itemize}
    \item Lists are containers of collections of objects
    \item A list is initialized using square brackets with comma-separated elements
    \begin{lstlisting}[language=Python,numbers=none]
>>> brands = ['Audi', 'Toyota', 'Honda', 'Ford', 'Tesla']
    \end{lstlisting}
    \item Lists can contain and mix any object type, even other lists:
    \begin{lstlisting}[language=Python,numbers=none]
>>> another_list = [0.0, 0.1, 0.2, 'Hello', brands]
>>> print(another_list)
    \end{lstlisting}
    \begin{lstlisting}[style=PyOutput]
[0.0, 0.1, 0.2, 'Hello', ['Audi', 'Toyota', 'Honda', 'Ford', 'Tesla']]
    \end{lstlisting}
    \item Access (i.e., read) an entry in a list. Note that indexing starts at 0:
    \begin{lstlisting}[language=Python,numbers=none]
>>> another_list[3]
    \end{lstlisting}
    \begin{lstlisting}[style=PyOutput]
'Hello'
    \end{lstlisting}
  \end{itemize}
 \end{frame}
 
 \begin{frame}[fragile]
   \frametitle{Lists in Python (2)}
   \begin{itemize}
    \item Manipulate the value of an entry goes likewise:
    \begin{lstlisting}[language=Python,numbers=none]
>>> another_list[3] = 'Bye' # Becomes: [0.0, 0.1, 0.2, 'Bye', ['Audi', ...]]
    \end{lstlisting}
    \item Slicing is used to retrieve multiple elements:
    \begin{lstlisting}[language=Python,numbers=none]
>>> another_list[1:4] # This will give the elements from index 1 to index 3
    \end{lstlisting}
    \begin{lstlisting}[style=PyOutput]
[0.1, 0.2, 'Bye']
    \end{lstlisting}
    \item Lists can be unpacked into individual variables:
    \begin{lstlisting}[language=Python,numbers=none]
>>> a,b,c,d,e = brands
>>> print(f"The first list element was {a}, then {b}, {c}, {d} and finally {e}.")
    \end{lstlisting}
    \begin{lstlisting}[style=PyOutput]
The first list element was Audi, then Toyota, Honda, Ford and finally Tesla.
    \end{lstlisting}
    \item From here onwards, we will omit the \lstinline|print| statements from the slides
   \end{itemize}\vskip1em
  \end{frame}
 
  \begin{frame}[fragile]
    \frametitle{Lists in Python (3)}
    \begin{itemize}
      \item Lists can be concatenated or repeated by the addition and multiplication operators respectively: 
      \begin{lstlisting}[language=Python,numbers=none]
  >>> more_brands = ['Nissan','Kia'] + brands
      \end{lstlisting}
      \begin{lstlisting}[style=PyOutput]
  ['Nissan', 'Kia', 'Audi', 'Toyota', 'Honda', 'Ford', 'Tesla']
      \end{lstlisting}
      \begin{lstlisting}[language=Python,numbers=none]
  >>> zeros = 10*[0]
      \end{lstlisting}
      \begin{lstlisting}[style=PyOutput]
  [0, 0, 0, 0, 0, 0, 0, 0, 0, 0]
      \end{lstlisting}
      \item Find out which methods can be performed on a list by using \lstinline|dir(more_brands)|:
      \begin{lstlisting}
 more_brands.append('Volvo') # Append object (here: string literal) at the end of the list 
 more_brands.insert(1,'BMW') # Insert object at index 1
 more_brands.sort()          # Sorts the list in-place
 item = more_brands.pop(3)   # Removes element at index 3 from the list, stores it as item
 \end{lstlisting}
    \end{itemize}
   \end{frame}


 \begin{frame}[fragile]
  \frametitle{Getting started}
   \begin{itemize}
    \item Start the Python REPL (read–eval–print loop) by running \lstinline|python| or \lstinline|ipython|
    \item Enter the following commands on the command line. Evaluate the output.
   \end{itemize}
   \pause
\begin{pyblock}
print("This is an arbitrary print-statement from pyconsole")
\end{pyblock}
\end{frame}


 %% PRINTING
 \begin{frame}[fragile]
  \frametitle{Printing and formatting results}
  In Python, you can control the display format of the output of a command using various formatting methods such as `str.format()`, f-strings, or by using formatting specifiers while printing with `print()`. These methods only involve changing the way the numbers are displayed, not the underlying representation.
  
  \begin{lstlisting}[language=Python,numbers=none]
>> # Using str.format()
>> a = 19/4
>> print("{:.2f}".format(a)) # 2 decimal places
>> b = a**(-6)
>> print("{:.10f}".format(b)) # 10 decimal places
>> 
>> # Using f-strings (Python 3.6+)
>> c = (21)**0.5
>> print(f"{c:.10f}") # 10 decimal places
>> d = (2.71828)**(-c) 
>> print(f"{d:.2e}") # Scientific notation with 2 decimal places
>> 
>> # Using print formatting
>> print("{:.2f}".format(d)) # 2 decimal places
  \end{lstlisting}
\end{frame}

 
 {\nologo
 \begin{frame}[fragile]
  \frametitle{A slide without a logo}
  \begin{itemize}
    \item Using the \keystroke{$\uparrow$} and \keystroke{$\downarrow$} keys, you can cycle through recent commands
    \item Typing part of a command and pressing \keystroke{Tab} completes the command and lists the possibilities
    \item If a computation takes too long, you can press \keystroke{Ctrl}+\keystroke{C} to stop the program and return to the command line. Note that you may end up with incomplete results in the workspace.
    \item Sequences of commands (programs, scripts) are contained as py-files, plain text files with the \lstinline$.py$ extension.
    \item Python scripts can also be contained in jupyter notebooks, which have extension \lstinline$.ipynb$.
    \item Such py-files must be in the \emph{current working directory} or in the Python \emph{path}, the locations where Python searches for a command. If you try to run a script that is not in the path, Python will throw an Exception/Error.
    \item Anything following a \lstinline$#$ symbol is regarded as a comment
    \item There are several keyboard shortcuts (vary with text editor) that will make coding much more efficient.
  \end{itemize}
\end{frame}
}
