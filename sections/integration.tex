\part{Numerical integration}
\section{Introduction}
\subsection*{General}
\begin{frame}[label=contents_integration]
  \frametitle{Today's outline}
  \mode<beamer>{
    \only<1>{\tableofcontents}
  }
  \only<2>{\tableofcontents[currentsection,currentsubsection]}
\end{frame}

\frame{
  \frametitle{What is numerical integration?}
  To determine the integral $I(x)$ of an integrand $f(x)$, which can be used to compute the area underneath the integrand between $x=a$ and $x=b$.
  \[
    I(x) = \int_a^b f(x)dx
  \]
  \pause
  Today we will outline different numerical integration methods.
  \vskip2em
  \begin{itemize}
    \item Riemann integrals
    \item Trapezoidal rule
    \item Simpson's rule
  \end{itemize}
}
\begin{frame}
  \frametitle{Why do chemical engineers need integration?}
  \begin{itemize}
    \colorize<1> \item Obtaining the cumulative particle size distribution from a particle size distribution
    \colorize<2> \item The concentration outflow over time may be integrated to yield the residence time distribution
    \colorize<3> \item Integration of a varying product outflow yields the total product outflow
    \colorize<4> \item Quantitative analysis of mixture components via e.g. GC/MS
    \colorize<5> \item Not all function have an explicit antiderivative, e.g. $\int e^{x^2} dx$ or $\int \frac{1}{\ln x}dx$
  \end{itemize}
\end{frame}

\section{Riemann integrals}
\againframe<2>{contents_integration}
\begin{frame}
  \frametitle{Riemann integrals}
  \footnotesize\selectfont
  \tikz{\node[emphblock,text width=\textwidth] {Basic idea: Subdivide the interval $[a,b]$ into $n$ subintervals of equal length $\Delta x = \frac{b-a}{n}$ and use the sum of area to approximate the integral.};}
  \pause
  \begin{columns}[T]
 \column{0.33\textwidth}
    \begin{center}
      Left endpoint rule \vskip1em
      \begin{tikzpicture}[scale=0.5]
          \draw[gridline,step=1] (0,0) grid (5,5);
  \coordinate (O) at (0,0);
  % Axes
  \draw[line,->] (-0.3,0) -- (5,0) coordinate[label = {below:$x$}] (xmax);
  \draw[line,->] (0,-0.3) -- (0,5) coordinate[label = {left:$f(x)$}] (ymax);
  
  % Labels
  \draw (0,0) node[below left]{$0$};
  \foreach \s in {1,...,4}
  {
    \draw (\s,0) node[below]{$\s$};
    \draw (0,\s) node[left]{$\s$};
    }
    
    % Title
    %       \draw (2.5,5) node[above]{$f(x)=\frac{x^3}{2}-\frac{10x^2}{3}+\frac{11x}{2}+1$};
    
    % Plots
    \node[interp] (x0) at (0,1) {};
    \node[interp] (x1) at (1,3.667) {};
    \node[interp] (x2) at (2,2.667) {};
    \node[interp] (x3) at (3,1) {};
    \node[interp] (x4) at (4,1.667) {};
    \node (bb) at (5,5.3) {};
%       \draw [graph,domain=0:4.69] plot (\x, {0.5*\x*\x*\x-(10/3)*\x*\x+5.5*\x+1});
%       \draw [graph,opacity=0.4,dashed,domain=-0.15:4.73] plot (\x, {0.5*\x*\x*\x-(10/3)*\x*\x+5.5*\x+1});

        
        \draw [intblock] (1,0) rectangle (x0);
        \draw [intblock] (2,0) rectangle (x1);
        \draw [intblock] (3,0) rectangle (x2);
        \draw [intblock] (4,0) rectangle (x3);
        
        \draw [interp,smooth,domain=0:4.69] plot (\x, {0.5*\x*\x*\x-(10/3)*\x*\x+5.5*\x+1});
        \draw [interp,smooth,opacity=0.4,dashed,domain=-0.15:4.73] plot (\x, {0.5*\x*\x*\x-(10/3)*\x*\x+5.5*\x+1});
      \end{tikzpicture}
      \[
        L_n = \sum_{i=1}^n f(x_{i-1})\Delta x_i
      \]
    \end{center}
    \pause
    \column{0.33\textwidth}
    \begin{center}
      Right endpoint rule \vskip1em
      \begin{tikzpicture}[scale=0.5]
          \draw[gridline,step=1] (0,0) grid (5,5);
  \coordinate (O) at (0,0);
  % Axes
  \draw[line,->] (-0.3,0) -- (5,0) coordinate[label = {below:$x$}] (xmax);
  \draw[line,->] (0,-0.3) -- (0,5) coordinate[label = {left:$f(x)$}] (ymax);
  
  % Labels
  \draw (0,0) node[below left]{$0$};
  \foreach \s in {1,...,4}
  {
    \draw (\s,0) node[below]{$\s$};
    \draw (0,\s) node[left]{$\s$};
    }
    
    % Title
    %       \draw (2.5,5) node[above]{$f(x)=\frac{x^3}{2}-\frac{10x^2}{3}+\frac{11x}{2}+1$};
    
    % Plots
    \node[interp] (x0) at (0,1) {};
    \node[interp] (x1) at (1,3.667) {};
    \node[interp] (x2) at (2,2.667) {};
    \node[interp] (x3) at (3,1) {};
    \node[interp] (x4) at (4,1.667) {};
    \node (bb) at (5,5.3) {};
%       \draw [graph,domain=0:4.69] plot (\x, {0.5*\x*\x*\x-(10/3)*\x*\x+5.5*\x+1});
%       \draw [graph,opacity=0.4,dashed,domain=-0.15:4.73] plot (\x, {0.5*\x*\x*\x-(10/3)*\x*\x+5.5*\x+1});

        \draw [intblock] (0,0) rectangle (x1);
        \draw [intblock] (1,0) rectangle (x2);
        \draw [intblock] (2,0) rectangle (x3);
        \draw [intblock] (3,0) rectangle (x4);
        
        \draw [interp,smooth,domain=0:4.69] plot (\x, {0.5*\x*\x*\x-(10/3)*\x*\x+5.5*\x+1});
        \draw [interp,smooth,opacity=0.4,dashed,domain=-0.15:4.73] plot (\x, {0.5*\x*\x*\x-(10/3)*\x*\x+5.5*\x+1});
      \end{tikzpicture}
      \[
        R_n = \sum_{i=1}^n f(x_i)\Delta x_i
      \]

    \end{center}
    \pause
    \column{0.33\textwidth}
      \begin{center}
        Midpoint rule \vskip1em
        \begin{tikzpicture}[scale=0.5]
            \draw[gridline,step=1] (0,0) grid (5,5);
  \coordinate (O) at (0,0);
  % Axes
  \draw[line,->] (-0.3,0) -- (5,0) coordinate[label = {below:$x$}] (xmax);
  \draw[line,->] (0,-0.3) -- (0,5) coordinate[label = {left:$f(x)$}] (ymax);
  
  % Labels
  \draw (0,0) node[below left]{$0$};
  \foreach \s in {1,...,4}
  {
    \draw (\s,0) node[below]{$\s$};
    \draw (0,\s) node[left]{$\s$};
    }
    
    % Title
    %       \draw (2.5,5) node[above]{$f(x)=\frac{x^3}{2}-\frac{10x^2}{3}+\frac{11x}{2}+1$};
    
    % Plots
    \node[interp] (x0) at (0,1) {};
    \node[interp] (x1) at (1,3.667) {};
    \node[interp] (x2) at (2,2.667) {};
    \node[interp] (x3) at (3,1) {};
    \node[interp] (x4) at (4,1.667) {};
    \node (bb) at (5,5.3) {};
%       \draw [graph,domain=0:4.69] plot (\x, {0.5*\x*\x*\x-(10/3)*\x*\x+5.5*\x+1});
%       \draw [graph,opacity=0.4,dashed,domain=-0.15:4.73] plot (\x, {0.5*\x*\x*\x-(10/3)*\x*\x+5.5*\x+1});

          
          \draw [intblock] (0,0) rectangle (1,2.9792);
          \node [intdot] at (0.5, 2.9792) {};
          \draw [intblock] (1,0) rectangle (2,3.4375);
          \node [intdot] at (1.5, 3.4375) {};
          \draw [intblock] (2,0) rectangle (3,1.7292);
          \node [intdot] at (2.5, 1.7292) {};
          \draw [intblock] (3,0) rectangle (4,0.8542);
          \node [intdot] at (3.5, 0.8542) {};
          
          \draw [interp,smooth,domain=0:4.69] plot (\x, {0.5*\x*\x*\x-(10/3)*\x*\x+5.5*\x+1});
          \draw [interp,smooth,opacity=0.4,dashed,domain=-0.15:4.73] plot (\x, {0.5*\x*\x*\x-(10/3)*\x*\x+5.5*\x+1});
        \end{tikzpicture}
        \[
          M_n = \sum_{i=1}^n f(\bar{x}_i)\Delta x_i
        \]
        with $\bar{x}_i = \frac{x_{i-1}+x_i}{2}$
      \end{center}
  \end{columns}
\end{frame}

{\nologo
\begin{frame}
  \frametitle{Errors in Riemann integrals}
  We define the exact integral as $ I = \int_a^b f(x)dx$, and $L_n$, $R_n$ and $M_n$ represent the left, right and midpoint rule approximations of $I$ based on $n$ intervals.
  \vskip1em \pause
  Writing $f^{(k)}_\text{max}$ for the maximum value of the $k$-th derivative, the upper-bounds of the errors by Riemann integrals are: \vskip1em
  \begin{itemize}
    \colorize<2> \item $\displaystyle \abs{I - L_n} \leq \frac{f^{(1)}_\text{max} (b - a)^2}{2n}$
    \colorize<3> \item $\displaystyle \abs{I - R_n} \leq \frac{f^{(1)}_\text{max} (b - a)^2}{2n}$
    \colorize<4> \item $\displaystyle \abs{I - M_n} \leq \frac{f^{(2)}_\text{max} (b - a)^3}{24n^2}$
  \end{itemize}
  \vskip1em
  \onslide<5>{%
  Note that while $\abs{I - L_n}$ and $\abs{I - R_n}$ give the same \emph{upper-bounds} of the error, this does not mean the same error. Rather, the error is of opposite sign!}
\end{frame}
}

\section{Trapezoid rule}
\againframe<2>{contents_integration}
\begin{frame}
  \frametitle{Trapezoid rule}
  \footnotesize\selectfont
  Since the sign of the approximation error of the left and right endpoint rules is opposite, we can take the average of these approximations:
  \[
    T_n = \frac{L_n + R_n}{2}
  \]
  \pause
  \begin{columns}
\column{0.5\textwidth}
  The total area is obtained by geometric reconstruction of trapezoids:
  \[
    T_n = \sum_{i=1}^n \frac{f(x_{i+1})+f(x_i)}{2}\Delta x_i
  \]
\onslide<3>{%
Note that this can be rewritten for equidistant intervals:
\begin{equation*}
  T_n = \frac{b-a}{2n} \left( f(x_0) + 2f(x_1) + \ldots + 2f(x_{n-1}) + f(x_n) \right)
\end{equation*}}
\column{0.5\textwidth}
   \begin{tikzpicture}[scale=0.8]
          \draw[gridline,step=1] (0,0) grid (5,5);
  \coordinate (O) at (0,0);
  % Axes
  \draw[line,->] (-0.3,0) -- (5,0) coordinate[label = {below:$x$}] (xmax);
  \draw[line,->] (0,-0.3) -- (0,5) coordinate[label = {left:$f(x)$}] (ymax);
  
  % Labels
  \draw (0,0) node[below left]{$0$};
  \foreach \s in {1,...,4}
  {
    \draw (\s,0) node[below]{$\s$};
    \draw (0,\s) node[left]{$\s$};
    }
    
    % Title
    %       \draw (2.5,5) node[above]{$f(x)=\frac{x^3}{2}-\frac{10x^2}{3}+\frac{11x}{2}+1$};
    
    % Plots
    \node[interp] (x0) at (0,1) {};
    \node[interp] (x1) at (1,3.667) {};
    \node[interp] (x2) at (2,2.667) {};
    \node[interp] (x3) at (3,1) {};
    \node[interp] (x4) at (4,1.667) {};
    \node (bb) at (5,5.3) {};
%       \draw [graph,domain=0:4.69] plot (\x, {0.5*\x*\x*\x-(10/3)*\x*\x+5.5*\x+1});
%       \draw [graph,opacity=0.4,dashed,domain=-0.15:4.73] plot (\x, {0.5*\x*\x*\x-(10/3)*\x*\x+5.5*\x+1});

        \draw [intblock] (0,0) -- (1,0) -- (x1.center) -- (x0.center) -- cycle;
        \draw [intblock] (1,0) -- (2,0) -- (x2.center) -- (x1.center) -- cycle;
        \draw [intblock] (2,0) -- (3,0) -- (x3.center) -- (x2.center) -- cycle;
        \draw [intblock] (3,0) -- (4,0) -- (x4.center) -- (x3.center) -- cycle;
        
        \draw [interp,smooth,domain=0:4.69] plot (\x, {0.5*\x*\x*\x-(10/3)*\x*\x+5.5*\x+1});
        \draw [interp,smooth,opacity=0.4,dashed,domain=-0.15:4.73] plot (\x, {0.5*\x*\x*\x-(10/3)*\x*\x+5.5*\x+1});
    \end{tikzpicture}
\end{columns}
\end{frame}

\begin{frame}
  \frametitle{Error in trapezoid integration}
  The trapezoid rule result over $n$ intervals $T_n$ approximates the exact integral $I=\int_a^b f(x)dx$. The upper-bounds of the error is given as:  
  \[
    \abs{I - T_n} \leq \frac{f^{(2)}_\text{max} (b-a)^3}{12n^2}
  \]
  \pause
  Recall that the midpoint rule approximates with an upper-bound error of
  \[
    \abs{I - M_n} \leq \frac{f^{(2)}_\text{max} (b - a)^3}{24n^2}
  \]
  \pause
  \tikz{\node[emphblock,text width=\textwidth] {The midpoint rule approximation has lower error bounds than the trapezoid rule. A linear function is, however, better approximated by the trapezoid rule.}; }
\end{frame}

\section{Simpson's rule}
\againframe<2>{contents_integration}
\begin{frame}
  \frametitle{Towards higher-order integration}
  Compare how the midpoint and trapezoid functions behave on convex and concave parts of a graph.\vskip1em
  \begin{columns}
  \column{0.5\textwidth}
  \centering Midpoint rule
  \begin{tikzpicture}[scale=0.6]
        \draw[gridline,step=1] (0,0) grid (5,5);
  \coordinate (O) at (0,0);
  % Axes
  \draw[line,->] (-0.3,0) -- (5,0) coordinate[label = {below:$x$}] (xmax);
  \draw[line,->] (0,-0.3) -- (0,5) coordinate[label = {left:$f(x)$}] (ymax);
  
  % Labels
  \draw (0,0) node[below left]{$0$};
  \foreach \s in {1,...,4}
  {
    \draw (\s,0) node[below]{$\s$};
    \draw (0,\s) node[left]{$\s$};
    }
    
    % Title
    %       \draw (2.5,5) node[above]{$f(x)=\frac{x^3}{2}-\frac{10x^2}{3}+\frac{11x}{2}+1$};
    
    % Plots
    \node[interp] (x0) at (0,1) {};
    \node[interp] (x1) at (1,3.667) {};
    \node[interp] (x2) at (2,2.667) {};
    \node[interp] (x3) at (3,1) {};
    \node[interp] (x4) at (4,1.667) {};
    \node (bb) at (5,5.3) {};
%       \draw [graph,domain=0:4.69] plot (\x, {0.5*\x*\x*\x-(10/3)*\x*\x+5.5*\x+1});
%       \draw [graph,opacity=0.4,dashed,domain=-0.15:4.73] plot (\x, {0.5*\x*\x*\x-(10/3)*\x*\x+5.5*\x+1});

      
      \draw<2> [intblock,draw=tueorange,fill=tueorange] (0,0) rectangle (1,2.9792);
      \node<2> [intdot,draw=tueorange,fill=tueorange] at (0.5, 2.9792) {};
      \draw<1> [intblock] (0,0) rectangle (1,2.9792);
      \node<1> [intdot] at (0.5, 2.9792) {};
      \draw<1> [intblock] (1,0) rectangle (2,3.4375);
      \node<1> [intdot] at (1.5, 3.4375) {};
      \draw<1> [intblock] (2,0) rectangle (3,1.7292);
      \node<1> [intdot] at (2.5, 1.7292) {};
%       \draw<1> [intblock] (3,0) rectangle (4,0.8542);
%       \node<1> [intdot] at (3.5, 0.8542) {};
      \draw [intblock] (3,0) rectangle (4,0.8542);
      \node [intdot] at (3.5, 0.8542) {};
      
      \draw [interp,smooth,domain=0:4.69] plot (\x, {0.5*\x*\x*\x-(10/3)*\x*\x+5.5*\x+1});
      \draw [interp,smooth,opacity=0.4,dashed,domain=-0.15:4.73] plot (\x, {0.5*\x*\x*\x-(10/3)*\x*\x+5.5*\x+1});
    \end{tikzpicture}
  \column{0.5\textwidth}
  \centering Trapezoid rule
    \begin{tikzpicture}[scale=0.6]
        \draw[gridline,step=1] (0,0) grid (5,5);
  \coordinate (O) at (0,0);
  % Axes
  \draw[line,->] (-0.3,0) -- (5,0) coordinate[label = {below:$x$}] (xmax);
  \draw[line,->] (0,-0.3) -- (0,5) coordinate[label = {left:$f(x)$}] (ymax);
  
  % Labels
  \draw (0,0) node[below left]{$0$};
  \foreach \s in {1,...,4}
  {
    \draw (\s,0) node[below]{$\s$};
    \draw (0,\s) node[left]{$\s$};
    }
    
    % Title
    %       \draw (2.5,5) node[above]{$f(x)=\frac{x^3}{2}-\frac{10x^2}{3}+\frac{11x}{2}+1$};
    
    % Plots
    \node[interp] (x0) at (0,1) {};
    \node[interp] (x1) at (1,3.667) {};
    \node[interp] (x2) at (2,2.667) {};
    \node[interp] (x3) at (3,1) {};
    \node[interp] (x4) at (4,1.667) {};
    \node (bb) at (5,5.3) {};
%       \draw [graph,domain=0:4.69] plot (\x, {0.5*\x*\x*\x-(10/3)*\x*\x+5.5*\x+1});
%       \draw [graph,opacity=0.4,dashed,domain=-0.15:4.73] plot (\x, {0.5*\x*\x*\x-(10/3)*\x*\x+5.5*\x+1});

      \draw<2> [intblock,draw=tueorange,fill=tueorange] (0,0) -- (1,0) -- (x1.center) -- (x0.center) -- cycle;
      \draw<1> [intblock] (0,0) -- (1,0) -- (x1.center) -- (x0.center) -- cycle;
      \draw<1> [intblock] (1,0) -- (2,0) -- (x2.center) -- (x1.center) -- cycle;
      \draw<1> [intblock] (2,0) -- (3,0) -- (x3.center) -- (x2.center) -- cycle;
%       \draw<1> [intblock] (3,0) -- (4,0) -- (x4.center) -- (x3.center) -- cycle;
      \draw [intblock] (3,0) -- (4,0) -- (x4.center) -- (x3.center) -- cycle;
      
      \draw [interp,smooth,domain=0:4.69] plot (\x, {0.5*\x*\x*\x-(10/3)*\x*\x+5.5*\x+1});
      \draw [interp,smooth,opacity=0.4,dashed,domain=-0.15:4.73] plot (\x, {0.5*\x*\x*\x-(10/3)*\x*\x+5.5*\x+1});
  \end{tikzpicture}
  \end{columns}
  \pause
  \tikz{\node[emphblock,text width=\textwidth]{\color{tueorange}In convex parts (bending down), the midpoint rule tends to overestimate the integral (trapezoid underestimates). \color{tuesteel}In concave parts (bending up), the midpoint rule tends to underestimate the integral (trapezoid overestimates).};}
\end{frame}

{\nologo
\begin{frame}
  \frametitle{Towards higher-order integration}
  The errors of the midpoint rule and trapezoid rule behave in a similar way, but have opposite signs.
  \begin{itemize}
    \item Midpoint: $\displaystyle \abs{I - M_n} \leq \frac{f^{(2)}_\text{max}(b - a)^3}{24n^2}$
    \item Trapezoid: $\displaystyle \abs{I - T_n} \leq \frac{f^{(2)}_\text{max} (b-a)^3}{12n^2}$
  \end{itemize}
  \pause
  For a quadratic function, the errors relate as:
  \[
    \abs{I-M_n} = \frac{1}{2}\abs{I-T_n} 
    \]
    \pause
    Taking the weighted average of these two yields the Simpson's rule:
    \[
      S_{2n} = \frac{2}{3}M_n + \frac{1}{3}T_n
      \]
      The $2n$ means we have $2n$ subintervals: the $n$ trapezoid intervals are subdivided by the midpoint rule.
\end{frame}
}
    
    \begin{frame}
      \frametitle{Simpson's rule}
  Consider the interval $i\in[x_0,x_2]$, subdivided in three equidistant interpolation points: $x_0,x_1,x_2$. \vskip1ex
  \begin{itemize}
    \item Midpoint: $\displaystyle M_i = f(\frac{x_0+x_2}{2})2\Delta x = f(x_1)2\Delta x$
    \item Trapezoid: $\displaystyle T_i = \frac{f(x_0)+f(x_2)}{2}2\Delta x$
    \item Simpson:  $\displaystyle S_i = \frac{2}{3}M_i + \frac{1}{3}T_i$
  \end{itemize}
  Note that $M_i$ and $T_i$ were computed on interval $x_2-x_0=2\Delta x$. 
  \vskip1ex \pause
  Now we have:
  \begin{align*}
    S_i &= \frac{2}{3}\left[f(x_1)2\Delta x\right] + \frac{1}{3}\left[\frac{f(x_0)+f(x_2)}{2}2\Delta x\right] \\
    &= \frac{4\Delta x}{3}f(x_1) + \frac{\Delta x}{3}f(x_0)+f(x_2)
    \onslide<3>{=\color{scharlaken} \frac{\Delta x}{3}\left( f(x_0) + 4f(x_1) + f(x_2)\right)}
  \end{align*}
\end{frame}

\begin{frame}
  \frametitle{Simpson's rule}
  We write $f(x_k) = f_k$. The integral of an interval $i\in[x_0,x_2]$ is approximated as:
  \[
    S_i = \frac{\Delta x}{3}\left( f_0 + 4f_1 + f_2\right)
  \]
  \pause
  The next interval, $S_{j}$ with $j\in[x_2,x_4]$ with midpoint $x_3=\frac{x_2+x_4}{2}$ is approximated as:
  \[
    S_j = \frac{\Delta x}{3}\left( f_2 + 4f_3 + f_4\right)
  \]
  \pause
  If we sum these two intervals we obtain:
  \begin{align*}
    I \approx S_i + S_j &= \left[\frac{\Delta x}{3}\left( f_0 + 4f_1 + f_2\right)\right] +  \left[\frac{\Delta x}{3}\left( f_2 + 4f_3 + f_4\right)\right] \\
    &= \frac{\Delta x}{3}\left(f_0 + 4f_1+2f_2 +4f_3 + f4 \right)
  \end{align*}
\end{frame}
\begin{frame}
  \frametitle{Simpson's rule}
  \rowcolors[]{50}{white}{white}
  In general, Simpson's rule can be written as:
%   \hspace*{-2em}
  \begin{flalign*}
    \int_a^b f(x)dx &\approx \sum^n_{\begin{matrix}
                             k=2 & \\
                             k\, \text{even}
                           \end{matrix}} \frac{\Delta x}{3}\left( f_{k-2} + 4f_{k-1} + f_k\right) &\\
                           &= \frac{\Delta x}{3}\left(f_0 + 4f_1 +2f_2+4f_3+2f_4+\ldots+2f_{n-2}+4f_{n-1}+f_n \right) &
  \end{flalign*}
  \pause
  The error is given by:
  \[
    \abs{I-S_n} \leq \frac{f^{(4)}_\text{max}(b-a)^5}{180n^4}
  \]
  if integrand $f$ is differentiable on $[a,b]$.
\end{frame}

{\nologo
\begin{frame}
  \frametitle{Simpson's rule: example}
  \footnotesize\selectfont
  Recall our example data, described by $f(x)=\frac{x^3}{2}-\frac{10x^2}{3}+\frac{11x}{2}+1$ \\
  $ I = \int_0^4 \frac{x^3}{2}-\frac{10x^2}{3}+\frac{11x}{2}+1 = \frac{80}{9} \approx 8.888\ldots$
  \begin{columns}
    \column{0.5\textwidth}
    \begin{itemize}
      \item<2-> \color{scharlaken}Interpolating $x_0$, $x_1$ and $x_2$:  $ p_{2a}(x) = -\frac{11}{6}x^2+4\frac{1}{2}x+1 $ \\
      $\int_0^2 p_{2a} = \frac{55}{9} \approx 6.1111$
      \item<3-> \color{tuesteel} Interpolating $x_2$, $x_3$ and $x_4$: $ p_{2b}(x) = \frac{7x^2}{6}-7\frac{1}{2}x+13 $ \\
      $\int_2^4 p_{2b} = \frac{25}{9} \approx 2.777\ldots$
      \item<4-> Adding the separate integrals: \\
      $\int_0^2 p_{2a} + \int_2^4 p_{2b} = \frac{80}{9}$
    \end{itemize}
    \column{0.50\textwidth}
    \centering
    \begin{tikzpicture}[domain=-1:6,scale=0.8]
        \draw[gridline,step=1] (0,0) grid (5,5);
  \coordinate (O) at (0,0);
  % Axes
  \draw[line,->] (-0.3,0) -- (5,0) coordinate[label = {below:$x$}] (xmax);
  \draw[line,->] (0,-0.3) -- (0,5) coordinate[label = {left:$f(x)$}] (ymax);
  
  % Labels
  \draw (0,0) node[below left]{$0$};
  \foreach \s in {1,...,4}
  {
    \draw (\s,0) node[below]{$\s$};
    \draw (0,\s) node[left]{$\s$};
    }
    
    % Title
    %       \draw (2.5,5) node[above]{$f(x)=\frac{x^3}{2}-\frac{10x^2}{3}+\frac{11x}{2}+1$};
    
    % Plots
    \node[interp] (x0) at (0,1) {};
    \node[interp] (x1) at (1,3.667) {};
    \node[interp] (x2) at (2,2.667) {};
    \node[interp] (x3) at (3,1) {};
    \node[interp] (x4) at (4,1.667) {};
    \node (bb) at (5,5.3) {};
%       \draw [graph,domain=0:4.69] plot (\x, {0.5*\x*\x*\x-(10/3)*\x*\x+5.5*\x+1});
%       \draw [graph,opacity=0.4,dashed,domain=-0.15:4.73] plot (\x, {0.5*\x*\x*\x-(10/3)*\x*\x+5.5*\x+1});

      % Polynomial interpolant
      \draw<2->[graph,domain=0:2,draw=scharlaken,fill=scharlaken,fill opacity=0.5] (0,0) -- plot (\x, {-(11/6)*\x*\x+(4.5)*\x+1}) -- (2,0) -- cycle;
      \draw<3->[graph,domain=2:4,draw=tuesteel,fill=tuesteel,fill opacity=0.5] (2,0) -- plot (\x, {(7/6)*\x*\x-7.5*\x + 13}) -- (4,0) -- cycle;
    \end{tikzpicture}
  \end{columns}
  \onslide<5->{
  Using Simpson's rule:
  $ I \approx \frac{\Delta x}{3}\left(f_0 + 4f_1 +2f_2+4f_3+f_4 \right) = \frac{1}{3}\left(1+4\cdot3.6667+2\cdot2.6667+4\cdot1.0000+1.6667\right) = 8.88888 = \frac{80}{9}$
  }
  \onslide<6>{\tikz{\node[emphblock,text width=\textwidth]{Simpson's method is of fourth order, and it gives exact approximations of third order polynomials!};}}
\end{frame}}

\begin{frame}[fragile]
  \frametitle{Integration in Python}
  Integration can be done numerically in Python.
  \begin{itemize}
    \item \lstinline$np.trapz(y, x)$ uses the trapezoid rule to integrate the data. Make sure you use the \lstinline$x$ variable if your data is not spaced with $\Delta x=1$. Can handle non-equidistant data.
    \begin{lstlisting}
import numpy as np
x = np.linspace(-2, 2, 2001)
y = 1 / (x**2 + 1)
I = np.trapz(y, x) # Or: scipy.integrate.trapezoid
print(I)
    \end{lstlisting}
    \begin{lstlisting}[style=PyOutput]
2.214297328921525
    \end{lstlisting}
    \item Integration of functions can be done using the \lstinline$quad(func, a, b)$ function:
    \begin{lstlisting}
import numpy as np
from scipy.integrate import quad
f = lambda x: np.exp(-x**2)
I, err = quad(f, 0, 10)
print(I, err)
\end{lstlisting}
    \begin{lstlisting}[style=PyOutput]
0.886226925452758 1.8483380528941764e-13
    \end{lstlisting}
  \end{itemize}
\end{frame}

\section{Conclusion}
\againframe<2>{contents_integration}
\begin{frame}
  \frametitle{What hasn't been discussed?}
  This course is by no means complete, and further reading is possible.
   \begin{itemize}
      \item Gaussian quadrature: A third-order integration method that requires only two base points (in contrast to the third order Simpson's method, which requires three points)
      \item Adaptive techniques: Parts of a function that are relatively steady (no wild oscillations) and differentiable can be integrated with much larger step sizes than other parts of the function.
      \item Simpson's 3/8-rule: Yet another integration technique, requiring an additional data point
   \end{itemize}
\end{frame}

\begin{frame}
  \frametitle{Summary}
  \begin{itemize}
    \item Several techniques for numerical integration were discussed:
    \begin{itemize}
      \item Riemann sums, trapezoid rule, Simpson's rule
      \item Upper-bound errors were given for each technique
      \item Built-in Python functions were illustrated
    \end{itemize}
    \item Continue with characterization of convergence of the integration methods in the tutorials!
  \end{itemize}
\end{frame}

\section{Tutorials}
\subsection*{Integration tutorials}
{\nologo
\begin{frame}[fragile]
  \frametitle{Integration tutorials}
  \begin{enumerate}
    \item Implement a function to integrate a mathematical function for a specific number of integration intervals. Implement it as a function, which can be called with arguments:
    \begin{itemize}
      \item Function (handle) to integrate
      \item Integration boundaries (as separate arguments or as a $2 \times 1$ numpy array)
      \item Number of integration intervals
    \end{itemize}
    For instance: \lstinline$def leftrule(func, x0, x1, N):$.\\
    \item Set up a function to integrate:
    \begin{lstlisting}
def myfunction(x):
    return x**2 - 4*x + 6 + np.sin(5*x)
    \end{lstlisting}
    \item Integrate the function, e.g. \lstinline$int_left = leftrule(myfunction, 0, 10, 25)$
    \item Assess how the number of intervals affects the deviation from the true integral value. 
    \item Create a log-log plot of the deviation vs. number of intervals used.
    \item Do this for all methods discussed\footnote{Riemann left, right, midpoint, trapezoid, and Simpson} and compare their performance in a graph
  \end{enumerate}
\end{frame}
}
