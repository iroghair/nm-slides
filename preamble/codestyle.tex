% \lstset{language=Matlab,%
%     %basicstyle=\color{red},
%     basicstyle=\scriptsize\ttfamily,
%     breaklines=true,%
%     morekeywords={matlab2tikz},
%     keywordstyle=\color{blue},%
%     morekeywords=[2]{1}, keywordstyle=[2]{\color{black}},
%     identifierstyle=\color{black},%
%     stringstyle=\color{mylilas},
%     commentstyle=\color{mygreen},%
%     showstringspaces=false,%without this there will be a symbol in the places where there is a space
%     numbers=none,%
% %     numberstyle={\tiny \color{black}},% size of the numbers
% %     numbersep=-2pt, % this defines how far the numbers are from the text
% %     emph=[1]{for,end,break},emphstyle=[1]\color{red}, %some words to emphasise
% emph=[2]{ones,int,str2double,long,single,simplify,diff,log,atan,solve,vpa,syms,doc,int,simplify,diff,log,atan,syms,interp3,interpn,histogram,ribbon,contourf,fzero,feval,fminsearch,fsolve,fminbnd,ezplot,varargin,optimset,odeset,ode15s,plotyy,ones,linprog,cftool,optimset,lsqnonlin}, emphstyle=[2]{\color{blue}},
%     backgroundcolor=\color{gray!15},frame=tlbr, framerule=0pt,
%     escapeinside={(*@}{@*)}
% }

% Settings for listings package
\lstset{
  language=Python,                % choose the language of the code
  basicstyle=\scriptsize\ttfamily,% the size of the fonts that are used for the code
  % basicstyle=\fontfamily{pcr}\selectfont\footnotesize,
  numbers=left,                   % where to put the line-numbers
  numberstyle=\tiny\color{gray},  % the size of the fonts for the line-numbers
  stepnumber=1,                   % the step between two line-numbers
  numbersep=5pt,                  % how far the line-numbers are from the code
  showspaces=false,               % show spaces adding particular underscores
  showstringspaces=false,         % underline spaces within strings
  showtabs=false,                 % show tabs within strings adding particular underscores
  frame=single,                   % adds a frame around the code
  tabsize=4,                      % sets default tab size to 2 spaces
  captionpos=b,                   % sets the caption-position to bottom
  breaklines=true,                % automatic line breaking only at whitespace
  breakatwhitespace=false,        % automatic breaks should only happen at whitespace
  escapeinside={(*@}{@*)},         % if you want to add LaTeX within your code
  keywordstyle=\color{scharlaken}\bfseries,  % keyword style
  commentstyle=\color{tuesteel}\mdseries,     % comment style
  stringstyle=\color{tueorange},        % string literal style
  morekeywords={str, int, float, list, dict, bool,list, True, False, tuple,with,as,dir,min,max,sum},
  % frameround=tttt,
  backgroundcolor=\color{gray!15},frame=tlbr, framerule=0.5pt,
  rulecolor=\color{gray!30},
  columns=fullflexible
}

\lstdefinestyle{tiny}{
basicstyle=\tiny\ttfamily
}

\lstdefinestyle{PyOutput}{
  basicstyle=\scriptsize\ttfamily,% the size of the fonts that are used for the code
  % basicstyle=\fontfamily{pcr}\selectfont\footnotesize,
  numbers=none,                   % where to put the line-numbers
  frame=single,                   % adds a frame around the code
  tabsize=4,                      % sets default tab size to 2 spaces
  breaklines=true,                % automatic line breaking only at whitespace
  breakatwhitespace=false,        % automatic breaks should only happen at whitespace
  escapeinside={(*@}{@*)},         % if you want to add LaTeX within your code
  keywordstyle=\color{black},  % keyword style
  commentstyle=\color{black},     % comment style
  stringstyle=\color{black},        % string literal style
  backgroundcolor=\color{white!15},frame=tlbr, framerule=0.5pt,
  rulecolor=\color{black!30}
}