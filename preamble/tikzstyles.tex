\usetikzlibrary{decorations.pathmorphing} % Drawing
\usetikzlibrary{decorations.shapes}
\usetikzlibrary{patterns}
\usetikzlibrary{positioning}
\usetikzlibrary{shadows}
% \usetikzlibrary{snakes}
\usetikzlibrary{calc}
\usetikzlibrary{arrows}
\usetikzlibrary{fit}
\usetikzlibrary{fadings}
\usetikzlibrary{matrix}
\usetikzlibrary{plotmarks}
\usetikzlibrary{shapes}
\usetikzlibrary{shadings}
\usetikzlibrary{intersections}
% Blocks
\tikzset{block/.style={rectangle,draw=maincolor,fill=maincolor!20,text width=10em,text centered,rounded corners,minimum height=4em,thick}}
\tikzset{emphblock/.style={rectangle,draw=maincolor,text centered,rounded corners,thick,top color=maincolor!10,bottom color=maincolor!30}}
\tikzset{emphblocko/.style={rectangle,draw=tueorange,text centered,rounded corners,thick,top color=tueorange!10,bottom color=tueorange!30}}
\tikzset{emphblocky/.style={rectangle,draw=tueyellow,text centered,rounded corners,thick,top color=tueyellow!10,bottom color=tueyellow!30}}
% Dots
\tikzset{dot/.style={draw=scharlaken,circle,thick,minimum size=1mm,inner sep=0pt,outer sep=0pt,fill=white}}
\tikzset{fdot/.style={circle,draw=black,fill=black,inner sep=1.5pt}}
\tikzset{gdot/.style={circle,draw=black,inner sep=3pt}}
\tikzset{cross/.style={cross out, draw=black, fill=none, minimum size=2*(#1-\pgflinewidth), inner sep=0pt, outer sep=0pt}, cross/.default={4pt}}
% Graphs and lines
\tikzset{line/.style={black,>=stealth',semithick}}
\tikzset{graph/.style={smooth,samples=400,scharlaken,semithick}}
\tikzset{interp/.style={dot,draw=scharlaken,inner sep=1.5pt,minimum size=4pt,color=scharlaken,fill=none}}
\tikzset{intblock/.style={line,draw=tuesteel,fill=tuesteel!50!white,fill opacity=0.3,opacity=0.6}}
\tikzset{intdot/.style={line,dot,draw=tuesteel,fill=tuesteel,opacity=0.6}}
\tikzset{gridline/.style={lightgray,ultra thin,dashed}}
\newcommand{\tikzmark}[2]{\tikz[overlay,remember picture,
  baseline=(#1.base)] \node (#1) {#2};}

  \tikzset{style tueturq/.style={
    set fill color=tueturq!60,
    set border color=white,
  },
  style tuesteel/.style={
    set fill color=tuesteel!60,
    set border color=white,
  },
  style white/.style={
    set fill color=white!60,
    set border color=white,
  },
  style tueorange/.style={
    set fill color=tueorange!60,
    set border color=white,
  },
  style yellow/.style={
    set fill color=tueyellow!60,
    set border color=white,
  },
  mat/.style={
    above left offset={-0.15,0.38},
    below right offset={0.15,-0.125},
    #1
  },
  mat2/.style={
    above left offset={-0.15,0.31},
    below right offset={0.15,-0.125},
    #1
  },
  txt/.style={
    above left offset={-0.1,0.34},
    below right offset={0.15,-0.15},
    #1
  }
}

\newcolumntype{L}[1]{>{\raggedright\arraybackslash}p{#1}}
\newcolumntype{R}[1]{>{\raggedleft\arraybackslash}p{#1}}

% \pgfplotsset{
% % every axis y label/.append style={at={axis cs:14,14},rotate=0,anchor=south east}
% exery axis/.style={ylabel near ticks},
% }
\tikzset{
  invisible/.style={opacity=0},
  visible on/.style={alt={#1{}{invisible}}},
  alt/.code args={<#1>#2#3}{%
    \alt<#1>{\pgfkeysalso{#2}}{\pgfkeysalso{#3}} % \pgfkeysalso doesn't change the path
  },
}

% Keyboard strokes
\newcommand*\keystroke[1]{%
  \tikz[baseline=(key.base)]
    \node[%
      draw,
      fill=white,
      drop shadow={shadow xshift=0.25ex,shadow yshift=-0.25ex,fill=black,opacity=0.75},
      rectangle,
      rounded corners=2pt,
      inner sep=1pt,
      line width=0.5pt,
      font=\scriptsize\sffamily
    ](key) {\ensuremath{\ }#1\strut\ensuremath{\ }}
  ;
}


% \newcommand{\ExternalLink}{% from: https://tex.stackexchange.com/questions/99316/symbol-for-external-links
%     \tikz[x=1.2ex, y=1.2ex, baseline=-0.05ex]{% 
%         \begin{scope}[x=1ex, y=1ex]
%             \clip (-0.1,-0.1) 
%                 --++ (-0, 1.2) 
%                 --++ (0.6, 0) 
%                 --++ (0, -0.6) 
%                 --++ (0.6, 0) 
%                 --++ (0, -1);
%             \path[draw, 
%                 line width = 0.5, 
%                 rounded corners=0.5] 
%                 (0,0) rectangle (1,1);
%         \end{scope}
%         \path[draw, line width = 0.5] (0.5, 0.5) 
%             -- (1, 1);
%         \path[draw, line width = 0.5] (0.6, 1) 
%             -- (1, 1) -- (1, 0.6);
%         }
%     }