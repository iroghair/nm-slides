\documentclass[11pt,table,final,fleqn,xcolor={usenames,dvipsnames,table},handout]{beamer}
\usetheme[]{Frankfurt}


\usepackage{listings}
\usepackage{multimedia} % Movies
\usepackage{fancyvrb,relsize}
\usepackage{commath}
\usepackage{graphicx}
\usepackage{array}
\usepackage{longtable}
\usepackage{algpseudocode} 
\usepackage{multirow}
\usepackage[math]{iwona}
\usepackage{wasysym}
% \usepackage[fleqn]{amsmath}
\usepackage{amssymb}
\usepackage{siunitx}
\usepackage{tikz} % Drawing
\usepackage{pgfplots}
\usepackage{pgfplotstable}
% \usepackage{fontspec}

% Presentation settings
\rowcolors[]{1}{maincolor!20}{maincolor!10}

\newcommand{\Fo}{\ensuremath{\mathit{Fo}}}

\lstset{language=Matlab,%
    %basicstyle=\color{red},
    basicstyle=\scriptsize\ttfamily,
    breaklines=true,%
    morekeywords={matlab2tikz},
    keywordstyle=\color{blue},%
    morekeywords=[2]{1}, keywordstyle=[2]{\color{black}},
    identifierstyle=\color{black},%
    stringstyle=\color{mylilas},
    commentstyle=\color{mygreen},%
    showstringspaces=false,%without this there will be a symbol in the places where there is a space
    numbers=none,%
%     numberstyle={\tiny \color{black}},% size of the numbers
%     numbersep=-2pt, % this defines how far the numbers are from the text
%     emph=[1]{for,end,break},emphstyle=[1]\color{red}, %some words to emphasise
emph=[2]{ones,int,str2double,long,single,simplify,diff,log,atan,solve,vpa,syms,doc,int,simplify,diff,log,atan,syms,interp3,interpn,histogram,ribbon,contourf,fzero,feval,fminsearch,fsolve,fminbnd,ezplot,varargin,optimset,odeset,ode15s,plotyy,ones,linprog,cftool,optimset,lsqnonlin}, emphstyle=[2]{\color{blue}},
    backgroundcolor=\color{gray!15},frame=tlbr, framerule=0pt,
    escapeinside={(*@}{@*)}
}

% To have the navigation circles without declaring subsections
\usepackage{remreset}% tiny package containing just the \@removefromreset command
\makeatletter
\@removefromreset{subsection}{section}
\makeatother
\setcounter{subsection}{1}

% For convenient figure inclusion
\DeclareGraphicsExtensions{.pdf,.png,.jpg}
\graphicspath{ {../img/} }


% \setmainfont{Yanone Kaffeesatz}
% \setmathfont(Digits,Latin,Greek)[Numbers={Lining,Proportional}]{Gentium Plus}

% TU/e colors
\definecolor{tuered}{RGB}{247,49,49}
\definecolor{tuefuchsia}{RGB}{214,0,74}
\definecolor{tuelila}{RGB}{214,0,123}
\definecolor{tuepurple}{RGB}{173,32,173}
\definecolor{tuedblue}{RGB}{16,16,115}
\definecolor{tueblue}{RGB}{0,102,204}
\definecolor{tuelblue}{RGB}{0,162,222}
\definecolor{tueorange}{RGB}{255,154,0}
\definecolor{tueyellow}{RGB}{255,221,0}
\definecolor{tuedyellow}{RGB}{206,223,0}
\definecolor{tuegreen}{RGB}{132,210,0}
\definecolor{tuedgreen}{RGB}{0,172,130}
\definecolor{tueblue2}{RGB}{0,146,181}

% For Matlab script colors
\definecolor{mygreen}{RGB}{28,172,0} % color values Red, Green, Blue
\definecolor{mylilas}{RGB}{170,55,241}

\makeatletter
% \definecolor{beamer@blendedblue}{rgb}{0.5,0.5,0.3} % changed this
\useoutertheme{smoothbars}
\useinnertheme{circles}

%%%%%%%%%%%%%%%%%%%%%%%%%%%%%%%%%%%%%%%%%%%%%%%%%%%%%%%%%%%%%%%%%%%%%%%%%%%
\definecolor{maincolor}{named}{tuelblue}
\definecolor{textcolorfg}{named}{white}
\definecolor{tuealert}{named}{tueblue}
%%%%%%%%%%%%%%%%%%%%%%%%%%%%%%%%%%%%%%%%%%%%%%%%%%%%%%%%%%%%%%%%%%%%%%%%%%%

\setbeamercolor{normal text}{fg=black,bg=white}
\setbeamercolor{alerted text}{fg=tuealert}
\setbeamercolor{example text}{fg=tuegreen!50!black}

\setbeamercolor{background canvas}{parent=normal text,bg=white}
\setbeamercolor{background}{parent=background canvas}

\setbeamercolor{title}{bg=maincolor,fg=textcolorfg} % Presentation title colors
\setbeamercolor{structure}{fg=maincolor,bg=textcolorfg}
\setbeamercolor{section in head/foot}{fg=textcolorfg,bg=maincolor}
\setbeamercolor{palette primary}{fg=textcolorfg,bg=maincolor} % changed this

\setbeamercolor{palette primary}{fg=maincolor,bg=textcolorfg} % changed this
\setbeamercolor{palette secondary}{use=structure,fg=structure.fg!100!tueblue} % changed this
\setbeamercolor{palette tertiary}{use=structure,fg=structure.fg!100!tuered} % changed this

\setbeamertemplate{navigation symbols}{} % ( Dont use )
\setbeamercolor{navigation symbols}{use=structure,fg=structure.fg!40!bg}
\setbeamercolor{navigation symbols dimmed}{use=structure,fg=structure.fg!20!bg}

\setbeamercolor{block title}{fg=textcolorfg,bg=maincolor}
\setbeamercolor{block body}{fg=black,bg=maincolor!10}

\def\colorize<#1>{%
  \temporal<#1>{\color{tuedblue!40!gray!40}}{\color{tuealert}}{\color{black}}}
  
\setlength{\mathindent}{0pt}

\makeatother

% Colored urls
\hypersetup{colorlinks,linkcolor=,urlcolor=tueblue}

% Vector format
\renewcommand{\vec}[1]{\mathbf{#1}}

\usetikzlibrary{decorations} % Drawing
\usetikzlibrary{patterns}
\usetikzlibrary{positioning}
\usetikzlibrary{shadows}
\usetikzlibrary{calc}
\usetikzlibrary{arrows}
\usetikzlibrary{decorations}
\usetikzlibrary{plotmarks}
\usetikzlibrary{shapes}
\usetikzlibrary{shadings}
\usetikzlibrary{intersections}
% Blocks
\tikzset{block/.style={rectangle,draw=maincolor,fill=maincolor!20,text width=10em,text centered,rounded corners,minimum height=4em,thick}}
\tikzset{emphblock/.style={rectangle,draw=maincolor,text centered,rounded corners,thick,top color=maincolor!10,bottom color=maincolor!30}}
% Dots
\tikzset{dot/.style={draw=tuered,circle,thick,minimum size=1mm,inner sep=0pt,outer sep=0pt,fill=white}}
\tikzset{fdot/.style={circle,draw=black,fill=black,,inner sep=1.5pt}}
\tikzset{gdot/.style={circle,draw=black,inner sep=3pt}}
\tikzset{cross/.style={cross out, draw=black, fill=none, minimum size=2*(#1-\pgflinewidth), inner sep=0pt, outer sep=0pt}, cross/.default={4pt}}
% Graphs and lines
\tikzset{line/.style={black,>=stealth',semithick}}
\tikzset{graph/.style={smooth,samples=400,tuered,semithick}}
\tikzset{interp/.style={dot,draw=tuealert,inner sep=1.5pt,minimum size=4pt,color=tuealert,fill=none}}
\tikzset{intblock/.style={line,draw=tuefuchsia,fill=tuefuchsia!50!white,fill opacity=0.3,opacity=0.6}}
\tikzset{intdot/.style={line,dot,draw=tuefuchsia,fill=tuefuchsia,opacity=0.6}}
\tikzset{gridline/.style={lightgray,ultra thin,dashed}}


\newcolumntype{L}[1]{>{\raggedright\arraybackslash}p{#1}}
\newcolumntype{R}[1]{>{\raggedleft\arraybackslash}p{#1}}

% \pgfplotsset{
% % every axis y label/.append style={at={axis cs:14,14},rotate=0,anchor=south east}
% exery axis/.style={ylabel near ticks},
% }


\renewcommand*\familydefault{\sfdefault}  % Use sans font


\title{Ordinary differential equations}

\author[M.~Van Sint Annaland]{\underline{Martin van Sint Annaland}, Ivo~Roghair \\ \vskip1em {\scriptsize \underline{m.v.sintannaland@tue.nl}}}

\institute[SPI]{{Chemical Process Intensification,\\
  Eindhoven University of Technology}}

\date

% BEGIN PRESENTATION
\begin{document}
\sisetup{detect-all}
\lstset{language=Matlab,%
    %basicstyle=\color{red},
    basicstyle=\footnotesize\ttfamily,
    breaklines=true,%
    morekeywords={matlab2tikz},
    keywordstyle=\color{blue},%
    morekeywords=[2]{1}, keywordstyle=[2]{\color{black}},
    identifierstyle=\color{black},%
    stringstyle=\color{mylilas},
    commentstyle=\color{mygreen},%
    showstringspaces=false,%without this there will be a symbol in the places where there is a space
    numbers=none,%
%     numberstyle={\tiny \color{black}},% size of the numbers
%     numbersep=-2pt, % this defines how far the numbers are from the text
%     emph=[1]{for,end,break},emphstyle=[1]\color{red}, %some words to emphasise
    emph=[2]{ones,int,simplify,diff,log,atan,syms,doc,int,simplify,diff,log,atan,syms,interp3,interpn,histogram,ribbon,contourf,fzero,feval,fminsearch,fsolve,fminbnd,ezplot,varargin,optimset,odeset,plotyy,ones,linprog,cftool,optimset,lsqnonlin}, emphstyle=[2]{\color{blue}},
    backgroundcolor=\color{gray!15},frame=tlbr, framerule=0pt,
    escapeinside={(*@}{@*)}
}

\frame[plain]{
  \titlepage
}

% \renewcommand{\vec}[1]{\\boldsymbol{#1}}
\part{Explicit and Implicit methods}
\frame{\partpage}

\section{Introduction}
\subsection*{General}
\begin{frame}[label=contents]
  \frametitle{Today's outline}
  \mode<beamer>{
    \only<1>{\tableofcontents}
  }
  \only<2>{\tableofcontents[currentsection]}
\end{frame}

\begin{frame}
  \frametitle{Overview}
    \begin{block}{Ordinary differential equations}
      An equation containing a function of one independent variable and its derivatives, in contrast to a \emph{partial differential equation}, which contains derivatives with respect to more independent variables.
    \end{block}
    \pause
  \begin{block}{Main question}
  How to solve 
  \[
    \frac{d\vec{y}}{dx} = f(\vec{y}(x),x) \quad \text{with} \quad \vec{y}(x=0) = \vec{y}_0
  \]
  accurately and efficiently?
  \end{block}
\end{frame}

\begin{frame}
  \frametitle{What is an ODE?}
  \begin{itemize}
    \item Algebraic equation:
    \[
      f(y(x),x) = 0 \qquad \text{e.g.} \, -\ln(K_{eq})=(1-\zeta)
    \]
    \item First order ODE:
    \[
      f\left(\frac{dy}{dx}(x),y(x),x\right) = 0 \quad \text{e.g.} \, \frac{dc}{dt} = -kc^n
    \]
    \item Second order ODE:
    \[
      f\left(\frac{d^2y}{dx^2}(x),\frac{dy}{dx}(x),y(x),x \right) = 0 \quad \text{e.g.} \quad \mathcal{D}\frac{d^2c}{dx^2}= - \frac{kc}{1+Kc}
    \]
  \end{itemize}
\end{frame}

\begin{frame}
  \frametitle{About second order ODEs}
  Very often a second order ODE can be rewritten into a system of first order ODEs (whether it is handy depends on the boundary conditions!)
  \vskip1em
  \pause
  \mode<beamer>{
  \only<1-2>{
  \begin{block}{Example}
  Recall:
  \[
    \mathcal{D}\frac{d^2c}{dx^2}= - \frac{kc}{1+Kc}
  \]
  Define $y = -\mathcal{D} \frac{dc}{dx}$, then $\frac{dy}{dx}=\frac{kc}{1+Kc}$, thus solve system:
  \begin{align*}
    \frac{dc}{dx} &= -\frac{1}{\mathcal{D}}y \\
    \frac{dy}{dx} &= \frac{kc}{1+Kc}
  \end{align*}
  \end{block}}}
  \only<3>{
  \begin{block}{More general}
  Consider the second order ODE:
  \[
    \frac{d^2y}{dx^2} + q(x)\frac{dy}{dx} = r(x)
  \]
  Now define and solve using $z$ as a new variable:
  \begin{align*}
    \frac{dy}{dx} &= z(x) \\
    \frac{dz}{dx} &= r(x) - q(x)z(x)
  \end{align*}
  \end{block}}
  \vskip1em
  \pause
\end{frame}

\begin{frame}[label=ivpbvp]
  \frametitle{Importance of boundary conditions}
  \footnotesize\selectfont
  The nature of boundary conditions determines the appropriate numerical method. Classification into 2 main categories:
  \begin{itemize}
    \item \emph{Initial value problems (IVP)} \\
    We know the values of all $y_i$ at some starting position $x_s$, and it is desired to find the values of $y_i$ at some final point $x_f$. \vskip1em
    \begin{center}
      \begin{tikzpicture}[scale=5]
        \node[] (y1s) at (0,0.1) {$y_{1,s}$};
        \node[] (y1f) at (1,0.1) {\color{tuered}$y_{1,f}$};
        \node[] (y2s) at (0,0) {$y_{2,s}$};
        \node[] (y2f) at (1,0) {\color{tuered}$y_{2,f}$};
        \node[fdot] (xs) at (0,-0.1) {};
        \node[fdot] (xf) at (1,-0.1) {};
        \node[anchor=north] at (xs.south) {$x_s$};
        \node[anchor=north] at (xf.south) {$x_f$};
        \draw[line] (xs) -- (xf);
        \draw[line,->,densely dashed] (0.1,0.05) -- node[midway,above] {``Marching''} (0.9,0.05);
      \end{tikzpicture}
    \end{center}
    \item \emph{Boundary value problems (BVP)} \\
    Boundary conditions are specified at more than one $x$. Typically, some of the BC are specified at $x_s$ and the remainder at $x_f$. \vskip1em
    \centering
    \begin{tikzpicture}[scale=5]
      \node[] (y1s) at (0,0.1) {$y_{1,s}$};
\node[] (y1f) at (1,0.1) {\color{scharlaken}$y_{1,f}$};
\node[] (y2s) at (0,0) {\color{scharlaken}$y_{2,s}$};
\node[] (y2f) at (1,0) {$y_{2,f}$};
\node[fdot] (xs) at (0,-0.1) {};
\node[fdot] (xf) at (1,-0.1) {};
\node[anchor=north] at (xs.south) {$x_s$};
\node[anchor=north] at (xf.south) {$x_f$};
\draw[line] (xs) -- (xf);
\draw[line,->,densely dashed] (0.1,0.05) -- node[midway,above] {``Shooting''} (0.9,0.05);

\coordinate[] (b) at ($(y1f)!0.5!(y2f) + (0.1,0) $) {};
\coordinate[right of=b] (c1) {};
\coordinate[below=1.4cm] (c2) at (c1)  {};
%       \coordinate[below of=1cm] at (c2) (c5) {};

\coordinate[] (s) at ($(y1s)!0.5!(y2s) - (0.1,0) $) {};
\coordinate[left of=s] (c3) {};
\coordinate[below=1.4cm] (c4) at (c3)  {};
%       \coordinate[below of=c4] (c6) {};

%       \node[] (b) at ($(y1f)!0.5!(y2f)$) {};
\draw[line,->,densely dashed,draw=tuegreen,rounded corners=10pt] (b) -- (c1) -- (c2) -- (c4) -- (c3) -- (s);
%       \draw[line,densely dashed,draw=tuegreen] (y2f) to [controls=(1,-0.2) and +(-0.1,-0.2)] (y2s);
    \end{tikzpicture}
  \end{itemize}
\end{frame}

\begin{frame}
  \frametitle{Overview}
  Initial value problems:
  \begin{itemize}
    \colorize<2> \item Explicit methods
    \begin{itemize}
      \colorize<2> \item First order: forward Euler
      \colorize<2> \item Second order: improved Euler (RK2)
      \colorize<2> \item Fourth order: Runge-Kutta 4 (RK4)
      \colorize<2> \item Step size control
    \end{itemize}
    \colorize<3> \item Implicit methods
    \begin{itemize}
      \colorize<3> \item First order: backward Euler
      \colorize<3> \item Second order: midpoint rule
    \end{itemize}
  \end{itemize}
  \onslide<4>{
  Boundary value problems
  \begin{itemize}
    \colorize<4> \item Shooting method
  \end{itemize}
  }
\end{frame}

\section{Euler's method}
\againframe<2>{contents}
\subsection{Forward Euler}
\begin{frame}
  \frametitle{Euler's method}
  Consider the following single initial value problem:
  \[
    \frac{dc}{dt} = f(c(t),t) \quad \text{with} \quad c(t=0)=c_0 \quad \text{(initial value problem)}
  \]
  \pause
  Easiest solution algorithm: Euler's method, derived here via Taylor series expansion:
  \[
    c(t_0 + \Delta t) \approx c(t_0) + \left.\frac{dc}{dt}\right|_{t_0}\Delta t + \frac{1}{2} \left.\frac{d^2c}{dt^2}\right|_{t_0} \left(\Delta t\right) ^2 + \mathcal{O}{(\Delta t^3)}
  \]
  \pause
  Neglect terms with higher order than two: $\left. \frac{dc}{dt}\right|_{t_0} = \frac{c(t_0 + \Delta t) - c(t_0)}{\Delta t}$
  Substitution: 
  \[
    \frac{c(t_0 + \Delta t) - c(t_0)}{\Delta t} = f(c_0,t_0)\Rightarrow c(t_0+\Delta t) = c(t_0) + \Delta t f(c_0,t_0) 
  \]
\end{frame}

\begin{frame}[t,fragile]
  \frametitle{Euler's method: graphical example}
  \[
    \frac{c(t_0 + \Delta t) - c(t_0)}{\Delta t} = f(c_0,t_0)\Rightarrow c(t_0+\Delta t) = c(t_0) + \Delta t f(c_0,t_0) 
  \]
  \vskip1em
   \begin{tikzpicture}
      \begin{axis}[every axis/.append style={font=\footnotesize},
      width=\columnwidth, height=5.5cm,
      xmin=0,xmax=4,ymin=0,ymax=4,
      xtick={1,2,3},ytick={1,2,2.5},
      axis x line=middle,axis y line=middle,
      xlabel=$t$,xticklabels={$t_0$,$\Delta t$,$2\Delta t$},
      ylabel=$c(t)$,yticklabels={$c_0$,$c_1$,$c_2$},tick align=outside,ymajorgrids=true,major grid style={dashed}]

    \addplot[graph,quiver={u=\thisrow{u},v=\thisrow{v},scale arrows=0.4},-stealth']
      table {
      x y u v
      1 1 1 1
      2 2 1 0.5
      };
      \addplot[graph,sharp plot,densely dashed,mark=*,mark options={solid,fill=tuered},mark size=1.7pt,nodes near coords={\coordindex}]
      table {
      x y
      1 1
      2 2
      3 2.5
      };
      \node[black] at (axis cs:1.4,0.9) {$\left.\frac{dc}{dt}\right|_{t_0}$};
      \node[black] at (axis cs:2.4,1.8) {$\left.\frac{dc}{dt}\right|_{\Delta t}$};
%       \node[black] at (axis cs:2.1,0.5) {$k_1$};
    \end{axis}
  \end{tikzpicture}
\end{frame}

\begin{frame}
  \frametitle{Euler's method - solution method}
    \begin{tikzpicture}[snake=zigzag, line before snake = 5mm, line after snake = 5mm]
    %draw horizontal line   
    \draw (0,0) -- (6,0);
    \draw[snake] (5.5,0) -- (8.5,0);
    \draw (8,0) -- (10,0);

    %draw vertical lines
    \foreach \x in {0,2,4,6,8,10}
      \draw (\x cm,3pt) -- (\x cm,-3pt);

    %draw nodes
    \draw (0,0) node[below=3pt] {$ t=0 $} node[above=3pt] {$ 1 $};
    \draw (2,0) node[below=3pt] {$ \Delta t $} node[above=3pt] {$ 2 $};
    \draw (4,0) node[below=3pt] {$ 2\Delta t $} node[above=3pt] {$ 3 $};
    \draw (6,0) node[below=3pt] {$ 3\Delta t $} node[above=3pt] {$ 4 $};
    \draw (8,0) node[below=3pt] {$ t_\mathrm{end}-\Delta t $} node[above=3pt] {$ n-1 $};
    \draw (10,0) node[below=3pt] {$ t_\mathrm{end} $} node[above=3pt] {$ n $};
    
    \onslide<1>{
      \draw (0,-1.5) node[above=1pt] {$c_0$};
    }
    \onslide<2>{
      \draw (0,-1.5) node[above=1pt] {$c(t)$};
      \draw (2,-1.5) node[above=1pt] {$c(t+\Delta t)$};
    }
    \onslide<3>{
      \draw (2,-1.5) node[above=1pt] {$c(t)$};
      \draw (4,-1.5) node[above=1pt] {$c(t+\Delta t)$};
    }
    \onslide<4>{
      \draw (4,-1.5) node[above=1pt] {$c(t)$};
      \draw (6,-1.5) node[above=1pt] {$c(t+\Delta t)$};
    }
   \end{tikzpicture}
   \vskip1em
  Start with $t = t_0$, $c=c_0$, then calculate at discrete points in time: $c(t_1 = t_0 + \Delta t) = c(t_0) + \Delta t f(c_0,t_0)$. 
  \onslide<5>{
  \vskip1em
  \tikz{\node[emphblock,text width=\textwidth] {
  Pseudo-code Euler's method: $ \frac{dy}{dx} = f(x,y) \quad \text{and} \quad y(x_0) = y_0$.
  \begin{enumerate}
    \item Initialize variables, functions; set $h = \frac{x_1 - x_0}{N}$
    \item Set $x = x_0$, $y = y_0$
    \item While $x<x_\text{end}$ do\\
    $ \displaystyle x_{i+1} = x_i + h; \quad y_{i+1} = y_i + h f(x_i,y_i)$
  \end{enumerate}
  };}
  }
\end{frame}

\begin{frame}
  \frametitle{Euler's method - example}
  First order reaction in a batch reactor:
  \[
    \frac{dc}{dt} = -kc \quad \text{with} \quad c(t=0) = 1\, \si{\mole\per\cubic\meter}, \quad k = 1\, \si{s^{-1}}, \quad t_\text{end} = 2\, \si{\second}
  \]
  \pause
  \footnotesize\selectfont
  \begin{longtable}{p{0.3\textwidth}p{0.5\textwidth}}
  \hline
    Time [\si{\second}] & Concentration [\si{\mole\per\cubic\meter}] \\ \hline
    $t_0 = 0$                                 & $c_0 = 1.00$ \\
    $\begin{aligned}t_1 &= t_0 + \Delta t\\ &= 0 + 0.1 = 0.1\end{aligned}$    & $\begin{aligned}c_1 &= c_0 + \Delta t \cdot (-kc_0) \\ &= 1 + 0.1 \cdot (-1 \cdot 1) = 0.9\end{aligned}$ \\
    $\begin{aligned}t_2 &= t_1 + \Delta t\\ &= 0.1 + 0.1 = 0.2\end{aligned}$  & $\begin{aligned}c_2 &= c_1 + \Delta t \cdot (-kc_1)\\ &= 0.9 + 0.1 \cdot (-1 \cdot 0.9) = 0.81\end{aligned}$ \\ 
    $\begin{aligned} t_3 &= t_2 + \Delta t \\ &= 0.2 + 0.1 = 0.3\end{aligned}$  & $\begin{aligned}c_3 &= c_2 + \Delta t \cdot (-kc_2)\\ &= 0.81 + 0.1 \cdot (-1 \cdot 0.81) = 0.729 \end{aligned}$ \\ 
    \centering$\ldots$                                  & $\quad \quad \quad \quad \ldots$ \\
    $t_{i+1} = t_i + \Delta t$                & $c_{i+1} = c_i + \Delta t \cdot (-k c_i) $ \\
    \centering$\ldots$ & $\quad \quad \quad \quad \ldots$ \\
    $t_{20} = 2.0 $ & $c_{20} = c_{19} + \Delta t \cdot (-k c_{19}) = 0.1211577$ \\
    \hline
  \end{longtable}
\end{frame}

\begin{frame}
  \frametitle{Euler's method - example}
%   Accuracy: Results are shown for $N=20,4-,80,\ldots,320$ using $k = 1 \si{s^{-1}$
  \begin{center}
    \begin{tikzpicture}
      \begin{axis}[every axis/.append style={font=\footnotesize},
    width=\textwidth, height=7cm,     % size of the image
    grid = major,
    grid style={dashed, gray!30},
    xmin=0,     % start the diagram at this x-coordinate
    xmax=2,    % end   the diagram at this x-coordinate
    ymin=0,     % start the diagram at this y-coordinate
    ymax=1,   % end   the diagram at this y-coordinate
    xtick={0,0.2,...,2},
    ytick={0,0.2,...,1},
    axis background/.style={fill=white},
    axis x line=middle,
    axis y line=middle,
    ylabel style={at={(ticklabel* cs:1.05)},anchor=south west},
    ylabel=$c$ (\si{\mole\per\cubic\meter}),
    xlabel=$t$ (\si{\second}),
    tick align=outside,
    legend style={draw=none,fill=none,font=\tiny,at={(0.5,1.0)},anchor=south},
    legend columns=5
    ]

      \addplot[graph,mark=x] table [id=EE_20]{ODE_EulerExpl_20.dat};  \addlegendentry{20 points}
      \only<2->{\addplot[graph,draw=tueyellow] table [id=EE_20]{ODE_EulerExpl_40.dat}; } \addlegendentry{40 points}
      \only<3->{\addplot[graph,draw=tuelblue] table [id=EE_20]{ODE_EulerExpl_80.dat}; } \addlegendentry{80 points}
      \only<4->{\addplot[graph,draw=tuegreen] table [id=EE_20]{ODE_EulerExpl_160.dat};} \addlegendentry{160 points}
      \only<5->{\addplot[graph,draw=tuelila] table [id=EE_20]{ODE_EulerExpl_320.dat};} \addlegendentry{320 points}
    \end{axis}
    \end{tikzpicture}
  \end{center}
\end{frame}

\begin{frame}
  \frametitle{Problems with Euler's method}
  The question is: What step size, or how many steps to use?
  \begin{enumerate}
    \item \emph{Accuracy}  $\Rightarrow$ need information on numerical error!
    \item \emph{Stability} $\Rightarrow$ need information on stability limits!
  \end{enumerate}
  \vskip1em

  \begin{columns}
    \column{0.5\textwidth}
    \begin{tikzpicture}
      \begin{axis}[every axis/.append style={font=\tiny},
      width=\columnwidth, height=5.5cm,     % size of the image
      grid = major,
      grid style={dashed, gray!30},
      xmin=0,     % start the diagram at this x-coordinate
      xmax=2,    % end   the diagram at this x-coordinate
      ymin=0,     % start the diagram at this y-coordinate
      ymax=1,   % end   the diagram at this y-coordinate
      xtick={0,0.5,...,2},
      ytick={0,0.25,...,1},
%       axis background/.style={fill=white},
%       axis x line=middle,
%       axis y line=middle,
      xlabel style={at={(ticklabel* cs:1.1)},anchor=south},
      xlabel=$t$ (\si{\second}),
      ylabel style={at={(yticklabel* cs:0.5)},anchor=north},
      ylabel=$c$ (\si{\mole\per\cubic\meter}),
      tick align=outside,
      legend style={draw=none,fill=none,font=\tiny,at={(0.5,-0.1)},anchor=north},
      legend columns=3
      ]

	\addplot[graph] table [id=EE_20]{ODE_EulerExpl_20.dat}; \addlegendentry{$N=20$}
	\addplot[graph,draw=tueyellow] table [id=EE_20]{ODE_EulerExpl_40.dat}; \addlegendentry{$N=40$}
	\addplot[graph,draw=tuelblue] table [id=EE_20]{ODE_EulerExpl_80.dat};  \addlegendentry{$N=80$}
	\addplot[graph,draw=tuegreen] table [id=EE_20]{ODE_EulerExpl_160.dat}; \addlegendentry{$N=160$}
	\addplot[graph,draw=tuelila] table [id=EE_20]{ODE_EulerExpl_320.dat}; \addlegendentry{$N=320$}
      \end{axis}
    \end{tikzpicture}
     \centering Reaction rate: $k = 1$ \si{s^{-1}}
    \column{0.5\textwidth}
    \begin{tikzpicture}
      \begin{axis}[every axis/.append style={font=\tiny},
      width=\columnwidth, height=5.5cm,     % size of the image
      grid = major,
      grid style={dashed, gray!30},
      xmin=0,     % start the diagram at this x-coordinate
      xmax=0.25,    % end   the diagram at this x-coordinate
      ymin=-5,     % start the diagram at this y-coordinate
      ymax=5,   % end   the diagram at this y-coordinate
      restrict y to domain*=-300:300,
      xtick={0,0.05,...,0.25},
      ytick={-5,-2.5,...,5},
%       axis background/.style={fill=white},
% %       axis x line=middle,
%       axis y line=middle,
      xlabel style={at={(ticklabel* cs:1.1)},anchor=south},
      xlabel=$t$ (\si{\second}),
      ylabel style={at={(yticklabel* cs:0.5)},anchor=north},
      ylabel=$c$ (\si{\mole\per\cubic\meter}),
      tick align=outside,
      legend style={draw=none,fill=none,font=\tiny,at={(0.5,-0.1)},anchor=north},
      legend columns=3
      ]

	\addplot[graph,sharp plot,mark=+,mark size=1.3pt] table [id=EE_20]{ODE_Euler2Expl_20.dat}; \addlegendentry{$N=20$}
	\addplot[graph,sharp plot,draw=tueyellow,mark=x,mark size=1.3pt] table [id=EE_20]{ODE_Euler2Expl_40.dat}; \addlegendentry{$N=40$}
	\addplot[graph,sharp plot,draw=tuelblue,mark=o,mark size=1.3pt] table [id=EE_20]{ODE_Euler2Expl_80.dat};  \addlegendentry{$N=80$}
	\addplot[graph,sharp plot,draw=tuegreen] table [id=EE_20]{ODE_Euler2Expl_160.dat}; \addlegendentry{$N=160$}
	\addplot[graph,sharp plot,draw=tuelila] table [id=EE_20]{ODE_Euler2Expl_320.dat}; \addlegendentry{$N=320$}
      \end{axis}
    \end{tikzpicture}
     \centering Reaction rate: $k = 50$ \si{s^{-1}}
  \end{columns}
\end{frame}

\begin{frame}
  \frametitle{Accuracy}
  Comparison with analytical solution for $k=1$ \si{s^{-1}}:
  \[
    c(t) = c_0 \exp \left(-kt\right) \Rightarrow \zeta = 1-\exp\left(-kt\right) \Rightarrow \zeta_\text{analytical} = 0.864665
  \]
  \begin{columns}[T]
    \column{0.5\textwidth}
      \begin{longtable}{ccc}
      \hline
      $N$ & $\zeta$ & $\frac{\zeta^{}_\text{numerical}-\zeta_\text{analytical}}{\zeta_\text{analytical}}$ \\ \hline
      20  & 0.878423 & 0.015912 \\
      40  & 0.871488 & 0.007891 \\
      80  & 0.868062 & 0.003929 \\
      160 & 0.866360 & 0.001961 \\
      320 & 0.865511 & 0.000979\\
      \hline
    \end{longtable}
  \column{0.5\textwidth}
    \begin{tikzpicture}
      \begin{loglogaxis}[
      width=\columnwidth,height=5cm,
      ylabel style={at={(yticklabel* cs:0.5)},anchor=south},
      xlabel=N,
      ylabel=Relative error,
      log ticks with fixed point,
      xtick={20,40,80,160,320},
      ytick={0.02,0.01,0.005,0.002,0.001}]
      \addplot[graph,draw=tuelblue,mark=o] table {ODE_euler_err.dat};
      \end{loglogaxis}
    \end{tikzpicture}
  \end{columns}
\end{frame}

\begin{frame}
  \frametitle{Accuracy}
  \tikz{\node[emphblock,text width=\textwidth] {For Euler's method: Error halves when the number of grid points is doubled, i.e. error is proportional to $\Delta t$: first order method.};}
  \vskip1em
  Error estimate:
  \[
    \left. \frac{dx}{dt}\right|_{t_0} = \frac{x(t_0+\Delta t)-x(t_0)}{\Delta t} + \frac{1}{2} \left.\frac{d^2x}{dt^2}\right|_{t_0} (\Delta t) + \mathcal{O}{(\Delta t)\color{tuered}^2}
  \]
  \[
    \frac{x(t_0+\Delta t)-x(t_0)}{\Delta t} = f(x_0,t_0) - \frac{1}{2} \left.\frac{d^2x}{dt^2}\right|_{t_0} (\Delta t) +  \mathcal{O}{(\Delta t)\color{tuered}^2}
  \]

\end{frame}

% \subsection{Convergence rate}
% \begin{frame}
%   \frametitle{Errors and convergence rate}
%   \begin{block}{$L_2$ norm (Euclidean norm)}
%     $ \quad \norm{\vec{v}}_2 = \sqrt{v_1^2+v_2^2+\ldots+v_n^2} = \sqrt{\sum_{i=1}^n v_i^2} $
%   \end{block}
%   \begin{block}{$L_\infty$ norm (maximum norm)}
%     $ \quad \norm{\vec{v}}_\infty = \text{max}\left( \abs{v_1},\ldots,\abs{v_n}\right) $
%   \end{block}
%   \begin{block}{Absolute difference}
%     $ \quad \epsilon_\text{abs} = \norm{\vec{y}_\text{numerical} - \vec{y}_\text{analytical}}_{2,\infty} $
%   \end{block}
%   \begin{block}{Relative difference}
%     $ \quad \epsilon_\text{rel} = \frac{\norm{\vec{y}_\text{numerical} - \vec{y}_\text{analytical}}_{2,\infty}}{\norm{\vec{y}_\text{analytical}}_{2,\infty}} $
%   \end{block}
% \end{frame}
% 
% \begin{frame}
%   \frametitle{Errors and convergence rate}
%   \footnotesize\selectfont
%    \begin{block}{Convergence rate (or: order of convergence) $r$}
%   $\displaystyle \epsilon = \lim_{\Delta x \rightarrow 0} c(\Delta x)^r $
%   \end{block}
%   \begin{itemize}
%     \item A first order method reduces the error by a factor 2 when increasing the number of steps by a factor 2
%     \item A second order method reduces the error by a factor 4 when increasing the number of steps by a factor 2
%   \end{itemize}
%   What to do when there is no analytical solution available?
%   \pause
%   Compare to calculations with different number of steps: $\epsilon_1 = c(\Delta x_1)^r $ and $\epsilon_2 = c(\Delta x_2)^r $ and solve for $r$: \\ \vfill
%   $ \displaystyle
%     \frac{\epsilon_2}{\epsilon_1} = \frac{c(\Delta x_2)^r}{c(\Delta x_1)^r} = \left(\frac{\Delta x_2}{\Delta x_1}\right)^r \Rightarrow \log\left( \frac{\epsilon_2}{\epsilon_1}\right) = \log\left( \frac{\Delta x_2}{\Delta x_1}\right)^r $ 
%   \vfill
%   \tikz{\node[emphblock,text width=\textwidth] {$ \displaystyle
%     \Rightarrow r = \frac{\log\left( \frac{\epsilon_2}{\epsilon_1}\right)}{\log\left( \frac{\Delta x_2}{\Delta x_1}\right)} = \frac{\log\left( \frac{\epsilon_2}{\epsilon_1}\right)}{\log\left( \frac{N_1}{N_2}\right)} \quad \text{in the limit of} \quad \Delta x \rightarrow 0 \qquad \text{or} \qquad N \rightarrow \infty $};}
% \end{frame}

\section{Rates of convergence}
\againframe<2>{contents}
\subsection*{Rate of convergence}
\begin{frame}
  \frametitle{Errors and convergence rate}
  \begin{block}{$L_2$ norm (Euclidean norm)}
    $ \quad \norm{\vec{v}}_2 = \sqrt{v_1^2+v_2^2+\ldots+v_n^2} = \sqrt{\sum_{i=1}^n v_i^2} $
  \end{block}
  \begin{block}{$L_\infty$ norm (maximum norm)}
    $ \quad \norm{\vec{v}}_\infty = \text{max}\left( \abs{v_1},\ldots,\abs{v_n}\right) $
  \end{block}
  \begin{block}{Absolute difference}
    $ \quad \epsilon_\text{abs} = \norm{\vec{y}_\text{numerical} - \vec{y}_\text{analytical}}_{2,\infty} $
  \end{block}
  \begin{block}{Relative difference}
    $ \quad \epsilon_\text{rel} = \frac{\norm{\vec{y}_\text{numerical} - \vec{y}_\text{analytical}}_{2,\infty}}{\norm{\vec{y}_\text{analytical}}_{2,\infty}} $
  \end{block}
\end{frame}

\begin{frame}
  \frametitle{Errors and convergence rate}
  \footnotesize\selectfont
   \begin{block}{Convergence rate (or: order of convergence) $r$}
  $\displaystyle \epsilon = \lim_{\Delta x \rightarrow 0} c(\Delta x)^r $
  \end{block}
  \begin{itemize}
    \item A first order method reduces the error by a factor 2 when increasing the number of steps by a factor 2
    \item A second order method reduces the error by a factor 4 when increasing the number of steps by a factor 2
  \end{itemize}
\end{frame}

\begin{frame}
  \frametitle{Computing the rate of convergence}
  \footnotesize\selectfont
  When the analytical solution is available, choose \tikz{\node[circle,draw=none,fill=tuelblue,inner sep=0.7pt,text=white]{\scriptsize 1};} or \tikz{\node[circle,draw=none,fill=tuelblue,inner sep=0.7pt,text=white]{\scriptsize 2};} for a particular number of grid points $N$:
  \begin{enumerate}
  \footnotesize\selectfont
    \item Compute the relative or absolute error vector $\overline{\varepsilon}$. Take the norm to compute a single error value $\epsilon$ following:
    \begin{itemize}
    \footnotesize\selectfont
      \item Based on $L_1$-norm: $\displaystyle \epsilon = \frac{\left\Vert\mathbf{\overline{\varepsilon}}\right\Vert_1}{N}$ \vskip1ex
      \item Based on $L_2$-norm: $\displaystyle \epsilon = \frac{\left\Vert\mathbf{\overline{\varepsilon}}\right\Vert_2}{\sqrt{N}}$\vskip1ex
      \item Based on $L_\infty$-norm: $\displaystyle \epsilon = \left\Vert\mathbf{\overline{\varepsilon}}\right\Vert_\infty$
    \end{itemize}
    \item Compute the relative or absolute error at a single indicative points (e.g. middle of domain, outlet).
  \end{enumerate}
  \pause
  Compare to calculations with different number of steps: $\epsilon_1 = c(\Delta x_1)^r $ and $\epsilon_2 = c(\Delta x_2)^r $ and solve for $r$: \\ \vfill
  $ \displaystyle
    \frac{\epsilon_2}{\epsilon_1} = \frac{c(\Delta x_2)^r}{c(\Delta x_1)^r} = \left(\frac{\Delta x_2}{\Delta x_1}\right)^r \Rightarrow \log\left( \frac{\epsilon_2}{\epsilon_1}\right) = \log\left( \frac{\Delta x_2}{\Delta x_1}\right)^r $ 
  \vfill
  \tikz{\node[emphblock,text width=0.9\textwidth] {$ \displaystyle
    \Rightarrow r = \frac{\log\left( \frac{\epsilon_2}{\epsilon_1}\right)}{\log\left( \frac{\Delta x_2}{\Delta x_1}\right)} = \frac{\log\left( \frac{\epsilon_2}{\epsilon_1}\right)}{\log\left( \frac{N_1}{N_2}\right)} \quad \text{in the limit of} \quad \Delta x \rightarrow 0 \quad \text{or} \quad N \rightarrow \infty $};}
\end{frame}

\begin{frame}
  \frametitle{Computing the rate of convergence}
  \footnotesize\selectfont
  When the analytical solution is \textbf{not} available:
  \begin{enumerate}
  \footnotesize\selectfont
    \item Compute the solution with $N+1$, $N$, $N-1$ and $N-2$ grid points
    \item Select a single indicative grid point (e.g. middle of domain, outlet) that lies at exactly the same position in each computation
    \item Use the solution $c$ at this grid point for various grid sizes to compute:
    \[\displaystyle
       r = \dfrac{\log \dfrac{c_{N+1}  - c_N}{c_N - c_{N-1}}} {\log \dfrac{c_N - c_{N-1}}{c_{N-1} - c_{N-2}}}
    \]
   \end{enumerate} 
\end{frame}

\begin{frame}
  \frametitle{Example: Euler's method --- order of convergence}
  \begin{longtable}{cccc}
    \hline
    $N$ & $\zeta$ & $\frac{\zeta^{}_\text{numerical}-\zeta_\text{analytical}}{\zeta_\text{analytical}}$ & $ r = \frac{\log\left(\frac{\epsilon_i}{\epsilon_{i-1}}\right)}{\log \left( \frac{N_{i-1}}{N_i}\right)} $ \\ \hline
    20  & 0.878423 & 0.015912 & ---\\
    40  & 0.871488 & 0.007891 & 1.011832\\
    80  & 0.868062 & 0.003929 & 1.005969\\
    160 & 0.866360 & 0.001961 & 1.002996\\
    320 & 0.865511 & 0.000979 & 1.001500\\
    \hline
  \end{longtable}
  \pause
  $ \Rightarrow$ Euler's method is a first order method (as we already knew from the truncation error analysis) \\
  \pause \vskip1em
  Wouldn't it be great to have a method that can give the answer using much less steps? \pause $ \Rightarrow$ Higher order methods
\end{frame}

\section{Runge-Kutta methods}
\againframe<2>{contents}
\subsection{RK2 methods}
\begin{frame}[t,fragile]
\frametitle{Runge-Kutta methods}
  Propagate a solution by combining the information of several Euler-style steps (each involving one function evaluation) to match a Taylor series expansion up to some higher order.
  \vskip1em
  Euler: $ y_{i+1} = y_i + h f(x_i,y_i) $ with $h = \Delta x$, i.e. $\text{slope} = k_1 = f(x_i,y_i)$.\\
  \begin{columns}
    \column{0.5\textwidth}
    \begin{center}Euler's method\end{center}
    \begin{tikzpicture}
      \begin{axis}[every axis/.append style={font=\footnotesize},
      width=1.3\columnwidth, height=4cm,
      xmin=0,xmax=4,ymin=0,ymax=4,
      xtick={1,2,3},ytick={0},
      axis x line=middle,axis y line=middle,
      xlabel=$x$,xticklabels={$x_1$,$x_2$,$x_3$},
      ylabel=$y(x)$,tick align=outside]

    \addplot[graph,quiver={u=\thisrow{u},v=\thisrow{v},scale arrows=0.4},-stealth']
      table {
      x y u v
      1 1 1 1
      2 2 1 0.5
      };
      \addplot[graph,sharp plot,densely dashed,mark=*,mark options={solid,fill=tuered},mark size=1.7pt,nodes near coords={\coordindex}]
      table {
      x y
      1 1
      2 2
      3 2.5
      };
    \end{axis}
  \end{tikzpicture}
  \column{0.5\textwidth}
    \begin{center}RK2 method\end{center}
    \begin{tikzpicture}
      \begin{axis}[every axis/.append style={font=\footnotesize},
      width=1.3\columnwidth, height=4cm,
      xmin=0,xmax=4,ymin=0,ymax=1.1,
      xtick={1,2,3},ytick={0},
      axis x line=middle,axis y line=middle,
      xlabel=$x$,xticklabels={$x_0$,$x_1$,$x_2$},
      ylabel=$y(x)$,tick align=outside]

      \draw[line,-,black,densely dotted] (axis cs:1,1) -- (axis cs:2,0.2);
      \draw[line,-,black,densely dotted] (axis cs:2,0.6) -- (axis cs:3,0.2);

    \addplot[graph,quiver={u=\thisrow{u},v=\thisrow{v},scale arrows=0.7},-stealth',mark=*,nodes near coords={\coordindex}]
      table {
      x   y     u    v
      1   1     0.5   -0.4
      2   0.6   1 -0.4
      3  0.45   1  0
      };
      \addplot[graph,quiver={u=\thisrow{u},v=\thisrow{v},scale arrows=0.4},-stealth',mark=*,mark options={solid,fill=white}]
      table {
      x   y     u    v
      2   0.2   1   -0.4
      3   0.2   1.75   0
      };
      \addplot[graph,sharp plot,densely dashed]
      table {
      x   y     u    v
      1   1     0.5   -0.5
      2   0.6   1 -0.2
      3  0.45   1  0
      };
      \node[black] at (axis cs:1.1,0.75) {$k_1$};
      \node[black] at (axis cs:2.4,0.20) {$k_2$};
      \node[black] at (axis cs:1.7,0.9) {$\frac{k_1+k_2}{2}$};
%       \node[black] at (axis cs:2.6,0.4) {$k_2$};
      \end{axis}
  \end{tikzpicture}
  \end{columns}
\end{frame}
% \begin{frame}[t,fragile]
% \frametitle{Runge-Kutta methods}
%   Propagate a solution by combining the information of several Euler-style steps (each involving one function evaluation) to match a Taylor series expansion up to some higher order.
%   \vskip1em
%   Euler: $ y_{i+1} = y_i + h f(x_i,y_i) $ with $h = \Delta x$, i.e. $\text{slope} = k_1 = f(x_i,y_i)$.\\
%   \begin{columns}
%     \column{0.5\textwidth}
%     \begin{center}Euler's method\end{center}
%     \begin{tikzpicture}
%       \begin{axis}[every axis/.append style={font=\footnotesize},
%       width=1.3\columnwidth, height=4cm,
%       xmin=0,xmax=4,ymin=0,ymax=4,
%       xtick={1,2,3},ytick={0},
%       axis x line=middle,axis y line=middle,
%       xlabel=$x$,xticklabels={$x_1$,$x_2$,$x_3$},
%       ylabel=$y(x)$,tick align=outside]
% 
%     \addplot[graph,quiver={u=\thisrow{u},v=\thisrow{v},scale arrows=0.4},-stealth']
%       table {
%       x y u v
%       1 1 1 1
%       2 2 1 0.5
%       };
%       \addplot[graph,sharp plot,densely dashed,mark=*,mark options={solid,fill=tuered},mark size=1.7pt,nodes near coords={\coordindex}]
%       table {
%       x y
%       1 1
%       2 2
%       3 2.5
%       };
%     \end{axis}
%   \end{tikzpicture}
%   \column{0.5\textwidth}
%     \begin{center}RK2 method\end{center}
%     \begin{tikzpicture}
%       \begin{axis}[every axis/.append style={font=\footnotesize},
%       width=1.3\columnwidth, height=4cm,
%       xmin=0,xmax=4,ymin=0,ymax=1.1,
%       xtick={1,1.5,2,2.5,3},ytick={0},
%       axis x line=middle,axis y line=middle,
%       xlabel=$x$,xticklabels={$x_0$,$x_\frac{1}{2}$,$x_1$,$x_\frac{3}{2}$,$x_2$},
%       ylabel=$y(x)$,tick align=outside]
% 
%     \addplot[graph,quiver={u=\thisrow{u},v=\thisrow{v},scale arrows=0.4},-stealth',mark=*,nodes near coords={$c_\coordindex$}]
%       table {
%       x   y     u    v
%       1   1     0.5   -0.5
%       2   0.6 1 -0.2
%       3  0.45  1  0
%       };
%       \addplot[graph,quiver={u=\thisrow{u},v=\thisrow{v},scale arrows=0.4},-stealth',mark=*,mark options={solid,fill=white}]
%       table {
%       x   y     u    v
%       1.5 0.5 1   -0.4
%       2.5   0.5 1 -0.15
%       };
%       \addplot[graph,sharp plot,densely dashed]
%       table {
%       x   y     u    v
%       1   1     0.5   -0.5
%       2   0.6 1 -0.2
%       3  0.45  1  0
%       };
%       \node[black] at (axis cs:1.1,0.8) {$k_1$};
%       \node[black] at (axis cs:1.7,0.35) {$k_2$};
%       \node[black] at (axis cs:2.1,0.5) {$k_1$};
%       \node[black] at (axis cs:2.6,0.4) {$k_2$};
%       \end{axis}
%   \end{tikzpicture}
%   \end{columns}
% \end{frame}

\begin{frame}[t,fragile]
  \frametitle{Classical second order Runge-Kutta (RK2) method}
  \footnotesize\selectfont
   This method is also called Heun's method, or improved Euler method:
  \begin{enumerate}
    \item Approximate the slope at $x_i$: $k_1 = f(x_i,y_i)$
    \item Approximate the slope at $x_{i+1}$: $k_2 = f(x_{i+1},y_{i+1})$ where we use Euler's method to approximate $y_{i+1} = y_i + h f(x_i,y_i) = y_i + h k_1$
    \item Perform an Euler step with the average of the slopes: $y_{i+1} = y_i + h\frac{1}{2}(k_1+k_2)$
  \end{enumerate}
  \pause\vskip1em
  In pseudocode:\\
  \tikz{\node[emphblock, text width=0.5\textwidth] {
    \begin{algorithmic}
    \State $x = x_0$, $y = y_0$
    \While {$x < x_\text{end}$}
        \State $x_{i+1} = x_i + h$
        \State $k_1 = f(x_i,y_i)$
        \State $k_2 = f(x_i+h,y_i+h k_1)$
        \State $y_{i+1} = y_i + h \frac{1}{2}\left( k_1 + k_2 \right)$
    \EndWhile \\
    \end{algorithmic} };}
\end{frame}

\begin{frame}
  \frametitle{Runge-Kutta methods --- derivation}
  \footnotesize\selectfont
  \[ \frac{dy}{dx} = f(x,y(x)) \]
  \pause Using Taylor series expansion: $\displaystyle     y_{i+1} = y_i + h \left.\frac{dy}{dx}\right|_i + \left.\frac{h^2}{2}\frac{d^2y}{dx^2}\right|_i + \mathcal{O}{(h^3)} $
  \begin{align*}
    \left.\frac{dy}{dx}\right|_i &= f(x_i,y_i) \equiv f_i \\
    \left.\frac{d^2y}{dx^2}\right|_i &= \left.\frac{d}{dx}f(x,y(x))\right|_i = \left.\frac{\partial f}{\partial x}\right|_i + \left.\frac{\partial f}{\partial y}\right|_i \left.\frac{\partial y}{\partial x}\right|_i = \left.\frac{\partial f}{\partial x}\right|_i + \left.\frac{\partial f}{\partial y}\right|_i f_i \quad \text{(chain rule)}
  \end{align*}
  \pause Substitution gives:
   \begin{align*}
    y_{i+1} &= y_i + h f_i + \frac{h^2}{2} \left( \left.\frac{\partial f}{\partial x}\right|_i +  \left.\frac{\partial f}{\partial y}\right|_i f_i \right) + \mathcal{O}{(h^3)} \\
    y_{i+1} &= y_i + \frac{h}{2} f_i + \frac{h}{2} \left( f_i + h\left.\frac{\partial f}{\partial x}\right|_i + h f_i \left.\frac{\partial f}{\partial y}\right|_i \right) + \mathcal{O}{(h^3)}
  \end{align*}
\end{frame}

\begin{frame}
  \frametitle{Runge-Kutta methods --- derivation}
%   \footnotesize\selectfont
  Note multivariate Taylor expansion:
  \begin{multline*}
    f(x_i+h,y_i+k) = f_i + h \left. \frac{\partial f}{\partial x}\right|_i + k\left. \frac{\partial f}{\partial y}\right|_i + \mathcal{O}{(h^2)} \\
    \Rightarrow \frac{h}{2}\left(f_i + h\left.\frac{\partial f}{\partial x}\right|_i + h f_i \left. \frac{\partial f}{\partial y} \right|_i \right) = \frac{h}{2} f\left(x_i+h,y_i+kf_i\right) + \mathcal{O}{(h^3)}
  \end{multline*}
  Concluding:
  \[
    y_{i+1} = y_i + \frac{h}{2} f_i + \frac{h}{2} f\left(x_i+h,y_i+kf_i\right) + \mathcal{O}{(h^3)}
  \]
  Rewriting:\\
  \tikz{\node[emphblock, text width=0.5\textwidth] {
  \vspace*{-1.5em}
  \begin{align*}
    k_1 &= f\left(x_i,y_i\right) \\
    k_2 &= f\left(x_i+h,y_i+hk_1\right) \\
    \Rightarrow y_{i+1} &= y_i + \frac{h}{2}(k_1 + k_2)
    \end{align*} };}
\end{frame}

\begin{frame}
  \frametitle{Runge-Kutta methods --- derivation}
  \footnotesize\selectfont
  Generalization: {\color<4>{tuealert}$y_{i+1} = y_i + h(b_1 k_1 + b_2 k_2) + \mathcal{O}{(h^3)}$} \\
  with $k_1 = f_i$, $k_2 = f(x_i + c_2 h, y_1 + a_{2,1}h k_1)$\\
  (Note that classical RK2: $b_1 = b_2 = \frac{1}{2}$ and $c_2 = a_{2,1}=1$.) \vskip1em \pause
  Bivariate Taylor expansion:
  \[
    f(x_i+c_2 h, y_i + a_{2,1}h k_1 ) = f_i + c_2 h \left. \frac{\partial f}{\partial x}\right|_i + a_{2,1}hk_1\left. \frac{\partial f}{\partial y}\right|_i + \mathcal{O}{(h^2)}
  \]
\begin{align*}
    y_{i+1} &= y_i + h(b_1 k_1 + b_2 k_2) + \mathcal{O}{(h^3)} \\
    &= y_i + h\left[b_1 f_i + b_2 f(x_i + c_2 h, y_1 + a_{2,1}h k_1)\right] + \mathcal{O}{(h^3)} \\
    &= y_i + h\left[b_1 f_i + b_2 \left\{ f_i + c_2 h \left.\frac{\partial f}{\partial x}\right|_i + a_{2,1} h k_1 \left.\frac{\partial f}{\partial y}\right|_i + \mathcal{O}{(h^2)} \right\}\right] + \mathcal{O}{(h^3)} \\
    &= {\color<4>{tuealert}y_i + h(b_1 + b_2)f_i + h^2b_2\left( c_2 \left.\frac{\partial f}{\partial x}\right|_i + a_{2,1}f_i \left.\frac{\partial f}{\partial y}\right|_i \right) + \mathcal{O}{(h^3)} }
\end{align*} \pause
Comparison with Taylor: 
\[
  {\color<4>{tuealert}y_{i+1} = y_i + h f_i + \frac{h^2}{2}\left( \left. \frac{\partial f}{\partial x}\right|_i + \left. \frac{\partial f}{\partial y}\right|_i f_i \right) + \mathcal{O}{(h^3)}}
\]
Using $b_1+b_2=1$, $c_2b_2=\frac{1}{2}$, $a_{2,1}b_2=\frac{1}{2} \Rightarrow$ 3 eqns and 4 unknowns $\Rightarrow$ multiple possibilities!

\end{frame}

\begin{frame}
  \frametitle{Runge-Kutta methods --- derivation}
%   \footnotesize\selectfont
  \begin{align*}
%     y_{i+1} &= y_i + h(b_1 k_1 + b_2 k_2) + \mathcal{O}{(h^3)} \\
    y_{i+1} &= y_i + h(b_1 + b_2)f_i + h^2b_2\left( c_2 \left.\frac{\partial f}{\partial x}\right|_i + a_{2,1}f_i \left.\frac{\partial f}{\partial y}\right|_i \right) + \mathcal{O}{(h^3)} \\
    y_{i+1} &= y_i + h f_i + \frac{h^2}{2}\left( \left. \frac{\partial f}{\partial x}\right|_i + \left. \frac{\partial f}{\partial y}\right|_i f_i \right) + \mathcal{O}{(h^3)}
  \end{align*}
  \vfill
  $\Rightarrow$ 3 eqns and 4 unknowns $\Rightarrow$ multiple possibilities!\vskip1em 
  \begin{enumerate}
    \item Classical RK2: \\
    $b_1 = b_2 = \frac{1}{2}$ and $c_2 = a_{2,1}=1$
    \item Midpoint rule (modified Euler): \\
    $ b_1 = 0$, $b_2 = 1$, $c_2 = a_{2,1} = \frac{1}{2}$\\ 
  \end{enumerate}
  \vfill
\end{frame}


\begin{frame}[t,fragile]
  \frametitle{Second order Runge-Kutta methods}
  \footnotesize\selectfont
%     \rowcolors[]{2}{maincolor!20}{maincolor!10}
    \begin{longtable}{c c}
    \hline
    \begin{minipage}{0.4\textwidth}\centering Classical~RK2 method \\(=~Heun's~method, improved~Euler~method)\end{minipage} & \begin{minipage}{0.4\textwidth}\centering Explicit midpoint rule  (modified~Euler~method)\end{minipage} \\ \hline
    $k_1 = f_i$ & $k_1 = f_i$ \\
    $k_2 = f(x_i + h, y_i + h k_1)$ & $k_2 = f(x_i + \frac{1}{2}h, y_i + \frac{1}{2} h k_1)$ \\
    $y_{i+1} = y_i + \frac{1}{2} h (k_1 + k_2)$ & $y_{i+1} = y_i + h k_2$ \\
    \hline
  \end{longtable}
  \begin{columns}
    \column{0.5\textwidth}
    \begin{tikzpicture}
      \begin{axis}[every axis/.append style={font=\footnotesize},
      width=1.2\columnwidth, height=5.5cm,
      xmin=0,xmax=4,ymin=0,ymax=1.1,
      xtick={1,2,3},ytick={1,0.6,0.45},
      axis x line=middle,axis y line=middle,
      xlabel=$x$,xticklabels={$x_0$,$x_1$,$x_2$},
      ylabel=$y(x)$,yticklabels={$y_0$,$y_1$,$y_2$},tick align=outside]

      \draw[line,-,black,densely dotted] (axis cs:1,1) -- (axis cs:2,0.2);
      \draw[line,-,black,densely dotted] (axis cs:2,0.6) -- (axis cs:3,0.2);

    \addplot[graph,quiver={u=\thisrow{u},v=\thisrow{v},scale arrows=0.7},-stealth',mark=*,nodes near coords={\coordindex}]
      table {
      x   y     u    v
      1   1     0.5   -0.4
      2   0.6   1 -0.4
      3  0.45   1  0
      };
      \addplot[graph,quiver={u=\thisrow{u},v=\thisrow{v},scale arrows=0.4},-stealth',mark=*,mark options={solid,fill=white}]
      table {
      x   y     u    v
      2   0.2   1   -0.4
      3   0.2   1.75   0
      };
      \addplot[graph,sharp plot,densely dashed]
      table {
      x   y     u    v
      1   1     0.5   -0.5
      2   0.6   1 -0.2
      3  0.45   1  0
      };
      \node[black] at (axis cs:1.1,0.75) {$k_1$};
      \node[black] at (axis cs:2.4,0.20) {$k_2$};
      \node[black] at (axis cs:1.7,0.9) {$\frac{k_1+k_2}{2}$};
%       \node[black] at (axis cs:2.6,0.4) {$k_2$};
      \end{axis}
  \end{tikzpicture}
  \column{0.5\textwidth}
  \begin{tikzpicture}
      \begin{axis}[every axis/.append style={font=\footnotesize},
      width=1.2\columnwidth, height=5.5cm,
      xmin=0,xmax=4,ymin=0,ymax=1.1,
      xtick={1,1.5,2,2.5,3},ytick={1,0.6,0.45},
      axis x line=middle,axis y line=middle,
      xlabel=$x$,xticklabels={$x_0$,$x_\frac{1}{2}$,$x_1$,$x_\frac{3}{2}$,$x_2$},
      ylabel=$y(x)$,yticklabels={$y_0$,$y_1$,$y_2$},tick align=outside]

    \draw[line,-,black,densely dotted] (axis cs:1,1) -- (axis cs:1.5,0.5);
    \draw[line,-,black,densely dotted] (axis cs:2,0.6) -- (axis cs:2.5,0.5);
      
    \addplot[graph,quiver={u=\thisrow{u},v=\thisrow{v},scale arrows=0.4},-stealth',mark=*,nodes near coords={\coordindex}]
      table {
      x   y     u    v
      1   1     0.5   -0.5
      2   0.6 1 -0.2
      3  0.45  1  0
      };
      \addplot[graph,quiver={u=\thisrow{u},v=\thisrow{v},scale arrows=0.4},-stealth',mark=*,mark options={solid,fill=white}]
      table {
      x   y     u    v
      1.5 0.5 1   -0.4
      2.5   0.5 1 -0.15
      };
      \addplot[graph,sharp plot,densely dashed]
      table {
      x   y     u    v
      1   1     0.5   -0.5
      2   0.6 1 -0.2
      3  0.45  1  0
      };
      \node[black] at (axis cs:1.1,0.8) {$k_1$};
      \node[black] at (axis cs:1.7,0.35) {$k_2$};
      \node[black] at (axis cs:1.7,0.85) {$k_2$};
%       \node[black] at (axis cs:2.6,0.4) {$k_2$};
      \end{axis}
  \end{tikzpicture}
  \end{columns}
\end{frame}


\begin{frame}[t,fragile]
  \frametitle{Second order Runge-Kutta method --- Example}
  First order reaction in a batch reactor:$\frac{dc}{dt} = -kc$ with $c(t=0) = 1$~\si{\mole\per\cubic\meter}, $k = 1$ \si{s^{-1}}, $t_\text{end} = 2$ \si{\second}.
  \scriptsize\selectfont
  \begin{longtable}{p{0.1\textwidth}p{0.2\textwidth}p{0.25\textwidth}p{0.35\textwidth}}
  \hline
    Time [\si{\second}] & C [\si{\mole\per\cubic\meter}] & $k_1 = hf(x_i,y_i)$ & $k_2 = hf(x_i + \frac{1}{2}h,y_n + \frac{1}{2}k_1)$\\ \hline
    $0$   & $1.00$ & $0.1\cdot(-1\cdot1) = -0.1$& $0.1\cdot(-1\cdot(1-0.5\cdot0.1))=-0.095$\\
    $0.1$   & $1-0.095=0.905$ & $0.1\cdot(-1\cdot0.0905) = -0.0905$& $0.1\cdot(-1\cdot(0.905-0.5\cdot0.0905))=−0.085975$\\
   $\quad \ldots$ & $\quad \quad \ldots$ & $\quad \quad \quad \ldots$ & $\quad \quad \quad \ldots$ \\
   $2$ & $0.1358225$ & $-0.0135822$ & $-0.0129031$ \\
    \hline
  \end{longtable}
 \begin{tikzpicture}
      \begin{axis}[every axis/.append style={font=\footnotesize},
      width=\columnwidth, height=4.5cm,
      xmin=0,xmax=4,ymin=0,ymax=1.1,
      xtick={1,1.5,2,2.5,3},ytick={0},
      axis x line=middle,axis y line=left,
      xlabel=$t$,xticklabels={$t_0$,$t_\frac{1}{2}$,$t_1$,$t_\frac{3}{2}$,$t_2$},
      ylabel=$c(t)$,tick align=outside]

    \addplot[graph,quiver={u=\thisrow{u},v=\thisrow{v},scale arrows=0.4},-stealth',mark=*,nodes near coords={$c_\coordindex$}]
      table {
      x   y     u    v
      1   1     0.5   -0.5
      2   0.6 1 -0.2
      3  0.45  1  0
      };
      \addplot[graph,quiver={u=\thisrow{u},v=\thisrow{v},scale arrows=0.4},-stealth',mark=*,mark options={solid,fill=white}]
      table {
      x   y     u    v
      1.5 0.5 1   -0.4
      2.5   0.5 1 -0.15
      };
      \addplot[graph,sharp plot,densely dashed]
      table {
      x   y     u    v
      1   1     0.5   -0.5
      2   0.6 1 -0.2
      3  0.45  1  0
      };
      \node[black] at (axis cs:1.1,0.8) {$k_1$};
      \node[black] at (axis cs:1.7,0.35) {$k_2$};
      \node[black] at (axis cs:2.1,0.5) {$k_1$};
      \node[black] at (axis cs:2.6,0.4) {$k_2$};
      \end{axis}
  \end{tikzpicture}
\end{frame}

\begin{frame}
  \frametitle{RK2 method --- order of convergence}
  \begin{longtable}{cccc}
    \hline
    $N$ & $\zeta$ & $\frac{\zeta^{}_\text{numerical}-\zeta_\text{analytical}}{\zeta_\text{analytical}}$ & $ r = \frac{\log\left(\frac{\epsilon_i}{\epsilon_{i-1}}\right)}{\log \left( \frac{N_{i-1}}{N_i}\right)} $ \\ \hline
    20  & 0.864178 & \num{5.634E-04} & ---\\
    40  & 0.864548 & \num{1.355E-04} & 2.056\\
    80  & 0.864636 & \num{3.323E-05} & 2.028\\
    160 & 0.864658 & \num{8.229E-06} & 2.014\\
    320 & 0.864663 & \num{2.048E-06} & 2.007\\
    \hline
  \end{longtable}
  \pause
  $ \Rightarrow$ RK2 is a second order method. Doubling the number of cells reduces the error by a factor 4! \\
  \vskip1em
  Can we do even better?
\end{frame}

\subsection{RK4 method}
\begin{frame}
  \frametitle{RK4 method (classical fourth order Runge-Kutta method)}
  \vskip1em
  \centering
  \begin{tikzpicture}[scale=0.8,xscale=1.3]
    \draw[line,-,densely dashed] (0,4)  .. node [fdot,very near start,solid] (k1) {} node [fdot,midway,above=9pt,solid,fill=white](k2) {} node [fdot,midway,below=9pt,solid,fill=white](k3) {}controls(2,1) and (6,0) .. node [fdot,pos=0.8,solid] (k4) {} (8,0); 
    \draw[line,ultra thick] ($ (k1) +(0.45,-0.5) $) -- ($ (k1) -(0.45,-0.5) $);
    \node[below=2pt] at (k1) {$y_i$}; 
    \draw[line,ultra thick] ($ (k4) +(0.75,-0.127) $) -- ($ (k4) -(0.75,-0.127) $);
    \node[below=2pt] at (k4) {$y_{i+1}$}; 
    \draw[line,ultra thick] ($ (k2) +(0.6,-0.25) $) -- ($ (k2) -(0.6,-0.25) $);
    \draw[line,ultra thick] ($ (k3) +(0.6,-0.25) $) -- ($ (k3) -(0.6,-0.25) $);
    \node[above=3pt] at (k1) {1};
    \node[above=3pt] at (k2) {2};
    \node[below=3pt] at (k3) {3};
    \node[above=3pt] at (k4) {4};
  \end{tikzpicture} 
  \vspace*{-1cm}

  \begin{align*}
    k_1 &= f(x_i, y_i) \\
    k_2 &= f(x_i+\frac{1}{2}h, y_i + \frac{1}{2}hk_1) \\
    k_3 &= f(x_i + \frac{1}{2}h,y_i + \frac{1}{2}hk_2) \\
    k_4 &= f(x_i + h, y_i + hk_3) \\
    y_{i+1} &= y_i + h\left(\frac{1}{6}k_1 + \frac{1}{3}\left(k_2 + k_3\right) + \frac{1}{6}k_4\right)
  \end{align*}
\end{frame}


\begin{frame}
  \frametitle{RK4 method --- order of convergence}
  \begin{longtable}{cccc}
    \hline
    $N$ & $\zeta$ & $\frac{\zeta^{}_\text{numerical}-\zeta_\text{analytical}}{\zeta_\text{analytical}}$ & $ r = \frac{\log\left(\frac{\epsilon_i}{\epsilon_{i-1}}\right)}{\log \left( \frac{N_{i-1}}{N_i}\right)} $ \\ \hline
    20  & 0.864664472 & \num{2.836E-07} & ---\\
    40  & 0.864664702 & \num{1.700E-08} & 4.060\\
    80  & 0.864664716 & \num{1.040E-09} & 4.030\\
    160 & 0.864664717 & \num{6.435E-11} & 4.015\\
    320 & 0.864664717 & \num{4.001E-12} & 4.007\\
    \hline
  \end{longtable}
  \pause
  $ \Rightarrow$ RK4 is a fourth order method: Doubling the number of cells reduces the error by a factor 16! \\
  \vskip1em
  Can we do even better?
\end{frame}

\section{Step size control}
\againframe<2>{contents}
\subsection*{Adaptive step size}
\begin{frame}
  \frametitle{Adaptive step size control}
  The step size (be it either position, time or both (PDEs)) cannot be decreased indefinitely to favour a higher accuracy, since each additional grid point causes additional computation time. It may be wise to adapt the step size according to the computation requirements. \vskip1em
  Globally two different approaches can be used:
  \begin{enumerate}
    \item Step doubling: compare solutions when taking one full step or two consecutive halve steps
    \item Embedded methods: Compare solutions when using two approximations of different order
  \end{enumerate}
\end{frame}

\begin{frame}
  \frametitle{Adaptive step size control: step doubling}
    \begin{tikzpicture}[scale=5]
      \node[] at (1,0.3){};
      \node[] at (1,-0.3){};
      \node[fdot] (t0)  at (0   , 0) {};
      \node<3->[fdot] (t05) at (0.5 , 0) {};
      \node[fdot] (t1)  at (1   , 0) {};
      \node<3->[fdot] (t15) at (1.5 , 0) {};
      \node[fdot] (t2)  at (2   , 0) {};
      \node[above=13pt]  at (t0) {$t$};
      \node[above=13pt]  at (t1) {$t+\Delta t$};
      \node[above=13pt]  at (t2) {$t+2\Delta t$};
            
      \draw[gridline] (t0) -- (t1) -- (t2);
      \draw<2->[line,draw=tuered] (t0) .. node [midway,above] (dt1) {$\Delta t$} controls (0.25,0.2) and (0.75,0.2) .. (t1);
      \draw<2->[line,draw=tuered] (t1) .. node [midway,above] (dt2) {$\Delta t$} controls (1.25,0.2) and (1.75,0.2) .. (t2);
      
      \draw<3->[line,draw=tuelblue] (t0) .. node [midway,below] (dt05) {$\frac{\Delta t}{2}$} controls (0.125,-0.15) and (0.375,-0.15) .. (t05);
      \draw<3->[line,draw=tuelblue] (t05) .. node [midway,below] (dt15) {$\frac{\Delta t}{2}$} controls (0.625,-0.15) and (0.875,-0.15) .. (t1);
      \draw<3->[line,draw=tuelblue] (t1) .. node [midway,below] (dt05) {$\frac{\Delta t}{2}$} controls (1.125,-0.15) and (1.375,-0.15) .. (t15);
      \draw<3->[line,draw=tuelblue] (t15) .. node [midway,below] (dt15) {$\frac{\Delta t}{2}$} controls (1.625,-0.15) and (1.875,-0.15) .. (t2);
      
      \node<2->[above=3pt,color=tuered]  at (t1) {$y_1$};
      \node<3->[below=3pt,color=tuelblue]  at (t1) {$y_2$};
%       \node[above=3pt,color=tuered]  at (t2) {$y_1$};
%       \node[below=3pt,color=tuelblue]  at (t2) {$y_2$};
            

%       \draw[line] (t1) .. node [midway,above] (dt2) {$\Delta t$} controls (1.25,0.2) and (1.75,0.2) .. (t2);

%       \draw[line,densely dashed] (0.1,0.05) -- node[midway,above] {``Marching''} (0.9,0.05);
    \end{tikzpicture}\begin{itemize}
  \item<2-> \color{tuered} RK4 with one large step of $h$: $ y_{i+1} = y_1 + ch^5 + \mathcal{O}{(h^6)} $
  \item<3-> \color{tuelblue} RK4 with two steps of $\frac{1}{2}h$: $ y_{i+1} = y_2 + 2c(\frac{1}{2}h)^5 + \mathcal{O}{(h^6)} $
\end{itemize}
\end{frame}

\begin{frame}
  \frametitle{Adaptive step size control: step doubling}
\begin{itemize}
  \item Estimation of truncation error by comparing $y_1$ and $y_2$:\\
  $\Delta = y_2 - y_1$
  \item If $\Delta$ too large, reduce step size for accuracy
  \item If $\Delta$ too small, increase step size for efficiency.
  \item Ignoring higher order terms and solving for $c$:
  $ \Delta = \frac{15}{16}ch^5 \Rightarrow ch^5 = \frac{16}{15} \Delta \Rightarrow y_{i+1} = y_2 + \frac{\Delta}{15} + \mathcal{O}{(h^6)}$ \\ (local Richardson extrapolation)
\end{itemize}
  Note that when we specify a tolerance \emph{tol}, we can estimate the maximum allowable step size as:
  $ h_\text{new} = \alpha h_\text{old} \abs{\frac{\text{tol}}{\Delta}}^{\frac{1}{5}}$ with $\alpha$ a safety factor (typically $\alpha = 0.9$).
\end{frame}

\begin{frame}
  \frametitle{Adaptive step size control: embedded methods}
  Use a special fourth and a fifth order Runge Kutta method to approximate $y_{i+1}$
  \begin{itemize}
    \item The fourth order method is special because we want to use the same positions for the evaluation for computational efficiency.
    \item RK45 is the preferred method (minimum number of function evaluations) (this is built in Matlab as \lstinline$ode45$).
  \end{itemize}
\end{frame}

\section{Implicit methods}
\againframe<2>{contents}
\subsection{Backward Euler}
\begin{frame}
  \frametitle{Problems with Euler's method: instability}
  Consider the ODE:
  \[
    \frac{dy}{dx} = f(x,y(x)) \qquad \text{with} \qquad y(x=0) = y_0
  \]
  \vskip1em
  \pause
  First order approximation of derivative: $\frac{dy}{dx} = \frac{y_{i+1}-y_i}{\Delta x}$. 
  \vskip1em
  Where to evaluate the function $f$?
  \vskip1em
  \pause
  \begin{enumerate}
    \item Evaluation at $x_i$:  Explicit Euler method (forward Euler)
    \item Evaluation at $x_{i+1}$: Implicit Euler method (backward Euler)
  \end{enumerate}
\end{frame}

\begin{frame}
  \frametitle{Problems with Euler's method: instability -- forward Euler}
  Explicit Euler method (forward Euler):
    \begin{itemize}
      \item Use values at $x_i$: \\ $\frac{y_{i+1}-y_i}{\Delta x} = f(x_i,y_i) \Rightarrow y_{i+1} = y_i + h f(x_i,y_i)$. 
      \item This is an explicit equation for $y_{i+1}$ in terms of $y_i$.
      \item It can give instabilities with large function values.
    \end{itemize}
    \pause \vskip1em
    Consider the first order batch reactor: 
    \[
      \frac{dc}{dt} = -kc \Rightarrow c_{i+1} = c_i - k{\color{tuered}c_i}\Delta t \Rightarrow \frac{c_{i+1}}{c_i} = 1-k\Delta t
    \]
    \pause
    It follows that unphysical results are obtained for $k\Delta t \geq 1$!! \vskip1em
    \begin{block}{Stability requirement}
      \centering $k \Delta t < 1$\\ (but probably accuracy requirements are more stringent here!)
    \end{block}
\end{frame}
\begin{frame}
  \frametitle{Problems with Euler's method: instability -- backward Euler}
  Implicit Euler method (backward Euler):
    \begin{itemize}
      \item Use values at $x_{i+1}$: $\frac{y_{i+1}-y_i}{\Delta x} = f(x_{i+1},y_{i+1}) \Rightarrow y_{i+1} = y_i + h f(x_{i+1},y_{i+1})$. 
      \item This is an implicit equation for $y_{i+1}$, because it also depends on terms of $y_{i+1}$.
    \end{itemize}
    \pause \vskip1em
    Consider the first order batch reactor: 
    \[
      \frac{dc}{dt} = -kc \Rightarrow c_{i+1} = c_i - k{\color{tuered}c_{i+1}}\Delta t \Rightarrow \frac{c_{i+1}}{c_i} = \frac{1}{1+k\Delta t}
    \]
    \pause
    This equation does never give unphysical results!\\
    The implicit Euler method is \emph{unconditionally stable} \\
    (but maybe not very accurate or efficient).
\end{frame}


\begin{frame}
  \frametitle{Semi-implicit Euler method}
  \footnotesize\selectfont
  Usually $f$ is a non-linear function of $y$, so that linearization is required (recall Newton's method).
  
  \begin{align*}
    \frac{dy}{dx} &= f(y) \Rightarrow y_{i+1} = y_i + h f (y_{i+1}) \quad \text{using} \quad f(y_{i+1}) = f(y_i) + \left.\frac{df}{dy}\right|_i(y_{i+1}-y_i) + \ldots \\
  &\Rightarrow y_{i+1} =  y_i + h \left[ f(y_i) + \left.\frac{df}{dy}\right|_i (y_{i+1}-y_i) \right] \\ 
  &\Rightarrow \left(1-h\left.\frac{df}{dy}\right|_i \right)y_{i+1} = \left(1-h\left.\frac{df}{dy}\right|_i\right)y_i + h f(y_i) 
  \end{align*}
  
  \tikz{\node[emphblock,text width=\textwidth] {$\displaystyle \Rightarrow y_{i+1} = y_i + h \left( 1 - h \left.\frac{df}{dy}\right|_i \right)^{-1} f(y_i) $};}
  \pause
  For the case that $f(x,y(x))$ we could add the variable $x$ as an additional variable $y_{n+1}=x$. Or add one fully implicit Euler step (which avoids the computation of $\frac{\partial f}{\partial x}$): \vspace*{-1em}
  \[
    y_{i+1} = y_i + h f(x_{i+1},y_{i+1}) \Rightarrow y_{i+1} = y_i + h \left(1-h\left.\frac{df}{dy}\right|_i \right)^{-1} f(x_{i+1},y_i)
  \]
\end{frame}

\begin{frame}
  \frametitle{Semi-implicit Euler method - example}
  Second order reaction in a batch reactor:\\
  $\frac{dc}{dt} = -kc^2$ with $c_0 = 1$~\si{\mole\per\cubic\meter}, $k = 1$ \si{\cubic\meter\per\mole\per\second}, $t_\text{end} = 2$ \si{\second} \\
  Analytical solution: $c(t) = \frac{c_0}{1+kc_0t}$
  \vskip1em \pause  
  Define $f = -kc^2$, then $\frac{df}{dc} = -2kc \Rightarrow c_{i+1} = c_i - \frac{hkc_i^2}{1+2hkc_i}$.
  \pause
    \begin{longtable}{cccc}
    \hline
    $N$ & $\zeta$ & $\frac{\zeta^{}_\text{numerical}-\zeta_\text{analytical}}{\zeta_\text{analytical}}$ & $ r = \frac{\log\left(\frac{\epsilon_i}{\epsilon_{i-1}}\right)}{\log \left( \frac{N_{i-1}}{N_i}\right)} $ \\ \hline
    20  & 0.654066262 & \num{1.89E-002} & ---\\
    40  & 0.660462687 & \num{9.31E-003} & 1.02220\\
    80  & 0.663589561 & \num{4.62E-003} & 1.01162\\
    160 & 0.665134433 & \num{2.30E-003} & 1.00594\\
    320 & 0.665902142 & \num{1.15E-003} & 1.00300\\
    \hline
  \end{longtable}
\end{frame}

\subsection{Implicit midpoint method}
\begin{frame}
  \frametitle{Second order implicit method: Implicit midpoint method}
  \footnotesize\selectfont
\begin{longtable}{c c}
    \hline
    \begin{minipage}{0.4\textwidth}\centering Implicit midpoint rule \\(second order)\end{minipage} & \begin{minipage}{0.4\textwidth}\centering Explicit midpoint rule  (modified~Euler~method)\end{minipage} \\ \hline
    $y_{i+1} = y_i + hf\left(x_i + \frac{1}{2}h, {\color{tuered}\frac{1}{2}(y_i + y_{i+1})}\right)$  & $y_{i+1} = y_i + hf(x_i + \frac{1}{2}h, {\color{tuered}y_i + \frac{1}{2} h k_1})$ \\
    \hline
  \end{longtable}
  in case $f(y)$ then:
  \[
    f\left(\frac{1}{2}(y_i+y_{i+1})\right) = f_i + \left.\frac{df}{dy}\right|_i \left( \frac{1}{2}(y_i + y_{i+1})-y_i\right) = f_i + \frac{1}{2}\left.\frac{df}{dy}\right|_i(y_{i+1}-y_i)
  \]
  \pause
  Implicit midpoint rule reduces to: 
  \begin{align*}
    y_{i+1} &= y_i + h f_i + \frac{h}{2}\left.\frac{df}{dy}\right|_i(y_{i+1}-y_i)\\
    &\Rightarrow \left(1 - \frac{h}{2} \left.\frac{df}{dy}\right|_i\right)y_{i+1} = \left(1 - \frac{h}{2} \left.\frac{df}{dy}\right|_i\right)y_i + h f_i
  \end{align*}
  \tikz{\node[emphblock,text width=0.7\textwidth] {
    $ \displaystyle \Rightarrow y_{i+1} = y_i + h \left( 1 - \frac{h}{2} \left.\frac{df}{dy}\right|_i\right)^{-1} f_i$
    };}
\end{frame}

\begin{frame}
  \frametitle{Implicit midpoint method --- example}
  Second order reaction in a batch reactor: \\
  $\frac{dc}{dt} = -kc^2$ with $c_0 = 1$~\si{\mole\per\cubic\meter}, $k = 1$ \si{\cubic\meter\per\mole\per\second}, $t_\text{end} = 2$ \si{\second} (Analytical solution: $c(t) = \frac{c_0}{1+kc_0t}$). 
  \vskip1em \pause  
  Define $f = -kc^2$, then $\frac{df}{dc} = -2kc$.\vskip1em \pause
  Substitution: 
  \begin{align*}
    c_{i+1} &= c_i + h \left(1-\frac{h}{2} \cdot (-2kc_i)\right)^{-1} \cdot(-kc^2_i) \\
	    &= c_i - \frac{hkc_i^2}{1+hkc_i} = \frac{c_i + h k c_i^2 - h k c_i^2 }{1+hkc_i} \Rightarrow c_{i+1} = \frac{c_i}{1+hkc_i}
  \end{align*}
  
  \pause
  You will find that this method is exact for all step sizes $h$ because of the quadratic source term!
\end{frame}

\begin{frame}
  \frametitle{Implicit midpoint method --- example}
  {\color{tuealert}Second order} reaction in a batch reactor:\\
  $\frac{dc}{dt} = -kc^2$ with $c_0 = 1$~\si{\mole\per\cubic\meter}, $k = 1$ \si{\cubic\meter\per\mole\per\second}, $t_\text{end} = 2$ \si{\second}\\
  Analytical solution: $c(t) = \frac{c_0}{1+kc_0t}$
  \[
    c_{i+1} = \frac{c_i}{1+hkc_i}
  \]
  \pause  
  \begin{longtable}{cccc}
    \hline
    $N$ & $\zeta$ & $\frac{\zeta^{}_\text{numerical}-\zeta_\text{analytical}}{\zeta_\text{analytical}}$ & $ r = \frac{\log\left(\frac{\epsilon_i}{\epsilon_{i-1}}\right)}{\log \left( \frac{N_{i-1}}{N_i}\right)} $ \\ \hline
    20  & 0.6666666667 & \num{1.665E-016} & ---\\
    40  & 0.6666666667 & \num{0} & ---\\
    80  & 0.6666666667 & \num{0} & ---\\
    160 & 0.6666666667 & \num{0} & ---\\
    320 & 0.6666666667 & \num{0} & ---\\
    \hline
  \end{longtable}
\end{frame}

\begin{frame}
  \frametitle{Implicit midpoint method --- example}
  {\color{tuealert}Third order} reaction in a batch reactor\\
  Analytical solution: $c(t) = \frac{c_0}{\sqrt{1+2kc_0^2t}}$
  \[
    c_{i+1} = c_i - \frac{hkc_i^3}{1+\frac{3}{2}hkc_i^2}
  \]
  \pause  
  \begin{longtable}{cccc}
    \hline
    $N$ & $\zeta$ & $\frac{\zeta^{}_\text{numerical}-\zeta_\text{analytical}}{\zeta_\text{analytical}}$ & $ r = \frac{\log\left(\frac{\epsilon_i}{\epsilon_{i-1}}\right)}{\log \left( \frac{N_{i-1}}{N_i}\right)} $ \\ \hline
    20  & 0.5526916174 & \num{1.71E-004} & ---\\
    40  & 0.5527633731 & \num{4.17E-005} & 2.041\\
    80  & 0.5527807304 & \num{1.03E-005} & 2.021\\
    160 & 0.5527849965 & \num{2.55E-006} & 2.011\\
    320 & 0.5527860538 & \num{6.34E-007} & 2.005\\
    \hline
  \end{longtable}
\end{frame}

\part{Boudary Value Problems and Systems of ODEs}
\frame{\partpage}

\section{Boundary value problems}
\againframe<2>{contents}
\subsection{Shooting method}
\againframe{ivpbvp}
\begin{frame}
  \frametitle{Shooting method}
  How to solve a BVP using the shooting method:\\ \vskip1em
  \centering
    \begin{tikzpicture}[scale=5]
      \node[] (y1s) at (0,0.1) {$y_{1,s}$};
\node[] (y1f) at (1,0.1) {\color{scharlaken}$y_{1,f}$};
\node[] (y2s) at (0,0) {\color{scharlaken}$y_{2,s}$};
\node[] (y2f) at (1,0) {$y_{2,f}$};
\node[fdot] (xs) at (0,-0.1) {};
\node[fdot] (xf) at (1,-0.1) {};
\node[anchor=north] at (xs.south) {$x_s$};
\node[anchor=north] at (xf.south) {$x_f$};
\draw[line] (xs) -- (xf);
\draw[line,->,densely dashed] (0.1,0.05) -- node[midway,above] {``Shooting''} (0.9,0.05);

\coordinate[] (b) at ($(y1f)!0.5!(y2f) + (0.1,0) $) {};
\coordinate[right of=b] (c1) {};
\coordinate[below=1.4cm] (c2) at (c1)  {};
%       \coordinate[below of=1cm] at (c2) (c5) {};

\coordinate[] (s) at ($(y1s)!0.5!(y2s) - (0.1,0) $) {};
\coordinate[left of=s] (c3) {};
\coordinate[below=1.4cm] (c4) at (c3)  {};
%       \coordinate[below of=c4] (c6) {};

%       \node[] (b) at ($(y1f)!0.5!(y2f)$) {};
\draw[line,->,densely dashed,draw=tuegreen,rounded corners=10pt] (b) -- (c1) -- (c2) -- (c4) -- (c3) -- (s);
%       \draw[line,densely dashed,draw=tuegreen] (y2f) to [controls=(1,-0.2) and +(-0.1,-0.2)] (y2s);
    \end{tikzpicture}
  \begin{itemize}
    \item Define the system of ODEs
    \item Provide an initial guess for the unknown boundary condition
    \item Solve the system and compare the resulting boundary condition to the expected value
    \item Adjust the guessed boundary value, and solve again. Repeat until convergence.
    \begin{itemize}
      \item Of course, you can subtract the expected value from the computed value at the boundary, and use a non-linear root finding method
    \end{itemize}

  \end{itemize}
\end{frame}

\begin{frame}[t]
  \frametitle{BVP: example in Excel}
  \footnotesize\selectfont
  \setlength{\mathindent}{12pt}
  Consider a chemical reaction in a liquid film layer of thickness $\delta$:\\
  $\displaystyle \mathcal{D}\frac{d^2c}{dx^2} = k_Rc $ with \begin{minipage}{0.7\textwidth}
    \begin{align*}
      c(x=0) &= C_{A,i,L} = 1 &\quad \text{(interface concentration)} \\
      c(x=\delta) &= 0 &\quad \text{(bulk concentration)}\\
    \end{align*}
  \end{minipage}
  Question: compute the concentration profile in the film layer.
  \pause
  \only<2>{
    \begin{block}{Step 1: Define the system of ODEs}
      This second-order ODE can be rewritten as a system of first-order ODEs, if we define the flux $q$ as:
      \[
  q = -\mathcal{D}\frac{dc}{dx}
      \]
      Now, we find:
      \begin{align*}
  \frac{dc}{dx} &= -\frac{1}{\mathcal{D}}q\\
  \frac{dq}{dx} &= -k_Rc
      \end{align*}
    \end{block}
  }
  \only<3>{
    \vskip1em
  \begin{columns}
    \column{0.5\textwidth}
    \begin{block}{Step 2: Set the boundary conditions}
      The boundary conditions for the concentrations at $x=0$ and $x=\delta$ are known.\\ \vskip1em The flux at the interface, however, is not known, and should be solved for.
    \end{block}
    \column{0.5\textwidth}
    \centering\tikz\node [emphblock,cloud, draw,cloud puffs=11,minimum width=\columnwidth] 
    {
      \begin{minipage}{2cm}
        $\displaystyle \frac{dc}{dx} = -\frac{1}{\mathcal{D}}q$ \\ \vskip1ex
        $\displaystyle \frac{dq}{dx} = -k_Rc$
      \end{minipage}
    };
  \end{columns}
  }
\end{frame}

\begin{frame}
  \frametitle{BVP: example in Excel}  
  \scriptsize\selectfont
  \rowcolors[]{80}{white}{white}\renewcommand\arraystretch{1.1}
  
  Solving the two first-order ODEs in Excel. First, the cells with constants:
  \begin{columns}
    \column{0.5\textwidth}
    \begin{longtable}{|>{\columncolor{gray!40}}R{0.3cm}*{1}{|L{0.8cm}}*{1}{|L{1.2cm}}*{1}{|L{1.3cm}}|}
      \hline
      \rowcolor{gray!40}& \centering A  & \centering B& \centering C \tabularnewline
      \hline
      1 & CAiL& 1 & mol/m3  \\
      \hline
      2 & D \hfill & 1e--8 & m2/s \\
      \hline
      3 & kR \hfill& 10& 1/s \\
      \hline
      4 & delta\hfill & 1e--4 & m \\
      \hline
      5 & N \hfill& 100& \\
      \hline
      6 & dx \hfill& =B4/B5 &  \\
      \hline
    \end{longtable}
    \column{0.5\textwidth}
    \centering\tikz\node [emphblock,cloud, draw,cloud puffs=11,minimum width=\columnwidth] 
    {
      \begin{minipage}{2cm}
        $\displaystyle \frac{dc}{dx} = -\frac{1}{\mathcal{D}}q$ \\ \vskip1ex
        $\displaystyle \frac{dq}{dx} = -k_Rc$
      \end{minipage}
    };
  \end{columns}
  \pause \vskip1em
  Now, we program the forward Euler (explicit) schemes for $c$ and $q$ below: \\ \setlength{\LTleft}{-0.5cm}
  \begin{longtable}{|>{\columncolor{gray!40}}R{0.3cm}*{1}{|L{1.8cm}}*{1}{|L{4.05cm}}*{1}{|L{3.7cm}}|}
    \hline
    \rowcolor{gray!40}& \centering A  & \centering B& \centering C \tabularnewline
    \hline
    10 & x \hfill & c\hfill & q  \hfill\\
    \hline
    11 & 0     & =B1 & \color{tuered} 10 \\
    \hline
    12 & =A11+\$B\$6  & =B11+\$B\$6*(--1/\$B\$2*C11) & =C11+\$B\$6*(-\$B\$3*B11)\\
    \hline
    13 & =A12+\$B\$6  & =B12+\$B\$6*(--1/\$B\$2*C12) & =C12+\$B\$6*(-\$B\$3*B12)\\
    \hline
    $\dots$ & $\dots$ & $\dots$& $\dots$\\
    \hline
    111 & =A110+\$B\$6 & =B110+\$B\$6*(--1/\$B\$2*C110) & =C110+\$B\$6*(-\$B\$3*B110) \\
    \hline
  \end{longtable}
\end{frame}

\begin{frame}
  \frametitle{BVP: example in Excel}  
  \begin{itemize}
    \item We now have profiles for $c$ and $q$ as a function of position $x$.
    \item The concentration $c(x=\delta)$ depends (eventually) on the boundary condition at the interface $q(x=0)$
    \item We can use the solver to change $q(x=0)$ such that the concentration at the bulk meets our requirement: $c(x=\delta)=0$
  \end{itemize}

\end{frame}

\begin{frame}[fragile]
  \frametitle{BVP: example in Matlab}
  We first program the system of ODEs in a separate function:
  \begin{align*}
    \frac{dc}{dx} &= -\frac{1}{\mathcal{D}}q\\
    \frac{dq}{dx} &= -k_Rc
  \end{align*}
  \begin{lstlisting}
function [dxdt] = BVPODE(t,x,ps)
dxdt(1)=-1/ps.D*x(2);
dxdt(2)=-ps.kR*x(1);
dxdt=dxdt';
return
  \end{lstlisting}
  \pause\vskip1em
  Note that we pass a variable (type: struct) that contains required parameters: \lstinline$ps$.
\end{frame}

\begin{frame}[fragile]
  \frametitle{BVP: example in Matlab}
  The ODE function is solved via ode45, after setting a number of initial and boundary conditions:
  \begin{lstlisting}
function f = RunBVP(bcq,ps)
[x,cq] = ode45(@BVPODE,[0 ps.delta],[1 bcq], [], ps);
f = cq(end,1) - 0;
plotyy(x,cq(:,1),x,cq(:,2));
return;
  \end{lstlisting}
  \pause\vskip1em
  Note the following:
  \begin{itemize}
    \item We use the interval $0\leq x \leq \delta$
    \item Boundary conditions are given as: $c(x=0)=1$ and $q(x=0)=$ \lstinline$bcq$, which is given as an argument to the function (i.e. changable from 'outside'!)
    \item The function returns \lstinline$f$, the difference between the computed and desired concentration at $x=\delta$.
  \end{itemize}
\end{frame}

\begin{frame}[fragile]
  \frametitle{BVP: example in Matlab}
  Finally, we should solve the system so that we obtain the right boundary condition $q=$ \lstinline$bcq$ such that $c(x=\delta)=0$. We can use the built-in function \lstinline$fzero$ to do this
  \begin{lstlisting}
% Parameter definition
ps.D=1e-8;
ps.kR=10;
ps.delta=1e-4;

% Solve for flux boundary condition (initial guess: 0)
opt = optimset('Display','iter');
flux = fzero(@RunBVP,0,opt,ps);
  \end{lstlisting}
%   \vskip1em
%   \tikz \node[emphblock,text width=\textwidth] {Compare the analytical solution: \\
%   $\displaystyle q = k_L E_A C_{A,i,L} \quad$ with \begin{minipage}{0.7\textwidth}
%       \begin{align*}
%   E_A &= \frac{\mathit{Ha}}{\tanh \mathit{Ha}} &\quad \text{(Activation energy)} \\
%   \mathit{Ha} &= \frac{\sqrt{k_R\mathcal{D}}}{k_L} &\quad \text{(Hatta number)} \\
%   k_L &= \frac{\mathcal{D}}{\delta} &\quad \text{(mass transfer coefficient)}
%       \end{align*}
%     \end{minipage}
% };
\end{frame}

\begin{frame}[fragile]
  \frametitle{BVP example: analytical solution}
  Compare with the analytical solution: \\ \vskip1em
    \tikz \node[emphblock,text width=\textwidth] {
  $\displaystyle q = k_L E_A C_{A,i,L} \quad$ with \\ 
  \begin{minipage}{0.6\textwidth}
      \begin{align*}
  E_A &= \frac{\mathit{Ha}}{\tanh \mathit{Ha}} &\quad \text{(Enhancement factor)} \\
  \mathit{Ha} &= \frac{\sqrt{k_R\mathcal{D}}}{k_L} &\quad \text{(Hatta number)} \\ 
  k_L &= \frac{\mathcal{D}}{\delta} &\quad \text{(mass transfer coefficient)}
      \end{align*}
    \end{minipage}
};
\end{frame}

\section{Systems of ODEs}
\againframe<2>{contents}
\subsection{Solution methods for systems of ODEs}
\begin{frame}
  \frametitle{Systems of ODEs}
  A system of ODEs is specified using vector notation:
  \[
    \frac{d\vec{y}}{dx} = \vec{f}(x,\vec{y}(x))
  \]
  for
  \begin{multline*}
    \frac{dy_1}{dx} = f_1(x,y_1(x),y_2(x)) \quad \text{or} \quad f_1(x,y_1,y_2)\\
    \frac{dy_2}{dx} = f_2(x,y_1(x),y_2(x)) \quad \text{or} \quad f_2(x,y_1,y_2)\\
  \end{multline*}
  \pause
  \tikz{\node[emphblock,text width=\textwidth] {The solution techniques discussed before can also be used to solve systems of equations.};}
\end{frame}

\begin{frame}
  \frametitle{Systems of ODEs: Explicit methods}
  \begin{block}{Forward Euler method}
    $ \displaystyle  \vec{y}_{i+1} = \vec{y}_i + h \vec{f}(x_i,\vec{y}_i) $
  \end{block}
  \begin{block}{Improved Euler method (classical RK2)}
    $\displaystyle \vec{y}_{i+1} = \vec{y}_i + \frac{h}{2}(\vec{k}_1+\vec{k}_2)$
    \quad using \quad \begin{minipage}{0.4\textwidth}
      $\displaystyle \vec{k}_1 = \vec{f}(x_i,\vec{y}_i)$\\
      $\displaystyle \vec{k}_2 = \vec{f}(x_i+h,\vec{y}_i+h\vec{k}_1)$
    \end{minipage}
  \end{block}  
  \begin{block}{Modified Euler method (midpoint rule)}
    $\displaystyle \vec{y}_{i+1} = \vec{y}_i + h\vec{k}_2$
    \quad using \quad \begin{minipage}{0.4\textwidth}
      $\displaystyle \vec{k}_1 = \vec{f}(x_i,\vec{y}_i)$\\
      $\displaystyle \vec{k}_2 = \vec{f}(x_i+\frac{h}{2},\vec{y}_i+\frac{h}{2}\vec{k}_1)$
    \end{minipage}
  \end{block}   
\end{frame}

\begin{frame}
  \frametitle{Systems of ODEs: Explicit methods}
  \begin{block}{Classical fourth order Runge-Kutta method (RK4)}
    $\displaystyle \vec{y}_{i+1} = \vec{y}_i + h\left(\frac{\vec{k}_1}{6}+\frac{1}{3}\left( \vec{k}_2 + \vec{k}_3\right) + \frac{\vec{k}_4}{6} \right)$
    \\ \vskip2em
\quad using \quad \begin{minipage}{0.4\textwidth}
      $\displaystyle \vec{k}_1 = \vec{f}(x_i,\vec{y}_i)$\\ \vskip1ex
      $\displaystyle \vec{k}_2 = \vec{f}(x_i+\frac{h}{2},\vec{y}_i+\frac{h}{2}\vec{k}_1)$ \\ \vskip1ex
      $\displaystyle \vec{k}_3 = \vec{f}(x_i+\frac{h}{2},\vec{y}_i+\frac{h}{2}\vec{k}_2)$ \\ \vskip1ex
      $\displaystyle \vec{k}_4 = \vec{f}(x_i+h,\vec{y}_i+h\vec{k}_3)$ \\ \vskip1ex
    \end{minipage}
%     \end{center}
  \end{block}  
\end{frame}

\begin{frame}
  \frametitle{Systems of ODEs: Implicit methods}
  \begin{block}{Backward Euler method}
    $ \displaystyle  \vec{y}_{i+1} = \vec{y}_i + h \left(1 - h\left. \frac{d\vec{f}}{d\vec{y}}\right|_i \right)^{-1}\vec{f}(\vec{y}_i)$
  \end{block}
  \begin{block}{Implicit midpoint method}
    $ \displaystyle  \vec{y}_{i+1} = \vec{y}_i + h \left(1 - \frac{h}{2}\left. \frac{d\vec{f}}{d\vec{y}}\right|_i \right)^{-1} \vec{f}(\vec{y}_i)$
  \end{block}  
\end{frame}

\subsection{Stiff systems of ODEs}
\begin{frame}
  \frametitle{Stiff systems of ODEs}
  A system of ODEs can be stiff and require a different solution method. \pause
  For example:
  \[
    \frac{dc_1}{dt} = 998c_1 + 1998c_2 \qquad 
    \frac{dc_2}{dt} = -999c_1 -1999c_2
  \]
  with boundary conditions $c_1(t=0)=1$ and $c_2(t=0)=0$. \\
  The analytical solution is: 
  \[
    c_1 = 2e^{-t}-e^{-1000t} \qquad
    c_2 =-e^{-t}+e^{-1000t}
  \]
  For the explicit method we require $\Delta t<10^{-3}$ despite the fact that the term is completely negligible, but essential to keep stability. \pause
  \tikz{\node[emphblock,text width=\textwidth] {The ``disease'' of stiff equations: we need to follow the solution on the shortest length scale to maintain stability of the integration, although accuracy requirements would allow a much larger time step.};}
\end{frame}

\begin{frame}
  \frametitle{Demonstration with example}
  Forward Euler (explicit) 
%   \vskip2em
  \begin{align*}
 \frac{c_{1,i+1} - c_{1,i}}{dt} &= 998c_{1,i}+1998c_{2,i}\\
  \frac{c_{2,i+1} - c_{2,i}}{dt} &= -999c_{1,i}-1999c_{2,i}
  \end{align*}
    \qquad $\Rightarrow $ 
    \begin{minipage}{0.7\textwidth}
      $c_{1,i+1} = \left(1+998\Delta t\right)c_{1,i} + 1998\Delta t c_{2,i}$\\
      $c_{2,i+1} = -999\Delta t c_{1,i} + \left( 1- 1999\Delta t\right) c_{2,i}$
  \end{minipage}\vskip1em
  \phantom{
  $A \vec{c}_{i+1} = \vec{c}_i$ with 
  $A = \begin{pmatrix}
    1-998\Delta t & -1998\Delta t \\
    999\Delta t & 1+1999\Delta t
  \end{pmatrix}$ 
  and $\vec{b} = \begin{pmatrix}
             c_{1,i}\\
             c_{2,i}\\
           \end{pmatrix}$}
\end{frame}

\begin{frame}
  \frametitle{Demonstration with example}
 \rowcolors[]{20}{white}{white}
  Backward Euler (implicit)
%   \vskip1em
  \begin{align*}
 \frac{dc_{1,i+1} - c_{1,i+1}}{dt} &= 998c_{1,i+1}+1998c_{2,i+1}\\
  \frac{dc_{2,i+1} - c_{2,i+1}}{dt} &= -999c_{1,i+1}-1999c_{2,i+1}
  \end{align*}
    \qquad $\Rightarrow $ 
    \begin{minipage}{0.7\textwidth}
      $\left(1-998\Delta t\right)c_{1,i+1} - 1998\Delta t c_{2,i} = c_{1,i}$\\
      $999\Delta t c_{1,i+1}+\left(1+999\Delta t\right)c_{2,i+1} = c_{2,i}$
  \end{minipage} \vskip1em \pause
  $A \vec{c}_{i+1} = \vec{c}_i$ with 
  $A = \begin{pmatrix}
    1-998\Delta t & -1998\Delta t \\
    999\Delta t & 1+1999\Delta t
  \end{pmatrix}$ 
  and $\vec{b} = \begin{pmatrix}
             c_{1,i}\\
             c_{2,i}\\
           \end{pmatrix}$
\end{frame}

\begin{frame}
  \frametitle{Demonstration with example}
 \rowcolors[]{20}{white}{white}
  Backward Euler (implicit)
%   \vskip1em
  $A \vec{c}_{i+1} = \vec{c}_i$ with 
  $A = \begin{pmatrix}
    1-998\Delta t & -1998\Delta t \\
    999\Delta t & 1+1999\Delta t
  \end{pmatrix}$ 
  \quad and \quad $\vec{b} = \begin{pmatrix}
             c_{1,i}\\
             c_{2,i}\\
           \end{pmatrix}$
\vskip1em \pause
Cramers rule:\\
$c_{1,i+1} = \dfrac{\begin{vmatrix}
                                   c_{1,i} & -1998\Delta t \\
                                   c_{2,i} & 1+1999\Delta t
                                 \end{vmatrix}
}{\det{A}} = \frac{\left(1+1999\Delta t\right)c_{1,i}+1998 \Delta t c_{2,i}}{\left( 1-998\Delta t\right)\left( 1+1999\Delta t\right) + 1998\cdot 999 \Delta t^2}$\\

$c_{2,i+1} = \dfrac{\begin{vmatrix}
                                   1-998\Delta t & c_{1,i}\\
                                   999\Delta t & c_{2,i}
                                 \end{vmatrix}
}{\det{A}} = \frac{-999\Delta t c_{1,i} + \left(1-998\Delta t\right)c_{2,i}}{\left(1-998\Delta t\right)\left( 1+1999\Delta t\right) + 1998\cdot 999 \Delta t^2}$\\ \vskip1em 
% \tikz{\node[emphblock,text width=\textwidth] {
Forward Euler: $\Delta t \leq 0.001$ for stability\\
Backward Euler: always stable, even for $\Delta t > 100$ (but then not very accurate!)
% };}
\end{frame}

\begin{frame}
  \frametitle{Demonstration with example}
  \tikz{\node[emphblock,text width=\textwidth] {
  Cure for stiff problems: use implicit methods!
To find out whether your system is stiff: check whether one of the eigenvalues have an imaginary part};}
\end{frame}

\section{Solving systems of ODEs in Matlab}
\subsection*{Solving ODEs in Matlab}
\begin{frame}
  \frametitle{Solving systems of ODEs in Matlab}
  Matlab provides convenient procedures to solve (systems of) ODEs automatically.
  \vskip1em
  The procedure is as follows:
  \begin{enumerate}
    \item Create a function that specifies the ODEs. Specifically, this function returns the $\frac{d\vec{y}}{dx}$ vector.
    \item Initialise solver variables and settings (e.g. step size, initial conditions, tolerance), in a separate script
    \item Call the ODE solver function, using a \emph{function handle} to the ODE function described in point 1.
    \begin{itemize}
      \item The ODE solver will return the vector for the independent variable, and a solution vector (matrix for systems of ODEs).
    \end{itemize}
  \end{enumerate}
\end{frame}

\begin{frame}[fragile]
  \frametitle{Solving systems of ODEs in Matlab: example}
  We solve the system: $\displaystyle \frac{dx_1}{dt} = -x_1 - x_2, \quad  \frac{dx_2}{dt} = x_1 - 2x_2 $
  \begin{block}{Create an ODE function}
    \begin{lstlisting}[backgroundcolor=]
function [dxdt] = myODEFunction(t,x)
dxdt(1) = -x(1) - x(2);
dxdt(2) =  x(1) - 2*x(2);
dxdt=dxdt'; % Transpose to column vector
return
    \end{lstlisting}
  \end{block}
  \pause
  \begin{block}{Create a solution script}
    \begin{lstlisting}[backgroundcolor=]
x_init = [0 1];         % Initial conditions
tspan = [0 10];         % Time span
options = odeset('RelTol',1e-4,'AbsTol',[1e-4 1e-4]);
[t,x] = ode45(@myODEfunction,tspan,x_init,options);
    \end{lstlisting}
  \end{block}
\vfill
\end{frame}

\begin{frame}[fragile]
  \frametitle{Solving systems of ODEs in Matlab: example}
  Plot the solution:
  \begin{lstlisting}
plot(t,x(:,1),'r-x',t,x(:,2),'b-o')
  \end{lstlisting}
  \pause
  \begin{center}
    \begin{tikzpicture}
      \begin{axis}[%every axis/.append style={font=\footnotesize},
        width=\textwidth, height=7cm,     % size of the image
        grid = major,grid style={dashed, gray!30},
        axis background/.style={fill=white},
        axis x line=middle,axis y line=middle,ylabel=$x$,xlabel=$t$]

        \addplot[graph,mark=x] table []{ODE_matlabsolve1.dat};  \addlegendentry{$x_1$}
        \addplot[graph,mark=o,draw=tueblue,mark options={fill=white}] table []{ODE_matlabsolve2.dat};  \addlegendentry{$x_2$}
      \end{axis}
    \end{tikzpicture}
  \end{center}
\end{frame}

\begin{frame}[fragile]
  \frametitle{Solving systems of ODEs in Matlab: example}
  A few notes on working with \lstinline$ode45$ and other solvers. If we want to give additional arguments (e.g. \lstinline$a$, \lstinline$b$ and \lstinline$c$) to our ODE function, we can list them in the function line:
  \begin{lstlisting}
function [dxdt] = myODE(t,x,a,b,c)
  \end{lstlisting}
  The additional arguments can now be set in the solver script by \emph{adding them after the options}:
  \begin{lstlisting}
[t,x] = ode45(@myODE,tspan,x_0,options,a,b,c);
    \end{lstlisting}
  \pause
  \begin{itemize}
    \item Of course, in the solver script, the variables do not need to be called \lstinline$a$, \lstinline$b$ and \lstinline$c$:
      \begin{lstlisting}
[t,x] = ode45(@myODE,tspan,x_0,options,k1,phi,V);
    \end{lstlisting}
    \item These variables may be of any type (vectors, matrix, struct). Especially a struct is useful to carry many values in 1 variable.
  \end{itemize}
\end{frame}

\begin{frame}[fragile]
  \frametitle{Solving systems of ODEs in Matlab: example}
  You may have noticed that the step size in $t$ varied. This is because we have given the begin and end times of our time span:
  \begin{lstlisting}
tspan = [0 10];
  \end{lstlisting}
  \pause
  You can also solve at specific steps, by supplying all steps explicitly, e.g.:
  \begin{lstlisting}
tspan = linspace(0,10,101); 
    \end{lstlisting}
  This example provides 101 explicit time steps between 0 and 10 seconds.
  \vskip1em
  Note that you may affect the efficiency and accuracy of the solver algorithm by doing this!
  \vfill
\end{frame}

\section{Conclusion}
\againframe<2>{contents}
\begin{frame}
  \frametitle{Other methods}
  Other explicit methods:
  \begin{itemize}
    \item Burlisch-Stoer method (Richardson extrapolation + modified midpoint method)
  \end{itemize}
  \vskip1em
  Other implicit methods:
  \begin{itemize}
    \item Rosenbrock methods (higher order implicit Runge-Kutta methods)
    \item Predictor-corrector methods
  \end{itemize}
\end{frame}

\begin{frame}
  \frametitle{Summary}
  \begin{itemize}
    \item Several solution methods and their derivation were discussed:
    \begin{itemize}
      \item Explicit solution methods: Euler, Improved Euler, Midpoint method, RK45
      \item Implicit methods: Implicit Euler and Implicit midpoint method
      \item A few examples of their spreadsheet implementation were shown
    \end{itemize}
    \item We have paid attention to accuracy and instability, rate of convergence and step size
    \item Systems of ODEs can be solved by the same algorithms. Stiff problems should be treated with care.
    \item An example of solving ODEs with Matlab was demonstrated.
  \end{itemize}
\end{frame}

\end{document}


% References
% http://ocw.mit.edu/courses/electrical-engineering-and-computer-science/6-00sc-introduction-to-computer-science-and-programming-spring-2011/unit-1/lecture-1-introduction-to-6.00/
% http://www.greenteapress.com/thinkpython/html/thinkpython002.html
% https://www.youtube.com/channel/UCLMQ21H2ad95faYG3yGCwYA
%http://stackoverflow.com/questions/4227145/in-matlab-are-variables-really-double-precision-by-default
%http://www.exploringbinary.com/why-0-point-1-does-not-exist-in-floating-point/

