
\usepackage{listings}
\usepackage{multimedia} % Movies
\usepackage{fancyvrb,relsize}
\usepackage{commath}
\usepackage{graphicx}
\usepackage{array}
\usepackage{longtable}
\usepackage{algpseudocode} 
\usepackage{multirow}
\usepackage[math]{iwona}
\usepackage{wasysym}
\usepackage{amsmath}
\usepackage{amssymb}
\usepackage{siunitx}
\usepackage{tikz} % Drawing
\usepackage{pgfplots}
\usepackage{pgfplotstable}

% Presentation settings
% \rowcolors[]{1}{maincolor!20}{maincolor!10}

\newcommand{\Fo}{\ensuremath{\mathit{Fo}}}

\lstset{language=Matlab,%
    %basicstyle=\color{red},
    basicstyle=\scriptsize\ttfamily,
    breaklines=true,%
    morekeywords={matlab2tikz},
    keywordstyle=\color{blue},%
    morekeywords=[2]{1}, keywordstyle=[2]{\color{black}},
    identifierstyle=\color{black},%
    stringstyle=\color{mylilas},
    commentstyle=\color{mygreen},%
    showstringspaces=false,%without this there will be a symbol in the places where there is a space
    numbers=none,%
%     numberstyle={\tiny \color{black}},% size of the numbers
%     numbersep=-2pt, % this defines how far the numbers are from the text
%     emph=[1]{for,end,break},emphstyle=[1]\color{red}, %some words to emphasise
emph=[2]{ones,int,str2double,long,single,simplify,diff,log,atan,solve,vpa,syms,doc,int,simplify,diff,log,atan,syms,interp3,interpn,histogram,ribbon,contourf,fzero,feval,fminsearch,fsolve,fminbnd,ezplot,varargin,optimset,odeset,ode15s,plotyy,ones,linprog,cftool,optimset,lsqnonlin}, emphstyle=[2]{\color{blue}},
    backgroundcolor=\color{gray!15},frame=tlbr, framerule=0pt,
    escapeinside={(*@}{@*)}
}

% To have the navigation circles without declaring subsections
\usepackage{remreset}% tiny package containing just the \@removefromreset command
\makeatletter
\@removefromreset{subsection}{section}
\makeatother
\setcounter{subsection}{1}

% For convenient figure inclusion
\DeclareGraphicsExtensions{.pdf,.png,.jpg}
\graphicspath{ {../img/} }

\setbeamertemplate{footline}[frame number]

% TU/e colors
\definecolor{tuered}{RGB}{247,49,49}
\definecolor{tuefuchsia}{RGB}{214,0,74}
\definecolor{tuelila}{RGB}{214,0,123}
\definecolor{tuepurple}{RGB}{173,32,173}
\definecolor{tuedblue}{RGB}{16,16,115}
\definecolor{tueblue}{RGB}{0,102,204}
\definecolor{tuelblue}{RGB}{0,162,222}
\definecolor{tueorange}{RGB}{255,154,0}
\definecolor{tueyellow}{RGB}{255,221,0}
\definecolor{tuedyellow}{RGB}{206,223,0}
\definecolor{tuegreen}{RGB}{132,210,0}
\definecolor{tuedgreen}{RGB}{0,172,130}
\definecolor{tueblue2}{RGB}{0,146,181}

% For Matlab script colors
\definecolor{mygreen}{RGB}{28,172,0} % color values Red, Green, Blue
\definecolor{mylilas}{RGB}{170,55,241}

\makeatletter
% \definecolor{beamer@blendedblue}{rgb}{0.5,0.5,0.3} % changed this
\useoutertheme{smoothbars}
\useinnertheme{circles}

%%%%%%%%%%%%%%%%%%%%%%%%%%%%%%%%%%%%%%%%%%%%%%%%%%%%%%%%%%%%%%%%%%%%%%%%%%%
\definecolor{maincolor}{named}{tuelblue}
\definecolor{textcolorfg}{named}{white}
\definecolor{tuealert}{named}{tueblue}
%%%%%%%%%%%%%%%%%%%%%%%%%%%%%%%%%%%%%%%%%%%%%%%%%%%%%%%%%%%%%%%%%%%%%%%%%%%

\setbeamercolor{normal text}{fg=black,bg=white}
\setbeamercolor{alerted text}{fg=tuealert}
\setbeamercolor{example text}{fg=tuegreen!50!black}

\setbeamercolor{background canvas}{parent=normal text,bg=white}
\setbeamercolor{background}{parent=background canvas}

\setbeamercolor{title}{bg=maincolor,fg=textcolorfg} % Presentation title colors
\setbeamercolor{structure}{fg=maincolor,bg=textcolorfg}
\setbeamercolor{section in head/foot}{fg=textcolorfg,bg=maincolor}
\setbeamercolor{palette primary}{fg=textcolorfg,bg=maincolor} % changed this

\setbeamercolor{palette primary}{fg=maincolor,bg=textcolorfg} % changed this
\setbeamercolor{palette secondary}{use=structure,fg=structure.fg!100!tueblue} % changed this
\setbeamercolor{palette tertiary}{use=structure,fg=structure.fg!100!tuered} % changed this

\setbeamertemplate{navigation symbols}{} % ( Dont use )
\setbeamercolor{navigation symbols}{use=structure,fg=structure.fg!40!bg}
\setbeamercolor{navigation symbols dimmed}{use=structure,fg=structure.fg!20!bg}

\setbeamercolor{block title}{fg=textcolorfg,bg=maincolor}
\setbeamercolor{block body}{fg=black,bg=maincolor!10}

\def\colorize<#1>{%
  \temporal<#1>{\color{tuedblue!40!gray!40}}{\color{tuealert}}{\color{black}}}
  
% \setlength{\mathindent}{0pt}

\makeatother

% Colored urls
\hypersetup{colorlinks,linkcolor=,urlcolor=tueblue}

% Vector format
\renewcommand{\vec}[1]{\boldsymbol{#1}}
\renewcommand{\det}[1]{\text{det}\begin{vmatrix}#1\end{vmatrix}}

\usetikzlibrary{decorations} % Drawing
\usetikzlibrary{patterns}
\usetikzlibrary{positioning}
\usetikzlibrary{shadows}
\usetikzlibrary{snakes}
\usetikzlibrary{calc}
\usetikzlibrary{arrows}
\usetikzlibrary{snakes}
\usetikzlibrary{fit}
\usetikzlibrary{fadings}
\usetikzlibrary{matrix}
\usetikzlibrary{plotmarks}
\usetikzlibrary{shapes}
\usetikzlibrary{shadings}
\usetikzlibrary{intersections}
% Blocks
\tikzset{block/.style={rectangle,draw=maincolor,fill=maincolor!20,text width=10em,text centered,rounded corners,minimum height=4em,thick}}
\tikzset{emphblock/.style={rectangle,draw=maincolor,text centered,rounded corners,thick,top color=maincolor!10,bottom color=maincolor!30}}
\tikzset{emphblocko/.style={rectangle,draw=tueorange,text centered,rounded corners,thick,top color=tueorange!10,bottom color=tueorange!30}}
\tikzset{emphblocky/.style={rectangle,draw=tueyellow,text centered,rounded corners,thick,top color=tueyellow!10,bottom color=tueyellow!30}}
% Dots
\tikzset{dot/.style={draw=tuered,circle,thick,minimum size=1mm,inner sep=0pt,outer sep=0pt,fill=white}}
\tikzset{fdot/.style={circle,draw=black,fill=black,,inner sep=1.5pt}}
\tikzset{gdot/.style={circle,draw=black,inner sep=3pt}}
\tikzset{cross/.style={cross out, draw=black, fill=none, minimum size=2*(#1-\pgflinewidth), inner sep=0pt, outer sep=0pt}, cross/.default={4pt}}
% Graphs and lines
\tikzset{line/.style={black,>=stealth',semithick}}
\tikzset{graph/.style={smooth,samples=400,tuered,semithick}}
\tikzset{interp/.style={dot,draw=tuealert,inner sep=1.5pt,minimum size=4pt,color=tuealert,fill=none}}
\tikzset{intblock/.style={line,draw=tuefuchsia,fill=tuefuchsia!50!white,fill opacity=0.3,opacity=0.6}}
\tikzset{intdot/.style={line,dot,draw=tuefuchsia,fill=tuefuchsia,opacity=0.6}}
\tikzset{gridline/.style={lightgray,ultra thin,dashed}}
\newcommand{\tikzmark}[2]{\tikz[overlay,remember picture,
  baseline=(#1.base)] \node (#1) {#2};}

  \tikzset{style green/.style={
    set fill color=tuegreen!60,
    set border color=white,
  },
  style cyan/.style={
    set fill color=tuelblue!60,
    set border color=white,
  },
  style orange/.style={
    set fill color=tueorange!60,
    set border color=white,
  },
  style yellow/.style={
    set fill color=tueyellow!60,
    set border color=white,
  },
  mat/.style={
    above left offset={-0.15,0.38},
    below right offset={0.15,-0.125},
    #1
  },
  mat2/.style={
    above left offset={-0.15,0.31},
    below right offset={0.15,-0.125},
    #1
  },
  txt/.style={
    above left offset={-0.1,0.34},
    below right offset={0.15,-0.15},
    #1
  }
}

\newcolumntype{L}[1]{>{\raggedright\arraybackslash}p{#1}}
\newcolumntype{R}[1]{>{\raggedleft\arraybackslash}p{#1}}

% \pgfplotsset{
% % every axis y label/.append style={at={axis cs:14,14},rotate=0,anchor=south east}
% exery axis/.style={ylabel near ticks},
% }
\tikzset{
  invisible/.style={opacity=0},
  visible on/.style={alt={#1{}{invisible}}},
  alt/.code args={<#1>#2#3}{%
    \alt<#1>{\pgfkeysalso{#2}}{\pgfkeysalso{#3}} % \pgfkeysalso doesn't change the path
  },
}

\renewcommand*\familydefault{\sfdefault}  % Use sans font